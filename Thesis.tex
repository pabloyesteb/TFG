\documentclass[a4paper,11pt,twoside,makeidx]{ThesisStyle}
\usepackage[english]{babel}   % quita el punto de la section
\usepackage[latin1]{inputenc}
%\usepackage[T1]{fontenc}
%   \decimalpoint
\usepackage{color}
\usepackage{colortbl}
\usepackage{graphicx}
\usepackage{epic}
\usepackage{curves}
\usepackage{multicol}
%\usepackage{array}
\usepackage{longtable}
\usepackage{amstext}
\usepackage{amsbsy}
\usepackage{amsthm}
\usepackage{amsmath}
\usepackage{amssymb}
\usepackage{latexsym}  %\mathbb{R}
\usepackage{wrapfig}
%\usepackage{multind}
\usepackage{scalerel}
\usepackage{stackengine,wasysym}

\newcommand\reallywidetilde[1]{\ThisStyle{%
  \setbox0=\hbox{$\SavedStyle#1$}%
  \stackengine{-.1\LMpt}{$\SavedStyle#1$}{%
    \stretchto{\scaleto{\SavedStyle\mkern.2mu\AC}{.5150\wd0}}{.6\ht0}%
  }{O}{c}{F}{T}{S}%
}}
\usepackage{eurosym}
\usepackage{setspace}
\usepackage{makeidx}
\usepackage{textpos}
\usepackage{pgfplotstable,booktabs,siunitx,array}
\usepackage{import}
\usepackage{rotating}    % Sideways of figures & tables
\usepackage{multirow}
%\usepackage[FIGTOPCAP]{subfigure}
\usepackage[backend=biber,
    style=numeric-comp, hyperref, giveninits=true, backref = true]{biblatex}
\usepackage[hyperindex=true]{hyperref}
%\usepackage[pagebackref,hyperindex=true]{hyperref}
\usepackage[section]{placeins}
\usepackage{subcaption}
\usepackage{titlesec}
\usepackage[intoc, english]{nomencl}
\makenomenclature
\newcommand{\nomunit}[1]{%
\renewcommand{\nomentryend}{\hspace*{\fill}#1}}

%% This code creates the groups
% -----------------------------------------
\usepackage{ifthen}
  \renewcommand{\nomgroup}[1]{%
  \item[\bfseries
  \ifthenelse{\equal{#1}{C}}{Physical constants}{%
  \ifthenelse{\equal{#1}{N}}{Characteristic numbers}{%
  \ifthenelse{\equal{#1}{T}}{Orbital elements}{%
  \ifthenelse{\equal{#1}{G}}{Cartesian coordinates}{%
  \ifthenelse{\equal{#1}{D}}{ }{%
  \ifthenelse{\equal{#1}{S}}{Suffixes}{}}}}}}%
  ]}

% -----------------------------------------

\setcounter{secnumdepth}{3}
\captionsetup[subfigure]{subrefformat=simple,labelformat=simple}
\renewcommand\thesubfigure{(\alph{subfigure})}

\captionsetup[table]{name=Tabla}

% float para las figuras
\def\floatpagefraction{.8}

\DeclareGraphicsExtensions{.pdf,.png,.jpg} %solo para PDFLaTeX

%\setlength{\unitlength}{0.18mm}

%------------------------------------------------------------------------------------------------
%-------------- tama�o folio -------------------------------------------------------------------

\def\psnormal{\textwidth=19.5cm\textheight=23cm
          \oddsidemargin=-1.5cm\evensidemargin=0.2cm
          \parindent=1cm\topmargin=-2cm}
\psnormal

%------------------------------------------------------------------------------
% new commands
%------------------------------------------------------------------------------
\newcommand{\bm}[1]{\mbox{\boldmath{${#1}$}}} % negrita en modo matematico
%\newcommand{\dfrac}{\displaystyle\frac}
\newcommand{\dint}{\displaystyle\int}
\newcommand{\dsum}{\displaystyle\sum}
\newcommand{\tsum}{\textstyle\sum}
\newcommand{\dprod}{\displaystyle\prod}
\newcommand{\dlim}{\displaystyle\lim}
\newcommand{\eg}{{\it e.g.}\ }
\newcommand{\ie}{{\it i.e.}\ }
\newcommand{\ea}{\mbox{\it et al.}\ }
\renewcommand{\labelitemi}{$\bullet$}
\renewcommand{\labelitemii}{--}
\renewcommand{\labelitemiii}{$\spadesuit$}

% negrita en modo matematico

\newcommand{\be}{\begin{equation}}
% $$ 
\newcommand{\ee}{\end{equation}}

\newcommand{\caja}[2]{
\begin{center}
\begin{tabular}{|p{#1}|}
\hline
\vspace{#2}\\
\hline
\end{tabular}
\end{center}
}

%-----------------------------------------------------------------------------------------------
\definecolor{Blue}          {cmyk}{1,1,0,0}            % palabras clave
\definecolor{BrickRed}      {cmyk}{0,1.0,0.91,0.60}    % texto resaltado
\definecolor{PineGreen}     {cmyk}{0.92,0,0.59,0.25}   % comentarios programa
\definecolor{CornflowerBlue}{cmyk}{0.65,0.13,0,0}
\definecolor{SkyBlue}       {cmyk}{0.62,0,0.12,0}
\definecolor{Melon}         {cmyk}{0,0.46,0.50,0}
\definecolor{Apricot}       {cmyk}{0,0.32,0.52,0}
\definecolor{GreenYellow}   {cmyk}{0.15,0,0.69,0}

\definecolor{CadetBlue}     {cmyk}{0.62,0.57,0.23,0}
\definecolor{CornflowerBlue}{cmyk}{0.65,0.13,0,0}
\definecolor{MidnightBlue}  {cmyk}{0.98,0.13,0,0.43}
\definecolor{NavyBlue}      {cmyk}{0.94,0.54,0,0}
\definecolor{RoyalBlue}     {cmyk}{1,0.50,0,0}

\definecolor{verde}{cmyk}{0.40,0.0,0.32,0.70}
\definecolor{azul}{cmyk}{1.0,0.78,0.0,0.18}
\definecolor{linkcol}{rgb}{0,0,0.4}
\definecolor{citecol}{rgb}{0.5,0,0}
\definecolor{Skyblue}{cmyk}{0.4,0.1,0.0,0.0}
\newcommand{\BRnem}[1]{\normalsize\color{BrickRed}\em #1}
\newcommand{\BRlbf}[1]{\Large\color{BrickRed}\rmfamily\bfseries #1}
\newtheorem{prop}{Property}
\newtheorem{definition}{Definition}
\newenvironment{Figure}
  {\par\medskip\noindent\minipage{\linewidth}}
  {\endminipage\par\medskip}
%------------------------------------------------------------------------------
\def\floatpagefraction{.8}
\DeclareGraphicsExtensions{.pdf,.png,.jpg} % solo para PDFLaTeX
\def\baselinestretch{1.5}
%------------------------------------------------------------------------------

%------------------------------------------------------------------------------------------------
%-------------- tama�o folio -------------------------------------------------------------------

\def\psnormal{\textwidth=17cm\textheight=23cm
          \oddsidemargin=0cm\evensidemargin=0.2cm
          \parindent=1cm\topmargin=-1cm}
\psnormal
%------------------------------------------------------------------------------------------------

%------------------------------------------------------------------------------
%------------------------ Fancy page ------------------------------------------
\usepackage{fancyhdr}                    % Fancy Header and Footer

\pagestyle{fancy}                       % Sets fancy header and footer
\fancyfoot{}                            % Delete current footer settings

\fancyhead[LE,RO]{\bfseries\thepage}    % Page number (boldface) in left on even
% pages and right on odd pages
\fancyhead[RE]{\bfseries\nouppercase{\leftmark}}      % Chapter in the right on even pages
\fancyhead[LO]{\bfseries\nouppercase{\rightmark}}     % Section in the left on odd pages

\let\headruleORIG\headrule
\renewcommand{\headrule}{\color{black} \headruleORIG}
\renewcommand{\headrulewidth}{1.0pt}
\usepackage{colortbl}
\arrayrulecolor{black}

\fancypagestyle{plain}{
  \fancyhead{}
  \fancyfoot{}
  \renewcommand{\headrulewidth}{0pt}
}
%------------------------------------------------------------------------------


%------------------------------------------------------------------------------
%------------------------ Minitoc ---------------------------------------------
\usepackage{minitoc}
%\usepackage[tight,spanish]{minitoc}
%\usepackage[tight]{minitoc}

\setcounter{minitocdepth}{2}         %default
\setlength{\mtcindent}{24pt}         %default
\renewcommand{\mtcfont}{\small\rm}   %default
\renewcommand{\mtcSfont}{\small\bf}  %default
\mtcsetoffset{minitoc}{-2.0em}		% indent del mimnitoc
%------------------------------------------------------------------------------
%----------------------- Estilo de subrutina y función ------------------------
\usepackage{listings}
%---------------LO fundamental-------------------------
\lstset{basicstyle=\ttfamily\small,
        identifierstyle=\ttfamily,
        morekeywords={save}
        }

\lstdefinestyle{interface}
               {language=[95]Fortran,
                %frame=bottomlines,
                basicstyle=\ttfamily,
                identifierstyle=\ttfamily,
                %commentstyle=\ttfamily,
                %commentstyle=\color{PineGreen},
                %keywordstyle=\color{azul},
                commentstyle=\color{ao(english)},
                keywordstyle=\color{blue},
                emph={status,size,position,len},
                emphstyle=\em,
				morecomment=[l][\em]{Dec},
				morecomment=[l][\em]{sent},
				morecomment=[l][\em]{nombre},
				morecomment=[l][\em]{operando},
				morecomment=[l][\em]{operador},
                emph={nombre_variable}
               }
%--------------------------------------------------------------------------

\lstdefinestyle{interfaceNF}
               {basicstyle=\ttfamily\small,
                identifierstyle=\ttfamily,
                emphstyle=\ttfamily
               }

\lstdefinestyle{ejemplos}
               {basicstyle=\ttfamily,
                keywordstyle=\em,
                commentstyle=\ttfamily,
				morecomment=[l][\em]{Dec},
				morecomment=[l][\em]{sent},
                keywords={numero,estructura,componente}
               }

\lstdefinestyle{otros}
               {basicstyle=\ttfamily,
                keywordstyle=\em,
                commentstyle=\ttfamily,
%				morecomment=[l][\em]{Dec},
%				morecomment=[l][\em]{sent},
%                keywords={numero,estructura,componente}
               }

\lstdefinestyle{ejemplosNF}
               {basicstyle=\ttfamily,
                keywordstyle=\em,
                commentstyle=\ttfamily,
				morecomment=[l][\em]{Dec},
				morecomment=[l][\em]{sent},
                keywords={numero,estructura,componente,sentencia,expr,lógica,%
				          lista_de_variables,nombre,cuerpo_de_interface, %
						  nombre_variable, nombre_funcion, arg}
               }

\lstdefinestyle{numbered}
               {numbers=left,
                numberstyle=\tiny,
                escapeinside={(*@}{@*)},
%                language=[95]Fortran,
                basicstyle=\ttfamily,
%                identifierstyle=\ttfamily,
%                commentstyle=\ttfamily,
%                emph={status,size,position,len},
%                emphstyle=\ttfamily,
%				morecomment=[l][\em]{Dec},
%				morecomment=[l][\em]{sent},
               }

\lstdefinestyle{ejemplosubp}
               {basicstyle=\ttfamily,
                keywordstyle=\em,
                commentstyle=\ttfamily,
				morecomment=[l][\em]{Dec},
				morecomment=[l][\em]{sent},
                keywords={nombre_subrutina, lista_argumentos, %
				  nombre_del_module, cuerpo_de_interface, lista_de_subprgramas}
               }

\lstdefinestyle{ficheros}
               {basicstyle=\ttfamily,
                keywordstyle=\em,
                commentstyle=\ttfamily,
                keywords={nombre,estado,accion,lista_vbles, nombre_del_namelist}
               }

\lstdefinestyle{formatos}
               {basicstyle=\ttfamily,
                keywordstyle=\em,
                commentstyle=\ttfamily,
                keywords={nombre}
               }

\lstdefinestyle{parada}
               {basicstyle=\ttfamily,
                keywordstyle=\em,
                commentstyle=\ttfamily,
                keywords={frase_a_imprimir_por_pantalla}
               }

%------------------------------------------------------------------------------

%-------------------------------------------------------------------------------


\usepackage{hyperref}
%------------------------------------------------------------------------------
%---------------- para el Indice ----------------------------------------------
\hypersetup
{
bookmarksopen=true,
pdftitle=Advanced STM computation methods,
pdfauthor="Rodrigo_Fdez",
pdfsubject="a subject", 		% subject of the document
%pdftoolbar=false, 				% toolbar hidden
pdfmenubar=true, 				% menubar shown
pdfhighlight=/O,           		% effect of clicking on a link
colorlinks=true, 				% couleurs sur les liens hypertextes
pdfpagemode=None, 				% aucun mode de page
pdfpagelayout=SinglePage, 		% ouverture en simple page
pdffitwindow=true,              % pages ouvertes entierement dans toute la fenetre
linkcolor=linkcol, 				% couleur des liens hypertextes internes
citecolor=citecol, 				% couleur des liens pour les citations
urlcolor=linkcol 				% couleur des liens pour les url
}
%------------------------------------------------------------------------------

\makeindex

\addbibresource{refs.bib}

\begin{document}


\frontmatter

											% Compila parametros de la portada
%------------------------------------------
% portada
%------------------------------------------

% Genera portada
\renewcommand{\title}{Nombre del trabajo}
\renewcommand{\author}{Autor}

\renewcommand\maketitle{
\begin{titlepage}
\begin{center}  
	% --- --- --- --- Vertical offset --- --- --- --- --- 
   	\vspace*{-16mm}
   	\hspace{-18mm}
  	% --- --- --- --- THESIS TITLE --- --- --- --- --- 
    \begin{center}
    \rule{14cm}{1pt}\\
    \begin{huge}
    \textbf{Validation Methods for} \\
    \textbf{Industrial Machine Learning Problems}
    \rule{14cm}{1pt}\\[5mm]
    \end{huge}
    % --- --- --- --- INSTITUTION --- --- --- --- --- 
   	{	Trabajo Fin de Grado \\[7mm]
   	%
 		Grado en Ingenier�a Aeroespacial \\[7mm]
 	%
 		Escuela T�cnica Superior de Ingenier�a \\[0mm]
   		Aeron�utica y del Espacio\\[7mm]
	%   	
   		Universidad Polit�cnica de Madrid} \\[7mm]
   	% --- --- --- --- LOGOS --- --- --- --- --- 
    \begin{figure}[ht]
	\centering
	\medskip
	\begin{subfigure}[t]{0.25\linewidth}
	\centering\includegraphics[width=\linewidth]{Figures/logos/logo_upm}
	\end{subfigure}
	\begin{subfigure}[t]{.25\linewidth}
	\centering\includegraphics[width=\linewidth]{Figures/logos/logo_etsiae}
	\end{subfigure}
	\end{figure}
	\end{center}

	% --- --- --- --- AUTHOR & TUTORS --- --- --- --- ---
	\vfill % Alinear abajo
	\begin{flushright}      
	      { \textit{Autor}: Pablo Yeste Blesa} \\[3mm]
	      { \textit{Tutor acad�mico}: Javier de Vicente Buend�a}\\[3mm]
	\end{flushright}      
	% --- --- --- --- DATE --- --- --- --- --- 
    \begin{center}
      Marzo 2024
    \end{center}
\end{center}
\end{titlepage}
}


\maketitle												% Compila parametros de la portada


%\chapter{Agradecimientos.}
%%
%\minitoc
%%
%\indent Personalmente, este trabajo lo he vivido como una de esas monta�as, que te impone al principio, que se escalan paso a paso sin mirar d�nde est� la cima, y que cuando est�s cerca, y parece que ya ha terminado, aparece la �ltima subida. Es un desaf�o que disfrutas simplemente dejando la mente en blanco y confiando paso a paso en que llegar�s, y que, en el momento de llegar arriba, su inmensidad te hace dudar de que lo hayas conseguido, y te hace apreciar ese momento en el que decidiste comenzarla, con m�s ganas y fe que conocimiento de lo que te esperaba. Y el orgullo que sientes en ese momento s�lo es comparable al agradecimiento que sientes por haber compartido el camino y haber sido guiado por ciertas personas.\\
%%
%\indent En primer lugar, quiero dar las gracias a Mariola por el infinito apoyo, consejos, motivaci�n y pasi�n que ha mostrado por m�, no s�lo durante este TFG, que es lo m�s apreciable, sino tambi�n desde las clases de Estad�stica y de Ampliaci�n de Matem�ticas. Me decid� por el trabajo gracias a ella, y no me he arrepentido en ning�n momento de la decisi�n. Ha sembrado en m� la semilla de la investigaci�n, y espero poder devolv�rselo de alguna manera alg�n d�a. Quiero agradecer tambi�n al Departamento de Matem�ticas (Mancebo, Ignacio Delgado) de la ETSIAE por los excelentes docentes que me han educado desde el primer curso al �ltimo, demostrando un gran apoyo y un gran inter�s por el alumnado. No me puedo olvidar de todos los profesores que me han educado, todos ellos de la escuela p�blica, desde los tres a�os hasta el bachillerato. Cada vez que reflexiono sobre ello, siento un profundo orgullo del camino por el que he llegado hasta aqu�. \\
%%
%\indent A nivel personal, quiero agradecer infinitamente el apoyo de mis padres, �ngeles y Gonzalo, desde peque�o, cuando quer�a ser dise�ador de coches, hasta que entr� en la escuela y, como si nada, acab� la carrera. Es obvio que sin la educaci�n y los valores que me han ense�ado, ni una miga de todo lo que conseguido habr�a sido posible. Por supuesto, quiero tambi�n agradecer al resto de mi familia, en particular a mi hermano, por todo el apoyo y los consejos, especialmente en mi vida universitaria. Tambi�n a mi abuela, aunque piense que ahora soy piloto. A Paula, por traerme esa pizca de calma, alegr�a y felicidad que hace que todo funcione a las mil maravillas, m�s en esta �poca que nos toc� vivir. No me olvido de mis amigos, los de Bembibre y los de Madrid, que de m�s de uno y de dos aprietos me han sacado a rastras, y sin los cuales, a saber donde estar�a. \\
%%
%\begin{flushright}
%Rodrigo Fern�ndez\\
%10 de julio de 2020
%\end{flushright}

\chapter{Resumen.}
%
%
\indent Son numerosos los incidentes y accidentes que, a�o tras a�o, est�n asociados a la formaci�n de hielo en las aeronaves durante su vuelo. Este fen�meno, conocido como engelamiento, se produce por el impacto de gotas de agua en estado subenfriado (a una temperatura inferior al punto de fusi�n) sobre la aeronave, ya sea por precipitaci�n, como es el caso de la lluvia engelante, o por atravesar una nube con una cantidad suficiente de estas gotas. Sea como fuere, los efectos sobre la aeronave son cr�ticos, y pueden aparecer en cuesti�n de pocos minutos. Esto obliga a adquirir la capacidad de (a) preveer, (b) detectar y (c) proteger o remediar la formaci�n de hielo. \\
%
\indent A fin de comprender y cuantificar los efectos plasmados sobre el desempe�o de la aeronave de este fen�meno, es necesario contar con un modelo f�sico que represente correctamente los mecanismos que lo producen. Pese a los numerosos esfuerzos, a d�a de hoy la precisi�n de los resultados no es la que cabr�a desear, distando de una manera apreciable de los resultados obtenidos mediante experimentaci�n. Un aspecto com�n a todos los modelos es el enfoque modular, interaccionando tres modelos: la obtenci�n del campo fluido por el flujo del aire sobre la aeronave (a trav�s de m�todos CFD), la predicci�n de las trayectorias de las gotas de agua, y el an�lisis termodin�mico del propio fen�meno de la formaci�n y crecimiento de hielo. Una correcta compenetraci�n entre ambos es imprescindible para simular de manera precisa el engelamiento en aeronaves. \\
%
\indent En este proyecto, el principal objeto de inter�s es el desarrollo de un modelo termodin�mico de la formaci�n de hielo basado en ecuaciones diferenciales, partiendo desde la misma modelizaci�n de cada fen�meno f�sico que tiene lugar, pasando por la discretizaci�n del modelo mediante el m�todo de vol�menes finitos, y acabando con una simulaci�n del engelamiento bajo unas condiciones dadas.
%
\chapter{Abstract.}
%
\indent A remarkable number of aircraft incidents and accidents are related each year with in-flight ice accretion. This phenomenon, called in short icing, is due to the impact of supercooled water droplets (i. e. which temperature is below the dew point) on a certain part of the aircraft, in particular aerodynamic surfaces. This causes an almost instant performance decline, severely damaging aerodynamic attributes such as stability or drag, but also jeopardizing the operation of the engines. In order to reduce the potential danger of this meteorological hazard, it is necessary to (a) predict, (b) detect and (c) protect and mend.\\
%
\indent So as to understand and evaluate the effects regarding the aircraft performance as a consequence of icing, it is vital to develop a physical model which accurately portraits the procedures that ice accretion involves. Although huge and numerous efforts have been made, the correlation between the numerical results and experimentation is not as close as wished, and more work is necessary to crack the code of this phenomenon. One thing in common of all models is their modular approach, as there are three major blocks which interact with almost no exclusion. Those are (a) the numerical solution of the aerodynamic airflow around the aircraft, (b) the prediction of the water droplets' trajectories and (c) the thermodynamic ice accretion proccess simulation. A proper rapport between the three of them is mandatory in order to obtain realistic results.\\ 
%
\indent The main focus of this project is to develop a PDE (partial differential equation) based thermodynamic module, involving the physical modeling of the proccess, its discretization and its simulation under a certain set of conditions.


\dominitoc

\cleardoublepage

\tableofcontents
\printnomenclature
\listoffigures
\listoftables
\chapter{Introducci�n.}
%
\indent El engelamiento es un fen�meno meteorol�gico causado esencialmente por el impacto de gotas en estado subenfriado, es decir, a una temperatura inferior al punto de fusi�n. Se trata de un estado metaestable, en el que una perturbaci�n en la presi�n (por ejemplo mediante el impacto sobre una aeronave) es suficiente para inducir su cambio de fase, abocando a la formaci�n de hielo sobre la aeronave. Los efectos causados por este fen�meno pueden ser cr�ticos, al da�ar seriamente el comportamiento aerodin�mico del avi�n, el funcionamiento de los motores y de ciertos instrumentos. Es un fen�meno que causa un gran n�mero de incidentes y accidentes a�reos al a�o, con una mayor incidencia en la aviaci�n general, debido a su mayor precariedad en t�rminos de equipamiento en relaci�n a las aeronaves comerciales.\\
%
\indent A fin de reducir los perjuicios del engelamiento, es necesario afrontarlo desde varios puntos. En primer lugar, la detecci�n del hielo formado, lo cual puede realizarse \textit{in situ} por el propio piloto o mediante intrumentos de detecci�n. El segundo frente por el que avanzar es la predicci�n de las condiciones meteorol�gicas que inducen el engelamiento, lo cual parte de la base de que, actualmente, no existen datos expl�citos acerca de ello, sino m�s bien correlaciones emp�ricas con variables con otros fines. El desarrollo de modelos num�ricos meteorol�gicos junto a la reinterpretaci�n de datos de sat�lites son la punta de lanza de esta t�ctica contra el engelamiento. \\
%
\indent No obstante, las t�cnicas m�s pragm�ticas contra la formaci�n de hielo son las defensas anti-hielo. Estas se dividen en dos tipos generales: las meteorol�gicas, centradas en c�mo evitar las condiciones que lo propicien y c�mo actuar en caso de encontrarse en ellas, y las t�cnicas, centradas en la eliminaci�n o prevenci�n de hielo mediante dispositivos que act�en \textit{in situ}. Estas �ltimas son numerosas y variadas, distingui�ndose las siguientes:
%
\begin{itemize}
\item \underline{Medidas mec�nicas:} Mediante la modificaci�n de la forma del perfil, aprovechan la fragilidad del hielo para desprenderlo. Requieren de un espesor m�nimo de hielo formado, lo que supone una desventaja frente al resto.
\item \underline{Medidas t�rmicas:} Funden el hielo mediante la utilizaci�n de diversos sistemas calefactores. Funcionan como prevenci�n y como remedio.
\item \underline{Medidas qu�micas:} Basadas en rociar las superficies expuestas a la formaci�n de hielo con l�quidos anticongelantes.
\end{itemize}
%
\indent Todo lo relativo al engelamiento como fen�meno meteorol�gico, as� como las formas de proteger una aeronave ante �l, se encuentra desarrollado en el cap�tulo 1.\\
%
\indent La simulaci�n de la formaci�n de hielo requiere de la correcta interacci�n de tres m�dulos: un solver CFD, que proporcione la soluci�n del flujo aerodin�mico, un calculador de trayectorias de las gotas de agua, y un modelo termodin�mico de la formaci�n de hielo. La bibliograf�a es amplia en lo relativo a los c�digos CFD, con muy diversas aproximaciones de la mec�nica de fluidos, pero en este proyecto en particular se trabaja con datos obtenidos del solver TAU-Code \cite{TAU_code}, desarrollado por la agencia aeroespacial alemana \textit{DLR}. En cuanto a la simulaci�n de las trayectorias de las gotas de agua, existen dos enfoques f�sicos bastante distintos: el lagrangiano, centrado en la din�mica de cada gota, y el euleriano, que analiza en su lugar vol�menes de control y variables fluidas. El m�dulo utilizado en este proyecto es el m�dulo GOTA, desarrollado en la misma ETSIAE en a�os anteriores. Como se puede preveer, la tarea principal es por tanto desarrollar un m�dulo termodin�mico, basado en un enfoque diferencial mediante la aplicaci�n de las ecuaciones de conservaci�n de la mec�nica de fluidos. La informaci�n relativa a esta estructura de resoluci�n se introduce en el cap�tulo 2. \\
%
\indent Dado el enfoque planteado para el m�dulo termodin�mico, es conveniente realizar una peque�a introducci�n al modelado y discretizaci�n de sistemas f�sicos mediante ecuaciones diferenciales. Se tratan temas como las leyes de conservaci�n, los tipos de problemas f�sicos, los m�todos m�s habituales para la discretizaci�n de las ecuaciones diferenciales en derivadas parciales, todo ello en el cap�tulo 3. Se trata de una introducci�n que introduce los conceptos b�sicos tratados a la hora de resolver en la pr�ctica cualquier ecuaci�n diferencial, y en particular las ecuaciones de la formaci�n de hielo. La discretizaci�n espacial y temporal de las ecuaciones toma un papel importante, como no pod�a ser de otra manera. Como complemento, el cap�tulo 4 re�ne una colecci�n de casos crecientes en dificultad de aplicaciones del m�todo de los vol�menes finitos, en especial del m�todo de Godunov. Se comienza por las ecuaciones de la advecci�n y de Burgers, de las cuales es conocida la soluci�n anal�tica, para pasar a la implementaci�n del esquema de Roe para la resoluci�n num�rica de las ecuaciones de Euler de la mec�nica de fluidos. Adicionalmente, se puede encontrar un escueto desarrollo de la soluci�n anal�tica de estas ecuaciones en el ap�ndice B. En conclusi�n, los cap�tulos 3 y 4 act�an como la base del an�lisis desarrollado en el cap�tulo 5.\\
%
\indent En el �ltimo cap�tulo del trabajo se afronta la dif�cil tarea de la modelizaci�n, discretizaci�n e implementaci�n de las ecuaciones de la formaci�n de hielo. En primer lugar, se realiza un riguroso an�lisis de las ecuaciones integrales de la mec�nica de fluidos, particularizando para las condiciones que distinguen al problema. Una vez se han alcanzado unas ecuaciones manejables y adecuadas para la discretizaci�n mediante vol�menes finitos, se procede a ello, consiguiendo una base para el esquema num�rico final. A continuaci�n, se pormenorizan los modelos de cada uno de los fen�menos termodin�micos que aparecen en las ecuaciones, y se procede a su discretizaci�n. En �ltima instancia, se plantea la resoluci�n num�rica de las ecuaciones con ayuda de ciertas restricciones termodin�micas, acabando en unas ecuaciones discretizadas expl�citas que permiten resolver el problema, dado un conjunto de datos proveniente del solver CFD TAU y del m�dulo GOTA. Para los test de validaci�n desarrollados a posteriori, se utilizan en su lugar estimaciones de estos valores, en virtud de resultados ya existentes en la bibliograf�a. \\
%
\indent Adicionalmente, el anexo A presenta una visi�n general sobre la consistencia, estabilidad y precisi�n de esquemas num�ricos, algo que resulta muy interesante ya que al fin y al cabo se conf�a constantemente en que los esquemas y la discretizaci�n permitan obtener unos resultados cercanos a la realidad. Cada uno de los tres aspectos es presentado, caracterizado y ejemplificado de manera sencilla, de modo que se pueda obtener una visi�n general de ellos.\\
%
\indent A modo de resumen, los objetivos principales de este proyecto son:
%
\begin{itemize}
\item Establecer una visi�n general del fen�meno del engelamiento, a nivel f�sico y matem�tico.
\item Plantear los conceptos b�sicos de la discretizaci�n de ecuaciones diferenciales y sus caracer�sticas.
\item Desarrollar un modelo discreto de la termodin�mica de la formaci�n de hielo.
\end{itemize}
\mainmatter
%\chapter{Engelamiento en aeronaves.}
%
\label{chap: Engelamiento}
%
\section{Introducci�n.}
%
\indent El engelamiento es un fen�meno que consiste en la formaci�n de hielo por impacto de cristales de hielo y/o gotas de agua en estado subenfriado. Se trata de un fen�meno meteorol�gico que afecta a muchos sistemas, tales como aerogeneradores, pero su principal v�ctima son las aeronaves. El cuadro de perjuicios que el engelamiento induce en los aviones es de tal gravedad que es una de las mayores causas de siniestralidad en aeronaves. Solamente en Estados Unidos, se reportan alrededor de 20 fatalidades relacionadas con el engelamiento cada a�o \cite{Icing_accidents}, con un desenlace tr�gico en un porcentaje razonable de los mismos, especialmente en aviaci�n general, que no cuentan con las defensas ni con informaci�n tan precisa con que trabaja la aviaci�n comercial. \\
%
\indent La fenomenolog�a del engelamiento reviste una enorme complejidad en varios niveles. En primer lugar, la propia simulaci�n de la formaci�n de hielo en el ala, si bien ha progresado de manera apreciable, a d�a de hoy no presenta la precisi�n adecuada para representar ciertas condiciones, pese al �xito en otras. En segundo lugar, la predicci�n meteorol�gica de condiciones que propicien el engelamiento no est� muy avanzada, aunque se est�n redoblando esfuerzos en este aspecto. En numerosas ocasiones, el engelamiento se detecta cuando ya est� sucediendo, y ha de ser remediado o bien huyendo de la zona o bien con m�todos antihielo. Un fuerte esfuerzo est� siendo realizado en este respecto, con el ensayo de nuevas tecnolog�as m�s efectivas y eficientes que permitan eliminar el hielo una vez se ha formado sobre el ala \cite{politovich}.\\
%
\indent El estudio del engelamiento se realiza en tres escalas:
%
\begin{itemize}
\item La microescala (o microf�sica), que analiza el impacto y crecimiento de las gotas subenfriadas en distintas superficies del avi�n.
\item La mesoescala, que investiga la detecci�n in-situ de las condiciones de engelamiento y la relaci�n.
\item La escala a nivel de nube, que analiza la predicci�n de condiciones de engelamiento, y cuya distinci�n con la mesoescala es difusa.
\end{itemize}
%
\indent Existen distintas formas en las que la formaci�n de hielo puede tener lugar. El fen�meno f�sico en s� consiste b�sicamente en el impacto de gotas de agua en estado subenfriado o subfundido, es decir, en un estado metaestable, al sobrepasar el punto de fusi�n ($T<0�C$) pero manteni�ndose en estado l�quido. Dicho estado se denomina metaestable, dado que un ligero aumento en la presi�n provoca la transformaci�n inmediata en hielo, cosa que sucede al impactar sobre la superficie en cuesti�n. Los dos tipos m�s habituales de engelamiento son el hielo granular (\textit{rime ice}) y el hielo glaseado o v�treo (\textit{glaze ice}), aunque existen otros tipos menos comunes y con otra fenomenolog�a asociada. Contrariamente a lo que podr�a parecer, la adhesi�n de hielo por precipitaci�n no supone ning�n tipo de problema, ni en vuelo (no adhiere) ni en tierra (pues se limpia sencillamente) \cite{avion_y_piloto}.\\
%
\indent Resulta interesante preguntarse cu�nto tiempo permanece un avi�n en condiciones de engelamiento, tambi�n en relaci�n a su tiempo de vida. Pese a que las condiciones de engelamiento se producen con gran asiduidad, es necesario recordar que, a nivel puntual, se encuentran confinadas en una nube, y por grande que sea, un avi�n en condici�n de crucero tarda escasos minutos en atravesarla \cite{politovich}. Ello no quita que la gravedad de sus efectos sea capital, pero conviene tener en cuenta una escala temporal del problema. No obstante, seg�n el fabricante Airbus, una aeronave comercial se encuentra en condiciones de engelamiento hasta en un 15\% de su tiempo de vida .\\
%
\indent El objeto de este cap�tulo no es otro que presentar la casu�stica y la fenomenolog�a del engelamiento, as� como sus efectos, su predicci�n, su prevenci�n y su remedio. Estos aspectos ser�n tratados uno por uno a continuaci�n.
%
\section{El engelamiento atmosf�rico.}
%
\indent Pese a que el asunto de inter�s en este trabajo es la simulaci�n del engelamiento en aeronaves, no se trata del �nico escenario donde este fen�meno aparece. En este apartado, se presentar� una clasificaci�n general del fen�meno del engelamiento, que ser� extendido y desarrollado m�s adelante para los tipos m�s relevantes en la aeronave. De forma general, el engelamiento atmosf�rico puede clasificarse en:
%
\begin{itemize}
\item \textbf{Engelamiento de precipitaci�n} (lluvia engelante o \textit{freezing rain}): Se trata de unas condiciones en las que hay un alto contenido de agua en el aire (el denominado \textit{Liquid Water Content} o \textit{LWC}), oscilando entre 1 y 10 $g/m^3$; acompa�ado de un tama�o de gota grande, alrededor de los $100 \mu m$ de di�metro. Su formaci�n se produce cuando la nieve cae de una nube de precipitaci�n en una capa de aire c�lido, funde y progresa hacia una zona de aire fr�o, subenfri�ndose y alcanzando uno u otro estado, en funci�n de la distancia recorrida en dicho ambiente fr�o. A su vez puede dividirse en otros dos tipos de precipitaci�n.
%
	\begin{itemize}
	\item[\labelitemii] \underline{Hielo glaseado}: Se trata de un escenario muy peligroso para el engelamiento en aeronaves, caracterizado por una predominancia del estado l�quido de la precipitaci�n, tras un corto tiempo en condiciones de subenfriado. Se compone por tanto principalmente de gotas de agua subenfriadas, que en el momento del impacto solidifican parcial o totalmente, pudiendo fluir posteriormente. Se trata de un hielo de tipo muy claro, casi transparente, que se forma a temperaturas altas (entre 0 y -10�C) (ver figura \ref{fig: Glaze_cherry}).
	%
	\item[\labelitemii] \underline{Nieve h�meda}: La nieve h�meda (o aguanieve) es una precipitaci�n en la que el agua ha pasado un largo tiempo en condiciones fr�as, llegando a solidificar parcialmente, y coexistiendo por tanto fase s�lida y l�quida. Se trata de una precipitaci�n cuyos efectos no incluyen a las aeronaves, sino que afectan a estructuras de ingenier�a en tierra. Conviene distinguirla de la nieve seca, de un contenido en agua much�simo menor (ver figura \ref{fig: Aguanieve}).
	\end{itemize}
%

\begin{figure}[ht]
\centering
\medskip
\begin{subfigure}[t]{.49\linewidth}
\centering\includegraphics[width=\linewidth]{Chapter_01/Glaze_ice_cherry}
\captionsetup{justification=centering, margin = 0.5cm}
\caption{Formaci�n de hielo glaseado por precipitaci�n. \cite{cherry}}
\label{fig: Glaze_cherry}
\end{subfigure}
\begin{subfigure}[t]{.49\linewidth}
\centering\includegraphics[width=\linewidth]{Chapter_01/Aguanieve}
\captionsetup{justification=centering, margin = 0.5cm}
\caption{Formaci�n de hielo por nieve h�meda en un cable de alta tensi�n. \cite{aguanieve}}
\label{fig: Aguanieve}
\end{subfigure}
\caption{Engelamiento por precipitaci�n.}
\end{figure}
\item \textbf{Engelamiento en nube}: Se trata del principal escenario estudiado del engelamiento en aeronaves. Se caracteriza por unos contenidos de agua l�quida entre uno y dos �rdenes de magnitud m�s peque�o, rondando entre 0.1 y 1 $g/m^3$. Este tipo de formaci�n de hielo se produce por el contacto directo entre la niebla o nube y la estructura, y en tierra suele tener lugar en zonas de elevada altitud y climas fr�os. A su vez, se pueden distinguir distintos tipos de engelamiento en nube, dependiendo de cuatro par�metros principalmente: la temperatura, la cantidad de agua l�quida en la nube, el tama�o medio de gota (\textit{Median Volume Diameter}, MVD) y la velocidad relativa entre la nube y el cuerpo. Se discutir� con m�s detalle m�s adelante, no obstante se adjunta una escueta clasificaci�n y caracterizaci�n de los tipos:
%
	\begin{itemize}
	\item[\labelitemii] \underline{Hielo glaseado (\textit{glaze})}: Se trata de un proceso y de una formaci�n muy similar a la modalidad de engelamiento por precipitaci�n: Las gotas subenfriadas impactan sobre la superficie, solidificando parcialmente en general, y fluyendo hacia otros puntos de la superficie. A diferencia del engelamiento por precipitaci�n, los tama�os de gota son bajos (un orden de magnitud inferior), pero s� que sucede en el mismo rango aproximado de temperatura (-10-0�C). Es una acreci�n que presenta una gran adhesi�n sobre la superficie, lo cual supone un problema grave en aeronaves.
	%
	\item[\labelitemii] \underline{Hielo escarcha (\textit{rime})}: El hielo escarcha es una formaci�n que tiene lugar a temperaturas m�s bajas (hasta -20�C), en la cual las gotas subenfriadas solidifican de manera total e inmediata, sin fluir (\textit{runback}). Se trata de una formaci�n m�s porosa, debido a esta �ltima caracter�stica, y tiene por tanto una tonalidad blanquecina, en contraposici�n a la transparencia predominante en el hielo glaseado.
	\end{itemize}
\end{itemize}
\section{Efectos del engelamiento.}
% 	
	\subsection{Efectos del engelamiento en aeronaves.}
\indent Aunque el concepto b�sico del engelamiento en vuelo sea sencillo, los procesos que contribuyen a la formaci�n de hielo y sus resultados son muy complejos a la par que fascinantes, y suponen una preocupaci�n importante de meteor�logos, ingenieros y pilotos, dados los severos efectos que se inducen sobre la aeronave. En general, el hielo se adhiere a los elementos expuestos al viento relativo as� como a aquellas partes que sobresalen de la c�lula del avi�n, pudiendo dar lugar a \cite{tema_14}:
%
\begin{itemize}
\item Alteraci�n de la aerodin�mica del ala: Se produce un efecto dram�tico sobre la curva de sustentaci�n y sobre la polar del avi�n, incluso con espesores del orden de mil�metros. 
\item P�rdida de capacidad de mando y control: En el com�n caso de que se forme hielo en superficies estabilizadoras y de control, los efectos son cr�ticos.
\item Reducci�n de visibilidad.
\item Interferencias en las ondas de radio.
\item Errores en los instrumentos.
\item P�rdida de potencia en plantas motoras (engelamiento interno).
\item Vibraciones por la variaci�n en las propiedades inerciales del ala.
\item Incremento de peso: Reviste importancia �nicamente en aeronaves de peque�o tama�o.
\end{itemize}
%

\indent El vuelo de una aeronave en condiciones de engelamiento (\textit{icing conditions}) est� permitido, siempre y cuando est� certificado para ello. De hecho, se trata de una de las condiciones m�s importantes en las que el avi�n ha de ser certificado, estando recogido en las normas de la \textit{Federal Aviation Administration}, las denominadas FAR (parte 25, ap�ndice C) (ver \cite{FAR_icing}). Para alcanzar tal certificaci�n, suele ser com�n que cuenten con alg�n tipo de defensas antihielo, ya sean para prevenirlo o para remediarlo. \\
%
\indent El engelamiento suelen afectar con mayor asiduidad a aviaci�n general, en comparaci�n con aviaci�n comercial. Estos aviones, de menor tama�o, vuelan a altitudes m�s bajas, donde el engelamiento es una situaci�n m�s repetida. Adem�s, suelen carecer de equipos antihielo, as� como de potencia adicional para escapar r�pidamente de estas condiciones perjudiciales. La aviaci�n comercial suele tener como protocolo ascender de manera r�pida, proceso en el cual atraviesan nubes que pueden propiciar el engelamiento. En el vuelo de crucero, se encuentran a altitudes en las que el engelamiento no es posible. Entre ambos tipos de aeronave se encuentran los reactores ejecutivos, tanto por capacidad de actuaci�n ante el engelamiento como por la altitud de vuelo. Son por tanto m�s susceptibles que los aviones comerciales ante estas condiciones.
	%
		\subsubsection{Clasificaci�n de los efectos.}
		%
		\indent Puesto que la formaci�n de hielo puede producirse tanto en superficies exteriores como en elementos internos (motores), los efectos se pueden clasificar de manera an�loga:
		%
		\paragraph{Efectos estructurales. \\}
		%
		\indent El engelamiento estructural se produce por la acumulaci�n de hielo en el exterior del avi�n, y su efecto fundamental es la alteraci�n de las propiedades aerodin�micas de la aeronave:
		%
		\begin{itemize}
		\item \underline{Bordes de ataque de ala y cola}: La acumulaci�n de hielo modifica los perfiles aerodin�micos, degradando el coeficiente de sustentaci�n y el de resistencia. Por tanto, el avi�n tendr� que volar a mayor �ngulo de ataque, acerc�ndose con ello a la entrada en p�rdida. Adem�s, el coeficiente de sustentaci�n m�ximo disminuye, por lo cual aumenta la velocidad de entrada en p�rdida. De todos modos, los efectos dependen de la morfolog�a del hielo formado, pero en todo caso da�a el comportamiento del avi�n. V�anse las figuras \ref{fig: CL_alfa} y \ref{fig: CD_CL}.
		%
		\begin{figure}[ht]
\centering
\medskip
\begin{subfigure}[t]{.45\linewidth}
\centering\includegraphics[width=\linewidth]{Chapter_01/CL_alfa}
\captionsetup{justification=centering, margin = 0.5cm}
\caption{Curva de sustentaci�n en funci�n del �ngulo de ataque.}
\label{fig: CL_alfa}
\end{subfigure}	
\begin{subfigure}[t]{.45\linewidth}
\centering\includegraphics[width=\linewidth]{Chapter_01/CD_CL}
\captionsetup{justification=centering, margin = 0.5cm}	
\caption{Curva polar del perfil.}
\label{fig: CD_CL}
\end{subfigure}
\caption{Caracter�sticas aerodin�micas de un mismo perfil limpio y engelado.}
\end{figure}\\
		%
		\item \underline{H�lices}: Tal y como sucede con los perfiles del ala, se ven deformados, y adem�s se produce un desequilibrio m�sico que genera vibraciones y esfuerzos adicionales.
		%
		\item \underline{Tubos de Pitot y de Venturi}: Los tubos de pitot, que miden la velocidad respecto al aire del avi�n, pueden verse obstruidos por el hielo, proporcionando lecturas err�neas y causando que el piloto act�e como si volara en otras condiciones (subiendo el �ngulo de ataque por ejemplo). No ocurri� a consecuencia del engelamiento, pero el accidente Birgenair 301 (1996) tuvo como motivo potencial la obstrucci�n de un tubo de pitot (por una avispa en este caso).
		%
		\item \underline{Antenas}: La acumulaci�n de hielo en las antenas puede generar vibraciones que dificultan o incluso impiden las comunicaciones.
		%
		\item \underline{Cristales del parabrisas en cabina}: Se produce un descenso en la visibilidad por la acumulaci�n de hielo.
		%
		\item \underline{Tren de aterrizaje y flaps}: Se puede perder eficacia y, en un caso extremo, pueden llegar a bloquearse.
		\end{itemize}
		%
		
		\indent El engelamiento en la cola (\textit{tailplane icing}) constituye un fen�meno que difiere bastante del que tiene lugar en las alas. La configuraci�n aerodin�mica de la cola, as� como el comportamiento con formaci�n de hielo, es notablemente distinto al que sucede en un perfil t�pico de ala. Una cuesti�n relacionada con este problema es la p�rdida del plano de cola, un problema grav�simo de las aeronaves, pues desaparece el momento aerodin�mico que equilibra el cabeceo del avi�n as� como el m�todo b�sico de corregirlo. M�s informaci�n al respecto puede ser encontrada en la referencia \cite{Tailplane}.
		\paragraph{Efectos internos. \\}
		%
		\indent Los efectos internos hacen referencia a la operaci�n de los motores, incluyendo el engelamiento por impacto en las entradas de aire y el engelamiento en el carburador. El efecto principal es la p�rdida de potencia parcial o total. Desde 1988, se han reportado m�s de 150 p�rdidas de potencia sensibles por motivo de la formaci�n de hielo en motores \cite{compressor_icing}.
		%
		\begin{itemize}
		\item \underline{Formaci�n de hielo en las entradas de aire}: El concepto es igual que en el caso de las alas, solo que la superficie de an�lisis es la g�ndola y el difusor del motor (\textit{inlet}). Una reducci�n de su �rea a consecuencia de esto implica una progresiva reducci�n de potencia entregada por el motor. (ver figura \ref{fig: INLET_ICING})
		%
		\item \underline{Formaci�n de hielo en el carburador}: La entrada de aire h�medo en el motor puede provocar la formaci�n de hielo en el carburador (ver figura \ref{fig: CARBURETOR_ICING}), que puede llegar a obstruirse, y generar serios problemas en el funcionamiento del motor.
		%
		\begin{figure}[hb]
			\begin{minipage}{0.55\textwidth}
    			\centering\includegraphics[width = \linewidth]{Chapter_01/INLET_ICING}
    			\captionsetup{justification=centering, margin = 0.5cm}
				\caption{Formaci�n de hielo en la entrada de aire. \cite{inlet}}
    			\label{fig: INLET_ICING}
  			\end{minipage}
  			\hfill
  			\begin{minipage}{0.3\textwidth}
    			\centering\includegraphics[width = \linewidth]{Chapter_01/Carburetor_Icing}
    			\captionsetup{justification=centering, margin = 0.5cm}
				\caption{Formaci�n de hielo en el carburador. \cite{carburetor}}
				\label{fig: CARBURETOR_ICING}
  			\end{minipage}
		\end{figure} 
		\clearpage
		%
		\item \underline{Formaci�n de hielo en el compresor}: Abundantes cristales de hielo y gotas subenfriadas entran en el compresor, impactando y acumul�ndose en el est�tor. Llegado un punto, el hielo se separa del componente al que estuviera adherido, alcanzando fases posteriores del motor y eventualmente generando problemas graves, como entradas en p�rdida (\textit{surge}) con las consecuentes inversiones de flujo. (ver las figuras \ref{fig: Compressor_icing_1}, \ref{fig: Compressor_icing_2} y \ref{fig: Compressor_icing_3})
		%
		\begin{figure}[ht]
		\centering
		\medskip
		\begin{subfigure}[t]{.45\linewidth}
		\centering\includegraphics[width=\linewidth]{Chapter_01/Compressor_icing_1}
		\caption{Primeras etapas.}
		\label{fig: Compressor_icing_1}
		\end{subfigure}
		\vspace{5mm}\\
		\begin{subfigure}[t]{.45\linewidth}
		\centering\includegraphics[width=\linewidth]{Chapter_01/Compressor_icing_2}
		\caption{Acumulaci�n.}
		\label{fig: Compressor_icing_2}
		\end{subfigure}
		\vspace{5mm}\\
		\begin{subfigure}[t]{.45\linewidth}
		\centering\includegraphics[width=\linewidth]{Chapter_01/Compressor_icing_3}
		\caption{Desprendimiento.}
		\label{fig: Compressor_icing_3}
		\end{subfigure}
		\caption{Formaci�n de hielo en el compresor. NASA Glenn Research Center \cite{compressor_icing}.}
		\end{figure}
		\end{itemize}
		%
		\clearpage
		%
\section{Gravedad del engelamiento y factores determinantes.}
%
\indent La severidad del engelamiento se encuentra actualmente clasificada en cuatro categor�as, crecientes en gravedad: perceptible (\textit{trace}), d�bil (\textit{light}), moderada (\textit{moderate}) y grave (\textit{severe}). La severidad se eval�a como una combinaci�n del estado objetivo del entorno meteorol�gico, la respuesta de la aeronave y la opini�n del piloto al respecto de dicha respuesta. La \textit{FAA} \cite{FAR_icing} tiene caracterizaciones para cada uno de los anteriores estados, recogidos en la tabla \ref{table: Severity}.
%
\begin{table}[ht]
\centering
\begin{tabular}{|c|c|}
\hline 
\textbf{Categor�a} & \textbf{Descripci�n} \\ 
\hline 
  & El hielo se vuelve perceptible. La acumulaci�n es ligeramente superior   \\ 
\textit{Trace}		& a la sublimaci�n. No resulta peligroso a�n no contando con equipos  \\
				& antihielo, salvo si se encuentra durante $t>1h$ \\
\hline 
 & El ritmo de acumulaci�n de hielo puede causar problemas si   \\ 
\textit{Light} &              el vuelo se extiende en el entorno ($t>1h$). El uso ocasional \\
 				&  de defensas antihielo es efectivo. \\
\hline 
 & El ritmo de crecimiento de hielo es tal que peque�os per�odos   \\ 
\textit{Moderate} &		 de tiempo en la regi�n resultan potencialmente peligrosos.\\
				&  El uso de defensas antihielo o la evasi�n son necesarias.\\
\hline 
 & El ritmo de acumulaci�n de hielo es de tal magnitud que el  \\ 
\textit{Severe}& equipamiento antihielo es inefectivo. Es necesario una \\
 			& maniobra de evasi�n inmediata.\\
\hline 
\end{tabular} 
\centering\caption{Severidad del engelamiento seg�n la \textit{FAA} \cite{FAR_icing}.}
\label{table: Severity}
\end{table}\\
%
\indent Pese a la descripci�n anterior, es necesario establecer unos par�metros que caractericen las condiciones de engelamiento de una manera cuantitativa. De entre los que afectan al engelamiento, los m�s importantes son \textbf{el contenido de agua l�quida} ($LWC$, \textit{Liquid Water Content}), \textbf{la temperatura exterior} ($T_{\infty}$) y \textbf{el tama�o de las gotas}, caracterizado por el di�metro de las gotas promediado en volumen ($MVD$, \textit{Median Volume Diameter}). La velocidad relativa entre el flujo y el cuerpo (tambi�n denominada velocidad de impacto) es tambi�n un factor relevante, pero en una escala inferior. \\
%
\indent El contenido de agua l�quida favorece la formaci�n de hielo, puesto que pone m�s masa a disposici�n para su congelaci�n. Por tanto, unos valores altos de \textit{LWC} est�n directamente asociados con condiciones m�s severas de engelamiento. La temperatura ambiente $T_{\infty}$, por su parte, controla lo que le sucede a las gotas al impactar: ya sea una congelaci�n inmediata (para temperaturas suficientemente bajas) o un flujo posterior hacia otras posiciones (\textit{runback}), que podr�an no tener protecci�n antihielo. En consecuencia, una temperatura mayor (pero negativa) est� relacionada con una severidad mayor del engelamiento. El tama�o de las gotas controla la denominada eficiencia de colecci�n ($\beta$), es decir, la distribuci�n del agua al impactar sobre la estructura. Globalmente, el efecto de este par�metro no es tan determinante como el de los dos anteriores hasta que se alcanzan tama�os de salpicadura (\textit{drizzle}), rondando los $50 \mu m$, a partir del cual el impacto provoca en casi todos los casos una dispersi�n del agua, con el efecto de alcanzar zonas desprotegidas.\\
%
\indent En consecuencia, las condiciones que agravar�an de forma particular el engelamiento en aeronaves ser�a un alto contenido de agua l�quida, con una temperatura alta y negativa, y un tama�o de gota grande. Se ver� que en funci�n de estos par�metros es posible definir los tipos m�s generales de la formaci�n de hielo: \textit{glaze} y \textit{rime}. A modo de nota, el efecto de la velocidad de impacto es un tanto complejo de analizar. Por un lado, facilita el flujo hacia zonas desprotegidas, por contar con mayor energ�a cin�tica; pero por otra parte, una mayor velocidad implica un flujo de entalp�a positivo en la superficie, disipando m�s energ�a en el impacto y, por tanto, elevando la temperatura de la misma.
%	
	\subsection{Caracter�sticas microf�sicas.}
	%
	\indent Una vez han sido mostradas las tendencias de los par�metros determinantes, es interesante establecer en qu� rangos se mueven en situaciones habituales.
	\paragraph{A. Contenido de agua l�quida. \\}
	%
	\indent El contenido de agua l�quida en condiciones de engelamiento suele ser bastante baja. En nubes de tipo convectivo, alrededor del 90\% de los valores est� por debajo de $0.7 g/m^3$, y en el caso de nubes estratiformes, el valor baja hasta los $0.5 g/m^3$. Los valores m�ximos pueden oscilar en torno a $1.2 g/m^3$, en nubes convectivas profundas. Una caracter�stica a tener en cuenta es la distancia en que un valor de \textit{LWC} se mantiene por encima de un cierto l�mite. Como parece l�gico, al aumentar el l�mite, disminuye la distancia en que aparecen estas condiciones. Por ejemplo, un valor de \textit{LWC} superior a $0.5 g/m^3$ se ha encontrado de manera consistente en distancias inferiores a $13 km$; mientras que si el l�mite se reduce a $0.1 g/m^3$, la distancia crece a $83 km$ de vuelo.
	%
	\paragraph{B. Temperatura ambiente. \\}
	%
	\indent En general, se pueden establecer los l�mites superior e inferior para el engelamiento en aeronaves en $0�C$ y $-25 �C$, con una media en torno a los $-10�C$. A temperaturas inferiores a $-20�C$ no es com�n encontrar incidentes relacionados con el engelamiento; y a temperaturas superiores a $-5�C$, la compresi�n del aire que se produce en el perfil sit�a las temperaturas en los bordes de ataque por encima del punto de congelaci�n, al suponer un incremento de $1-2�C$ en aviones peque�os pero de hasta $6-8�C$ en aviones comerciales.\\
	%
	\paragraph{C. Tama�o de las gotas. \\}
	%
	\indent El tama�o de las gotas es t�picamente peque�o (entre 10 y 20 micras de di�metro). Los m�ximos valores que se encuentran en general alcanzan las 50 micras (a excepci�n de los \textit{SLDs}, ver apartado siguiente). Las tendencias generales muestran que las nubes de tipo cumuliforme contienen gotas de mayor di�metro en comparaci�n con las estratiformes y, adem�s, las nubes mar�timas presentan tambi�n gotas de mayor tama�o (en relaci�n a las continentales). Al igual que sucede con el \textit{LWC}, el tama�o de las gotas crece con la altura en nubes de una sola capa, siendo el comportamiento de nubes multicapa de mayor complejidad.
	%
%
\section{Tipos de engelamiento en aeronaves.}
%
\indent De todos los tipos de engelamiento atmosf�rico enunciados anteriormente, en el escenario particular de las aeronaves solamente se consideran dos tipos de formaciones principales: el hielo glaseado o transparente(\textit{glaze}) y el hielo escarcha o granulado (\textit{rime}). En el caso del primero, se pueden distinguir dos situaciones derivadas: la denominada lluvia engelante (el engelamiento por precipitaci�n de hielo glaseado) y el engelamiento en nube de hielo glaseado. Adicionalmente, existe la posibilidad de que ambas formas (\textit{glaze} y \textit{rime}) se combinen en un modo intermedio entre ambos; y tambi�n es necesario tener en cuenta el fen�meno de impacto de \textit{SLD}. Respectivamente, cada una de las situaciones mencionadas se pueden observar en las figuras \ref{fig: Glenn_glaze}, \ref{fig: Glenn_rime}, \ref{fig: Glenn_mixed} y \ref{fig: Glenn_SLD} \cite{icing_glenn}. \\
%
\begin{figure}[hb]
\centering
\medskip
\begin{subfigure}[t]{.49\linewidth}
\centering\includegraphics[width=\linewidth]{Chapter_01/Glenn_glaze}
\caption{Hielo \textit{glaze} de gravedad severa.}
\label{fig: Glenn_glaze}
\end{subfigure}
\begin{subfigure}[t]{.49\linewidth}
\centering\includegraphics[width=\linewidth]{Chapter_01/Glenn_rime}
\caption{Hielo \textit{rime} de gravedad leve.}
\label{fig: Glenn_rime}
\end{subfigure}
\end{figure}
\begin{figure}\ContinuedFloat
\begin{subfigure}[t]{.49\linewidth}
\centering\includegraphics[width=\linewidth]{Chapter_01/Glenn_mixed}
\caption{Hielo mezcla de gravedad moderada.}
\label{fig: Glenn_mixed}
\end{subfigure}
\begin{subfigure}[t]{.49\linewidth}
\centering\includegraphics[width=\linewidth]{Chapter_01/Glenn_SLD}
\caption{Hielo de \textit{SLDs}.}
\label{fig: Glenn_SLD}
\end{subfigure}
\caption{Fotograf�as tras vuelo de una aeronave. NASA Glenn Research Center \cite{icing_glenn}.}
\end{figure}
%\clearpage

%
	\subsection{Hielo glaseado (\textit{glaze}).}
	%
	\indent El hielo \textit{glaze} o transparente (figura \ref{fig: Glaze_scheme}) se forma a unas temperaturas no muy bajas, puesto que el modo en que se genera es mediante la congelaci�n parcial de las gotas subenfriadas durante el impacto, para extenderse a zonas posteriores y ah� solidificar de manera definitiva. El rango oscila entre los $-2�C$ y $-15�C$, aunque esto depende de la fuente consultada, al no existir un l�mite f�sico objetivo. Habitualmente est� asociado a gotas de tama�o moderado a grande, cuyo volumen permite su extensi�n tras el impacto. El contenido de agua l�quida es tambi�n grande en estos casos. El proceso de congelaci�n del hielo glaseado es lento, y resulta en un hielo con una gran capacidad de adhesi�n a la superficie por adaptarse a su forma (ver figura \ref{fig: Adhesion}). Esto, unido al mayor perjuicio aerodin�mico por su mayor extensi�n, hacen que sea uno de los tipos m�s peligrosos de engelamiento en aeronaves. De todas maneras, una cuantificaci�n de los efectos de este tipo de formaci�n requiere de un an�lisis de la aeronave, a nivel aerodin�mico, de condici�n de vuelo y de equipamiento antihielo.
	%
	\begin{figure}[ht]
	\centering\includegraphics[width = 0.6\linewidth]{Chapter_01/Glaze_scheme}
	\caption{Formaci�n de hielo \textit{glaze}.}
	\label{fig: Glaze_scheme}
	\end{figure}\\
	%
	\indent Un caso particular de la formaci�n de hielo glaseado es la denominada lluvia engelante, explicada esquem�ticamente en la figura \ref{fig: Freezing_rain}. Resulta especialmente peligrosa ya que, a la condici�n de hielo transparente, se une la uniformidad con que afecta a toda la aeronave.
	%
	\begin{figure}
	\centering\includegraphics[width = 0.6\linewidth]{Chapter_01/Freezing_rain}
	\caption{Fen�meno de la lluvia engelante \cite{Freezing_rain}.}
	\label{fig: Freezing_rain}
	\end{figure}
	\subsection{Hielo escarcha (\textit{rime})}
	%
	\indent El hielo granular o \textit{rime} (ver figura \ref{fig: Rime_scheme}) se forma por el congelamiento directo e instant�neo de las gotas que impactan sobre el perfil. Como parece normal, est� asociado a temperaturas m�s bajas, generalmente inferiores a $-10�C$, aunque de nuevo existen distintos criterios en funci�n de las referencias consultadas. Est� relacionado tambi�n con un tama�o de gota menor, capaz de congelarse de manera inmediata tras el impacto. Dado que no tiene tiempo de cambiar su forma, el hielo toma un aspecto m�s rugoso y blanquecino, recordando a la nieve. La adherencia es inferior a la del hielo glaseado, y el efecto aerodin�mico es comparativamente inferior. Dada su tonalidad, es m�s f�cil de detectar desde cabina, lo que supone un aspecto a su favor.
	%
	\begin{figure}[ht]
	\centering\includegraphics[width = 0.6\linewidth]{Chapter_01/Rime_scheme}
	\caption{Formaci�n de hielo \textit{rime}.}
	\label{fig: Rime_scheme}
	\end{figure}
	%
	%
	\begin{figure}[ht]
	\centering\includegraphics[width = 0.3\linewidth]{Chapter_01/Adhesion}
	\caption{Formaci�n por capas de hielo \textit{glaze} (arriba) y \textit{rime} (abajo).}
	\label{fig: Adhesion}
	\end{figure}\\
	%
	\subsection{Hielo mezcla.}
	%
	\indent El hielo mezcla aparece generalmente por la intercalaci�n de capas de hielo transparente y granular. Sus caracter�sticas suelen estar entre un tipo y otro, presentando una notable adhesividad pero con una cierta fragilidad.
	%
	\subsection{\textit{Supercooled large droplet icing} (SLD).}
	%
	\indent El fen�meno de impacto de gotas grandes subenfriadas (\textit{supercooled large droplet icing}) se analiza pr�cticamente como un caso aparte, dada la severidad que conlleva as� como la forma en que se generan. No obstante guarda una cierta similaridad con el hielo \textit{glaze}, pero con numerosos matices. Sus efectos son muy perjudiciales, ya que dado su gran tama�o (di�metros que superan los $50 \mu m$), son capaces de alcanzar zonas del avi�n que de otra manera permanecer�an impasibles. Adem�s, la formaci�n de hielo en zonas avanzadas del perfil genera una rugosidad con un influjo mucho m�s grave sobre la aerodin�mica del perfil que la formada en el borde de ataque.\\
	%
	\indent Existen dos situaciones generales que inducen la formaci�n de \textit{SLDs}. La primera de ellas es el fen�meno de la lluvia engelante (ya descrita anteriormente), que se puede predecir de una forma m�s o menos sencilla, dado lo espec�fico del perfil termodin�mico que tiene que tener lugar. La segunda situaci�n es la coalescencia de gotas m�s peque�as, mucho m�s dif�cil de detectar. Est� causada por esfuerzos cortantes del viento (\textit{wind shear}), es decir, diferencias en velocidad del viento en las partes superiores de las nubes, pero conviene aclarar que no se han detectado los mecanismos exactos mediante los cuales sucede este fen�meno. S� se ha estimado que el contenido en agua l�quida debe oscilar entre $0.2-0.25 g/m^3$ en nubes continentales y $0.1 g/m^3$ en nubes oce�nicas.
%
\section{El entorno engelante.}
%
\indent Aunque ya se han sentado las bases sobre los par�metros que afectan al engelamiento, es necesario establecer una relaci�n con el conjunto de condiciones meteorol�gicas que conforman el entorno engelante. En primer lugar, se introducir�n las condiciones meteorol�gicas que m�s com�nmente vienen relacionadas con el engelamiento, para posteriormente realizar un an�lisis geogr�fico de los reportes de sucesos de formaci�n de hielo. Finalmente, se describir� ligeramente la relaci�n entre el engelamiento y los tipos de nubes que lo producen.\\
%
	\subsection{Condiciones meteorol�gicas y engelamiento.}
	%
	\indent Dado que est�n ligados a nubes h�medas o de precipitaci�n, los sucesos de engelamiento est�n asociados a meteorolog�as que produzcan esas condiciones. La mayor parte de los reportes de pilotos (\textit{PIREPs}, \textit{pilot reports}) de condiciones de engelamiento tienen lugar en zonas cercanas a frentes c�lidos cercanos a la superficie, al sustentar aire h�medo desde capas m�s bajas. No obstante, aunque menos habitual, los frentes fr�os tambi�n generan estas condiciones.\\
	%
	\indent Las masas de aire mar�timo contienen una mayor cantidad de humedad, y est�n m�s frecuentemente asociadas a \textit{PIREPs} que las masas de aire continental, especialmente aquellas m�s alejadas de las zonas polares. La topograf�a tambi�n ejerce una influencia sobre las condiciones de engelamiento, al producir fuentes locales de elevaci�n de masas de aire. Tomando el ejemplo de Estados Unidos, las Monta�as Rocosas inducen la elevaci�n de masas de aire h�medo en el centro de la naci�n, induciendo condiciones favorables para sucesos de engelamiento lejos de la costa.\\
	%
	\indent Pese a lo que pueda parecer, las condiciones meteorol�gicas en tierra no son de mucha utilidad para estimar la posibilidad de sucesos engelantes: los \textit{PIREPs} se reparten pr�cticamente a partes iguales entre precipitaci�n de agua o nieve en la superficie y condiciones secas. Sin embargo, la siniestralidad s� se concentra en zonas fr�as y nubosas. En la figura \ref{fig: Incidencia_USA} se observa una representaci�n de la frecuencia de los sucesos de engelamiento en los Estados Unidos. La tendencia es bastante clara: Los incidentes se concentran en las latitudes m�s altas, multiplic�ndose casi por diez entre las zonas m�s septentrionales respecto a las meridionales. La siniestralidad media al a�o oscila los treinta incidentes s�lo en este pa�s, concentr�ndose en los meses del invierno.
	%
	\begin{figure}[ht]
	\centering\includegraphics[width = 0.7\linewidth]{Chapter_01/Incidencia_USA}
	\captionsetup{justification=centering, margin = 0.5cm}
	\caption{Frecuencia de incidentes de engelamiento en Estados Unidos, expresada en porcentaje de tiempo de vuelo. \cite{politovich}.}
	\label{fig: Incidencia_USA}
	\end{figure}\\
	%
	\indent En lo que respecta a la altitud en que suceden estos reportes, la media de estos se encuentra alrededor de los $3000\,m$, existiendo muy escasos sucesos en altitudes superiores a $6000 \, m$. Ello resulta tranquilizador en cierto modo, puesto que la altitud de crucero de un avi�n comercial se sit�a bastante por encima de dicho nivel. En cuanto al reparto de \textit{PIREPs} a lo largo de un d�a, �ste obedece b�sicamente a la frecuencia de vuelos, cayendo durante las horas nocturnas. Una prueba que reafirma este comportamiento es que a lo largo de una semana, de nuevo obedecen a los d�as de mayor tr�fico (de Martes a Jueves).
	%
	\subsection{Tipolog�a de nubes y relaci�n con el engelamiento.}
	%
	\indent Existen dos tipos principales de nubosidad que albergan condiciones engelantes \cite{meteorology}:
	%
	\paragraph{A. Nubes cumuliformes. \\}
	%
	\indent Las nubes de tipo cumuliforme se forman por fuertes corrientes de aire ascendente, lo cual les dota de la forma alargada verticalmente. Este tipo de nube incluye muchos otros tipos, pero las que provocan engelamiento generalmente son los cumulonimbos (Cb), denominadas coloquialmente nubes de tormenta, y los cumulus congestus, especialmente alargadas verticalmente. El contenido en agua l�quida de estas nubes es notablemente alto (alrededor de $3\,g/m^3$), lo cual las convierte en un tipo de nube peligrosa en estas condiciones. La ventaja que juega a su favor es su corta extensi�n, sin sobrepasar los $10km$, toda vez que puede ser una distancia m�s que suficiente para que la formaci�n de hielo sea severa. A continuaci�n se adjuntan dos gr�ficas, extra�das de las normas \textit{FAR} \cite{FAR_icing}, que muestran respectivamente \ref{fig: LWC_T_cumu} la envolvente de las condiciones engelantes para las nubes cumuliformes, en las cuales el avi�n ha de ser capaz de volar con seguridad para ser certificados y \ref{fig: LWC_horizontal_extent_cumu} la extensi�n de las nubes para un contenido de agua l�quida dado. Se observa que la certificaci�n abarca desde los $0�C$ hasta los $-40�C$, con unos valores de \textit{LWC} notablemente altos. Asimismo, el \textit{MVD} alcanza las 50 micras, valor al que las nubes estratiformes no llegan. Por tanto, la formaci�n de hielo en estas nubes es habitualmente hielo transparente (el m�s peligroso).
	%
	\begin{figure}[ht]
	\centering\includegraphics[width = 0.7\linewidth]{Chapter_01/LWC_T_cumu.png}
	\captionsetup{justification=centering, margin = 0.5cm}
	\caption{Envolvente de certificaci�n en condiciones de engelamiento continuo (nubes cumuliformes) \cite{FAR_icing}.}
	\label{fig: LWC_T_cumu}
	\end{figure}
	\begin{figure}[ht]
	\centering\includegraphics[width = 0.66\linewidth]{Chapter_01/LWC_horizontal_extent_cumu.png}
	\captionsetup{justification=centering, margin = 0.5cm}
	\caption{Extensi�n de nubes cumuliformes para un \textit{LWC} dado \cite{FAR_icing}.}
	\label{fig: LWC_horizontal_extent_cumu}
	\end{figure}
	%
	\FloatBarrier
	\paragraph{B. Nubes estratiformes. \\}
	%
	\indent Las nubes estratiformes se generan a partir de corrientes de aire que ascienden de una forma m�s d�bil, lo que provoca una forma m�s aplanada y alargada en sentido horizontal. Se trata de nubes de una gran extensi�n, que pueden superar los 300 kil�metros (aunque no sea lo com�n); mientras que su contenido en agua l�quida en la envolvente de certificaci�n es sensiblemente inferior al de las nubes cumuliformes ($<1\,g/m^3$). El tama�o de gota es tambi�n inferior, por debajo de las 40 micras. En las figuras \ref{fig: LWC_T_strat} y \ref{fig: LWC_horizontal_extent_strat} se pueden observar las mismas envolventes y curvas que se han mostrado para las nubes cumuliformes. Merece menci�n el t�tulo que reciben para la \textit{FAA}, denotando como \textit{continuous maximum} a las estratiformes (por la necesidad de exposici�n prolongada) y \textit{intermitent maximum} a las cumuliformes. En dichas envolventes adem�s se indica la exposici�n a la que debe someterse el avi�n para ser certificado, siendo 2.6 millas n�uticas para las cumuliformes y 17.4 para las estratiformes.
	%
	\begin{figure}[ht]
	\centering\includegraphics[width = 0.66\linewidth]{Chapter_01/LWC_T_strat.png}
	\captionsetup{justification=centering, margin = 0.5cm}
	\caption{Envolvente de certificaci�n en condiciones de engelamiento continuo (nubes estratiformes) \cite{FAR_icing}.}
	\label{fig: LWC_T_strat}
	\end{figure}
	\begin{figure}[ht]
	\centering\includegraphics[width = 0.7\linewidth]{Chapter_01/LWC_horizontal_extent_strat.png}
	\captionsetup{justification=centering, margin = 0.5cm}
	\caption{Extensi�n de nubes estratiformes para un \textit{LWC} dado \cite{FAR_icing}.}
	\label{fig: LWC_horizontal_extent_strat}
	\end{figure}
	\clearpage
	%
\section{Detecci�n de condiciones de engelamiento.}
%
	\subsection{Sensores \textit{in situ}.}
\indent La detecci�n \textit{in situ} del engelamiento se puede realizar o bien visualmente o bien a trav�s de alg�n tipo de instrumentos. En general, los pilotos cuentan con una vista bastante impedida de las alas del avi�n, de modo que se suele evaluar en vista de la formaci�n de hielo en el parabrisas o en tubos de pitot en el morro del avi�n. Adem�s, y de acuerdo a la tabla \ref{table: Severity}, el piloto puede detectar variaciones en el comportamiento del avi�n que le sirvan para evaluar la condici�n del mismo. \\
%
\indent Sin embargo, los m�todos m�s eficaces para detectar el engelamiento en ciertas partes del avi�n son los instrumentales. Los detectores de hielo avisan al piloto de la formaci�n de hielo en ciertas partes del avi�n. Habitualmente, estos instrumentos proporcionan un aviso temprano mucho antes de que el piloto pueda observar visualmente el hielo. Existen numerosos tipos de detectores, destacando los siguientes:
%
\begin{itemize}
\item \underline{Sensores de capacitancia}: Montados en la superficie del ala, detectan diferencias en la capacitancia de la superficie, la cual var�a seg�n avanza el engelamiento.
%
\item \underline{Varillas de vibraci�n} (\textit{vibrating rod}): Se trata de una varilla expuesta al flujo que detecta la variaci�n de las frecuencias naturales de la misma al formarse hielo. (figura \ref{fig: Vibrating_rod}, \cite{Vibrating_rod})
%
\item \underline{C�maras infrarrojas}: Actualmente en desarrollo, analizan radiaci�n infrarroja reflejada por el ala, y detectan con ello variaciones en su superficie.(figura \ref{fig: Capacitance}, \cite{Capacitance})
\end{itemize}

\begin{figure}[ht]
\centering
\medskip
\begin{subfigure}[ht]{.4\linewidth}
\centering\includegraphics[width=\linewidth]{Chapter_01/Vibrating_rod}
\caption{Varilla vibratoria \cite{Vibrating_rod}.}
\label{fig: Vibrating_rod}
\end{subfigure}
\begin{subfigure}[ht]{.57\linewidth}
\centering\includegraphics[width=\linewidth]{Chapter_01/Capacitance}
\caption{Sensor de capacitancia \cite{Capacitance}.}
\label{fig: Capacitance}
\end{subfigure}
\caption{Sensores \textit{in situ} de detecci�n de hielo.}
\end{figure}
%
\indent La ventaja de los sistemas de detecci�n \textit{in situ} es que proporcionan una descripci�n definida de las condiciones de engelamiento. Sin embargo, presentan una gran desventaja, y es que el avi�n debe encontrarse en tales condiciones, y ello no es algo para nada deseable.
%
	\subsection{Sensores remotos.}
	%
	\indent Los sensores remotos de condiciones de engelamiento se encuentran en fases muy tempranas de su desarrollo. De hecho, en la actualidad, no existe ning�n instrumento que permita la detecci�n precisa y suficientemente remota de agua subenfriada en la atm�sfera. La metodolog�a usada combina informaci�n de diversas fuentes, como lo son radares aeroportuarios (como el \textit{TDWR} o el \textit{NEXRAD}) que si bien no fueron dise�ados para ello, pueden dar alg�n tipo de informaci�n relevante acerca de las condiciones en un entorno razonable del avi�n. Esto, con ayuda de modelos num�ricos de predicci�n meteorol�gica, im�genes sat�lite o condiciones en tierra, puede dar pistas sobre la existencia de condiciones de engelamiento. Otra estrategia que se considera es la actualizaci�n de radares, que mediante peque�as modificaciones, pueden resultar de gran utilidad.\\
	%
	\indent Un problema de la detecci�n con sensores remotos es la imposibilidad de distinguir entre agua l�quida y hielo con el uso de la reflectividad. Una posible estrategia en desarrollo es la implementaci�n de sensores de polarizaci�n, que ayuden a determinar la forma del meteoro. En otro respecto, los sat�lites geoestacionarios opreacionales ambientales (\textit{GOES}) muestran resultados prometedores como detectores de condiciones engelantes. Su funcionamiento se basa en la detecci�n de radiaci�n en el espectro visible combinada con el infrarrojo, que permite la localizaci�n de nubes con agua subenfriada en su parte superior. Aun as�, por s� solos no resultan suficientes, necesitando una vez m�s de otros datos y sensores para poder dar una detecci�n lo suficientemente precisa.
	%
\section{Predicci�n de condiciones de engelamiento.}
%
\indent La predicci�n de condiciones de engelamiento es an�loga la de la existencia de agua subenfriada en las nubes, lo que supone una tarea muy complicada. En general, se suele realizar un proceso de predicci�n progresivamente restrictivo, contando con los pasos siguientes:
%
\[
\begin{array}{ccccccc}
\text{Nubes/precipitaci�n} & \rightarrow & \text{Temperaturas favorables} \: [-20, 0]�C & \rightarrow \\
 \rightarrow  \text{ Aire ascendente que produzca l�quido} & \rightarrow & \text{ Ausencia de hielo que pueda inducir solidificaci�n}

\end{array}
\]
%
	\subsection{M�todos actuales.}
	% 
	\indent Pese a la gran variedad de m�todos de predicci�n que existen en la actualidad, todos ellos tienen una serie de aspectos en com�n: requieren mucho personal y tiempo, tienen un cierto grado de subjetividad y los resultados finales suelen ser dif�ciles de interpretar. El mayor inconveniente existente en la actualidad es que se trabaja con unos datos recogidos para otros fines, es decir, hay que extrapolar en cierto modo datos impl�citos. Una metodolog�a habitual es la interpretaci�n de datos de temperatura y humedad relativa de modelos meteorol�gicos num�ricos, que proveen de una precisi�n del 75\% (entendiendo como precisi�n el n�mero de reportes de engelamiento en zonas donde estaba prevista su existencia).
	%
	\subsection{M�todos en desarrollo.}
	\indent Con la llegada de modelos num�ricos avanzados de predicci�n meteorol�gica, de la mano de la detecci�n del contenido de agua l�quida en nubes, se abre un nuevo potencial de detecci�n \textit{real} del engelamiento, en lugar de la inferencia llevada a cabo hasta ahora. Ejemplos de los mencionados modelos son el \textit{MM5} o el \textit{RAMS}, ambos desarrollados en los Estados Unidos. Un impedimento de su implementaci�n es la demanda de tiempo, potencia y almacenamiento computacional. De todos modos, el rastreo del contenido de agua l�quida es la �nica v�a hacia la detecci�n certera de condiciones de engelamiento.\\
	%
	\indent Ninguna fuente de informaci�n es suficiente por s� sola para predecir d�nde se encuentran condiciones propicias para el engelamiento, o aspectos como su severidad o su tipo. Los meteor�logos combinan la informaci�n de varias fuentes para construir la historia completa, y poder determinar con seguridad todos estos aspectos. Su gran valor y su enormes requerimientos de tiempo sugieren y justifican el hecho de desarrollar algoritmos que autom�ticamente alcancen esos resultados. La clave para una predicci�n efectiva de condiciones engelantes reside en un entendimiento profundo de (A) los procesos que resultan en la formaci�n de agua subenfriada, (B) el modo en que estos procesos se relacionan con fen�menos observables y (C) c�mo combinar la informaci�n de todas las fuentes posibles para obtener la imagen m�s precisa de la situaci�n.
	%
\section{Defensas antihielo.}
%
\indent A fin de protegerse ante los efectos que el engelamiento provoca sobre la aeronave, aquellas cuya vida incluya el vuelo en estas condiciones han de estar equipadas con alg�n tipo de defensa antihielo. En una aeronave, en general, hay tres tipos de zonas: las zonas provistas de sistemas antihielo (\textit{anti-ice}), las que han sido desheladas (\textit{de-ice}) y las que est�n desprotegidas. Para evitar la formaci�n de hielo y sus muy peligrosos efectos, hay dos l�neas de actuaci�n. O bien se previene la formaci�n de hielo, o bien se elimina una vez ha sido depositado. 
%
	\subsection{Medidas meteorol�gicas.}
	%
	\indent Las medidas meteorol�gicas simplemente marcan actitudes a la hora de volar en ciertas regiones, es decir, c�mo evitar las condiciones de engelamiento (al menos en su variante severa). Para ello, se debe evitar volar en las siguientes condiciones \cite{enemigo_oculto}:
	%
	\begin{itemize}
	\item Zonas con frentes temporales fr�os particularmente severos.
	\item Zonas con nubosidad cumuliforme de gran desarrollo vertical o simplemente densa.
	\item Zonas de precipitaci�n de un frente temporal dado.
	\item Nieblas densas.
	\item Nubes estratiformes densas con gran contenido de agua.
	\item En zonas monta�osas, evitar nubosidad de tipo convectivo.
	\end{itemize}
	%
	\indent En caso de encontrarse con alguna situaci�n de las anteriores, lo correcto es modificar la ruta como sea necesario para minimizar o anular la distancia recorrida en tales condiciones. Este comportamiento se encuentra ilustrado en la figura \ref{fig: Flight_path}.
	%
	\begin{figure}[ht]
	\centering\includegraphics[width = 0.5\linewidth]{Chapter_01/Flight_path}
	\caption{Traves�a correcta para atravesar un frente fr�o evitando condiciones de engelamiento \cite{avion_y_piloto}.}
	\label{fig: Flight_path}
	\end{figure}
	%
	\subsection{Medidas t�cnicas.}
	%
	\indent Pese a evitar en la medida de lo posible las condiciones que propicien el engelamiento, existe la posibilidad que de todos modos acabe form�ndose hielo en la aeronave; o tambi�n es posible necesitar de una protecci�n por pura precauci�n. Las medidas que est�n destinadas a eliminar (deshielo) o prevenir (antihielo) el hielo formado en la aeronave se denominan medidas t�cnicas, que generalmente est�n dise�adas en varias l�neas:
	%
	\begin{itemize}
	\item \textbf{Medidas qu�micas}: Se basan en rociar ciertas superficies de la aeronave antes del despegue con un l�quido anticongelante, cuyo principio f�sico es disminuir el punto de congelaci�n del agua. Se trata de un tratamiento que generalmente se aplica en zonas geogr�ficas fr�as y con precipitaci�n abundante y frecuente, buscando adem�s eliminar posibles acumulaciones de hielo preexistentes.
	%
	\begin{figure}
	\centering\includegraphics[width = 0.5\linewidth]{Chapter_01/De_icing_airport}
	\caption{Tratamiento qu�mico antihielo en un Airbus A330 \cite{De_icing_airport}. }
	\end{figure}\\
	%
	\item \textbf{Medidas mec�nicas}: Se basan en la modificaci�n mec�nica de la superficie sobre la cual hay hielo, a fin de desprenderlo por fragilidad. Existen distintas medidas dentro de este conjunto:
	%
		\begin{itemize}
		\item[\labelitemii] \underline{Botas neum�ticas}: El hielo formado es eliminado mediante el inflado c�clico de \textit{botas} (c�maras de aire) internas al borde de ataque. provocando la deformaci�n del borde de ataque, y la rotura del hielo por su inherente fragilidad, desprendi�ndose de manera bastante sencilla. Cuentan con el inconveniente de ser incapaces de eliminar capas peque�as de hielo.
		\item[\labelitemii] \underline{Sistemas de impulso neum�tico}: Est�n basados en el mismo principio que el sistema anterior, pero tienen la capacidad de eliminar capas peque�as de hielo al contar con varios tubos inflables en lugar de uno solo m�s grande (figura \ref{fig: Pneumatic_boot}).
		\item[\labelitemii] \underline{Electroimpulsos}: Una serie de electroimanes son excitados de manera c�clica, deformando la superficie met�lica. Su mayor inconveniente es la fatiga que generan sobre el metal.
		\item[\labelitemii] \underline{Electro-expulsi�n}: Se hace correr una corriente el�ctrica sobre unos alambres de cobre en la superficie, generando un campo magn�tico que permite romper el hielo en peque�os fragmentos que f�cilmente caen del avi�n.
		\end{itemize}
	%
	\item \textbf{Medidas t�rmicas:} El hielo se funde por calentamiento de la superficie. En el caso de que el calor sea suficiente para evitar la congelaci�n de las gotas subenfriadas, se denomina antihielo, mientras que si el hielo se elimina una vez formado, se denomina deshielo. Existe una serie de tecnolog�as actuales dentro de este grupo:
	%
		\begin{itemize}
		\item[\labelitemii] \underline{Sangrado de aire del motor}: Se extrae aire del motor (de la primera etapa de compresor en particular), y se redirige hacia el ala, donde calienta el borde de ataque mediante los denominados tubos de \textit{piccolo} (ver figura \ref{fig: Piccolo}). Otra variante es el uso de conductos S o D, que permiten el calentamiento de los bordes de ataque de las g�ndolas, evitando el engelamiento en el difusor. Tambi�n es com�n su utilizaci�n para evitar el congelamiento de los carburadores.
		\item[\labelitemii] \underline{Panel electrot�rmico}: Se trata de calentadores el�ctricos embebidos en la superficie a deshelar, basados en el efecto Joule.
		\end{itemize}
	\end{itemize}
	%
	\begin{figure}[ht]
\centering
\medskip
\begin{subfigure}[t]{.48\linewidth}
\centering\includegraphics[width=\linewidth]{Chapter_01/Pneumatic_boot}
\caption{Sistema de deshielo neum�tico \cite{Pneumatic_boot}.}
\label{fig: Pneumatic_boot}
\end{subfigure}
\begin{subfigure}[t]{.48\linewidth}
\centering\includegraphics[width=\linewidth]{Chapter_01/Piccolo}
\caption{Sistema de deshielo t�rmico (tubo de piccolo) \cite{Piccolo}.}
\label{fig: Piccolo}
\end{subfigure}
\caption{Sistemas de deshielo actuales.}
\end{figure}

%
	\indent Es necesario destacar que el deshielo mediante medidas t�rmicas provoca que el agua fluya hacia zonas posteriores, pudiendo volver a congelar y haci�ndolo adem�s en una zona desprotegida a priori. Por tanto, hay que ser cuidadosos con la aplicaci�n de estos sistemas. Tambi�n se debe remarcar que estos m�todos suponen una p�rdida de potencia notoria, que en muchas situaciones puede no ser importante, pero en aviones peque�os, que adicionalmente, se encuentren ascendiendo, puede ser algo a tener muy en cuenta. 
	%
		\subsubsection{Medidas t�cnicas en desarrollo.}
		%
		\indent Actualmente existen numerosas investigaciones sobre m�todos m�s efectivos y eficientes de deshielo y antihielo. A continuaci�n se enumeran algunas de las m�s importantes \cite{nanotech}:
		%
		\paragraph{I. Capa de nanotubos de carbono. \\}
		%
		\indent Desarrollado por el \textit{Battelle Memorial Institute} en Ohio, EE.UU., consiste en aplicar una capa de nanotubos de carbono en las superficies a proteger de la aeronave, y utilizarlos como un panel electrot�rmico, es decir, se calienta por el flujo de corriente el�ctrica. Se trata de un sistema realmente eficiente, pero sobre todo, muy ligero, rondando el 1\% de cualquier equipo antihielo actual \cite{nanotubos}.
		%
		\paragraph{II. Recubrimiento de part�culas hidr�fugas. \\}
		%
		\indent Obra de cient�ficos japoneses, se trata de un recubrimiento superhidr�fugo que impide que el agua se adhiera a la superficie del avi�n, compuesto por politetrafluoroetileno (PTFE), que reduce la energ�a necesaria para separar una gota de una superficie dada. Se est� estudiando su implementaci�n en todo el avi�n a modo de una capa de pintura.
		%
		\paragraph{III. Recubrimiento con plasma. \\}
		%
		\indent Llevada a cabo en el centro CICATA de Quer�taro, M�xico, se trata de un m�todo que planea cubrir con una pel�cula de plasma las superficies expuestas de la aeronave. Los pol�meros que componen dicho plasma inducen un aumento de la tensi�n superficial del agua, incit�ndola a formar gotas de mayor tama�o que acaban por escurrirse.


%\chapter{The industrial problem}\label{chap:2}
The whole validation pipeline proposed in the next section is illustrated with realistic data from a real industrial problem of particular interest to the aerospace sector: stress analysis and Reserve Factor prediction in fuselage structures. Results from statistical tests, graphs and tables are provided alongside the theoretical background at each step. This chapter is thus dedicated to introduce the industrial problem used for illustration of results.\\
%
\indent The regression problem at hand is an application of neural networks (NN) in aircraft stress engineering. For typical aircraft fuselage panel design, the dominant form of stiffened post buckling failure under shear loading is forced crippling\cite{bijlaard1955buckling}. This occurs when the shear buckles in the panel skin force the attached stiffener flanges to deform out-of-plane. There are other failure modes related to buckling, tension, and compression.\\
%
\indent The surrogate model presented here is the ''MS-S18'' model, whose aim is to predict Reserve Factors ($RF$) that quantify failure likelihood for regions of the aircraft subject to different loads happening in flight maneuvers (vid. \autoref{fig:diagrama_cajas}). There are six possible failure modes, whose names are ''Forced Crippling'', ''Column Buckling'', ''In Plane'', ''Net Tension'', ''Pure Compression'', and ''Shear Panel Failure''.\\
%
\indent Input data consists of 26 features (loads applied to a specific region of the aircraft and different maneuver specs). This variables are either numerical continuous variables (the magnitudes of the applied loads, for instnace) or categorical variables. This are variables which can only take a restricted set of values (\eg boolean variables, integer variables, etc.) There are three categorical variables alongside 23 numerical ones in MS-S18's dataset: ''dp'', ''Frame'', and ''Stringer''. They contain geometrical information about the region where the loads described by the numerical variables are being applied. For instance, ''Frame'' and ''Stringer'' identify structural elements of the fuselage (their values are string variables like ''Fr)\\
%
\indent Output data consists of 6 reserve factors that quantify the stress failure likelihood: $\mathbf{y}=(RF_1,RF_2,RF_3,RF_4,RF_5,RF_6)$, where $RF_i\in [0,5]$. In this (logarithmic) scale, $0$ means extreme risk and $5$ means risk extremely low.\\
%
\indent This study prioritizes assessing the efficacy of the developed validation tool, not optimizing the neural network model itself. Therefore, the specific architecture and additional features of the NN are intentionally treated as a black box. Our focus remains solely on its inputs --the 26 aforementioned features-- and its outputs --the 6 reserve factors representing 6 distinct failure modes--. This simplification allows us to isolate and evaluate the performance of the validation tool without introducing confounding variables related to the specific neural network design (\ie, without specifying the network's architecture and its parameters, $W$).\\
%
\begin{figure}[!htb]
	\centering
	\begin{tikzpicture}
		% Caja 1
		\node[draw, minimum width=2cm, minimum height=3cm, font=\scriptsize] (box1) at (0,0) {
			\begin{tabular}{c}
				\textbf{INPUT $X$}\\
				Loads + Geometry +\\
			\end{tabular}
		};
		\node[below=0.1cm of box1] {$\mathbf{X}=(x_1,x_2,...,x_n)$};
		
		% Caja 2
		\node[draw, minimum width=2cm, minimum height=3cm, right=0.5cm of box1, font=\scriptsize] (box2) {
			\begin{tabular}{c}
				\textbf{SURROGATE MODEL}\\
				Stress learning model\\
				trained by splitting cases into\\
				train/test set and minimize loss\\
				function (Deep Neural Network)
			\end{tabular}
		};
		\node[below=0.1cm of box2] {$\cal{F}:\mathbf{X} \rightarrow \mathbf{Y}$};
		
		% Caja 3
		\node[draw, minimum width=2cm, minimum height=3cm, right=0.5cm of box2, font=\scriptsize] (box3) {
			\begin{tabular}{c}
				\textbf{PREDICTION (OUTPUT) Y}\\
				A set of Reserve Factors ($RF$)\\
				which characterize the likelihood\\
				of different failure modes.
			\end{tabular}
		};
		\node[below=0.1cm of box3] {$\mathbf{Y}=(y_1,y_2,...,y_m)$};
		
		% Flechas
		\draw[->, >=Stealth] (box1.east) -- (box2.west);
		\draw[->, >=Stealth] (box2.east) -- (box3.west);
	\end{tikzpicture}
	\caption{Surrogate model pipeline}
	\label{fig:diagrama_cajas}
\end{figure}

%\chapter{Conclusiones y futuros estudios.}
%
\indent Una vez desarrollado por completo el proyecto, es de obligado cumplimiento realizar un an�lisis de las conclusiones que se puedan extraer al respecto. Resulta de especial valor hacerlo en referencia a los objetivos planteados al principio del documento, y evaluando en qu� grado han sido satisfechos cada uno de ellos:\\
\begin{itemize}
\item En primer lugar, el fen�meno del engelamiento ha sido estudiado e investigado de manera apropiada, a un nivel m�s que adecuado para los efectos del trabajo en su conjunto. Se han cubierto aspectos relativos a la fenomenolog�a meteorol�gica, sus efectos y la prevenci�n de los mismos, desde un enfoque predictivo, de detecci�n y de actuaci�n al respecto. 
%
\item Por otra parte, se han introducido con un cierto rigor los conceptos b�sicos de la simulaci�n del crecimiento de hielo en su conjunto, con una descripci�n preliminar de cada m�dulo que toma cartas en el proceso global y el modo en que interact�an. Asimismo, se han mostrado las caracter�sticas b�sicas de cada uno de los m�dulos auxiliares (no propiamente desarrollados) del problema: el solver TAU-Code y el m�dulo GOTA. Se han presentado tambi�n los distintos enfoques posibles en el modelo f�sico de cada uno de los bloques.
%
\item Con respecto al establecimiento de las bases del m�todo de los vol�menes finitos, se puede afirmar que ha sido un �xito, implementando y validando m�todos para problemas cada vez m�s complejos, pasando de la sencilla ecuaci�n de la advecci�n a las ecuaciones de Euler, con una f�sica y una discretizaci�n m�s compleja.
%
\item El desarrollo de un m�dulo termodin�mico para la formaci�n de hielo basado en ecuaciones diferenciales ha resultado realmente complicado, como es normal con un problema del que se desconoce la soluci�n y para el que no hay unas ecuaciones generales completamente establecidas, conocidas y asumidas por la comunidad cient�fica. Ello ha dado lugar a m�ltiples enfoques, consideraciones e hip�tesis, que han acabado en el desarrollo de un c�digo que vierte unos resultados razonablemente coherentes. Bien es cierto que carece de una cierta generalidad o de un modo automatizado de compenetrarse con los otros m�dulos, pero establece la base de un modelo s�lido que desarrollar m�s adelante.
%
\item Un aspecto negativo acerca de los resultados obtenidos es la ausencia de aplicaciones pr�cticas, en tanto en cuanto no se han realizado simulaciones con campos fluidos y datos de captaci�n reales. Es sin duda un aspecto que queda pendiente, pero una vez m�s, conviene remarcar que el modelo termodin�mico es ciertamente s�lido, o por lo menos prometedor.
\end{itemize}
%
\indent En todo caso, no hay que olvidar el incalculable valor del proyecto m�s all� del documento o de los resultados finales: el valor did�ctico. El aprendizaje obtenido alcanza no solo al nivel f�sico, num�rico o inform�tico, sino al proceso de investigaci�n, de problemas y soluciones, y de, al fin y al cabo, resoluci�n de algo nuevo. En opini�n del autor, este es el mayor resultado de todos.
%
\paragraph{\indent Futuros estudios. \\}
%
\indent Seguidamente se enumeran posibles puntos de mejora y futuros desarrollos que extiendan y prolonguen el camino avanzado por este proyecto:
%
\begin{itemize}
\item Como se indic� en las conclusiones, un primer paso ser�a observar los resultados que muestra el modelo al introducir datos reales, obtenidos del solver TAU-Code y el m�dulo GOTA. Ah� es donde se ver� el potencial del modelo.
%
\item Al igual que cualquier otro c�digo, ser� �til un replanteamiento y mantenimiento del mismo, tanto limpiando peque�as erratas o acciones innecesarias, como generalizando ciertos subprogramas. Resultar�a tambi�n muy interesante la automatizaci�n (en la medida de lo posible) de la interacci�n de este m�dulo los dem�s, as� como un posible enfoque \textit{multi-step} del proceso completo en sustituci�n del \textit{one-shot} utilizado en este caso.
%
\item Con respecto a cuestiones termodin�micas, existen varios puntos de gran margen de mejora. Las condiciones termodin�micas de compatibilidad pueden requerir un segundo estudio y un replanteamiento, y la modelizaci�n de los flujos m�sicos por evaporaci�n y sublimaci�n han resultado realmente complejas, con grandes variaciones entre unos c�digos usados de referencia y otros (LEWICE \cite{ruff} y ONERA \cite{hedde}).
%
\item En cuanto al aspecto num�rico del problema, una vez se hayan dominado y validado las ecuaciones, puede ser interesante la implementaci�n de m�todos de segundo orden en el tiempo, dado el gran hincapi� que Hirsch \cite{hirsch} realiza as� como los resultados obtenidos para los m�todos de primer orden de Euler y Runge-Kutta. Un an�lisis de la estabilidad y precisi�n de los hipot�ticos m�todos implementados resultar�a b�sico, pudiendo tomar como referencia el procedimiento realizado en el anexo A.
%
\item Como �ltima fase del desarrollo del c�digo puede establecerse la resoluci�n de problemas 3D, cuya complejidad crece a nivel f�sico (trayectorias de la pel�cula) y computacional (mallado, flujos num�ricos ...). Esto se plantea m�s a largo plazo, dada la complejidad a�adida sobre los otros m�dulos.
\end{itemize}


\nomenclature[T]{$LWC$}{Contenido en agua l�quida, \textit{liquid water content} \nomunit{$\left[\dfrac{kg}{m^3}\right]$}}%
\nomenclature[T]{$MVD$}{Tama�o volum�trico medio de las gotas, \textit{median volumetric diameter}\nomunit{[$\mu m$]}}%
\nomenclature[T]{$\beta$}{Coeficiente de captaci�n  \nomunit{[$---$]}}%
\nomenclature[T]{$\dot{m}$}{Flujo m�sico \nomunit{$\left[\dfrac{kg}{s}\right]$}}%
\nomenclature[T]{$\dot{m}'$}{Flujo m�sico por unidad de �rea \nomunit{$\left[\dfrac{kg}{s} \dfrac{1}{m^2}\right]$}}%
\nomenclature[T]{$\dot{Q}$}{Flujo de calor [$W$]}%
\nomenclature[T]{$\dot{q}$}{Flujo de calor por unidad de �rea \nomunit{$\left[\dfrac{W}{m^2}\right]$}}%
\nomenclature[T]{$\rho$}{Densidad de un fluido \nomunit{$\left[\dfrac{kg}{m^3}\right]$}}%
\nomenclature[T]{$d$}{Di�metro de las gotas  \nomunit{[$\mu m$]}}%
\nomenclature[C]{$\gamma$}{Coeficiente de dilataci�n adiab�tica de un gas \nomunit{[---]}}%
\nomenclature[C]{$T_C$}{Temperatura del punto triple \nomunit{$[K]$}}%
\nomenclature[T]{$p$}{Presi�n de un fluido \nomunit{[$Pa$]}}%
\nomenclature[T]{$p_{v, sat}$}{Presi�n de vapor de saturaci�n \nomunit{[$Pa$]}}%
\nomenclature[T]{$\mu$}{Viscosidad din�mica de un fluido \nomunit{[$Pa \cdot s$]}}%
\nomenclature[T]{$\nu$}{Viscosidad cinem�tica de un fluido \nomunit{$\left[\dfrac{m^2}{s}\right]$}}%
\nomenclature[T]{$C_D$}{Coeficiente de resistencia de un cuerpo \nomunit{[$---$]}}%
\nomenclature[T]{$\alpha_w$}{Fracci�n volum�trica de agua en aire \nomunit{[$---$]}}%
\nomenclature[T]{$\alpha$}{Difusividad t�rmica \nomunit{$\left[ \dfrac{m^2}{s}\right]$}}%
\nomenclature[T]{$\bar{U}$}{Vector velocidad de un cuerpo o fluido \nomunit{$\left[\dfrac{m}{s}\right]$]}}%
\nomenclature[T]{$\bar{u}$}{Vector velocidad adimensional de un cuerpo o fluido \nomunit{[$---$]}}%
\nomenclature[T]{$f$}{Fracci�n de agua congelada \nomunit{[$---$]}}%
\nomenclature[T]{$h$}{Espesor de una pel�cula de agua  \nomunit{[$m$]}}%
\nomenclature[T]{$T$}{Temperatura absoluta \nomunit{[$K$]}}%
\nomenclature[T]{$k$}{Conductividad t�rmica de un fluido  \nomunit{$\left[\dfrac{W}{m\,K}\right]$}}%
\nomenclature[T]{$r$}{Factor de recuperaci�n adiab�tica  \nomunit{[$---$]}}%
\nomenclature[T]{$C_p$}{Calor espec�fico a presi�n constante  \nomunit{$\left[\dfrac{J}{kg\, K}\right]$}}%
\nomenclature[T]{$h_c$}{Coeficiente de transferencia de calor por convecci�n  \nomunit{$\left[\dfrac{W}{m^2 \, K}\right]$}}%
\nomenclature[T]{$\overline{\tau}_{wall}$}{Esfuerzo viscoso de un fluido sobre una pared \nomunit{[$Pa$]}}%
\nomenclature[T]{$L$}{Calor latente de un fluido \nomunit{$\left[ \dfrac{J}{kg} \right]$}}%
\nomenclature[T]{$\overline{c}_f$}{Coeficiente de fricci�n viscosa sobre una pared \nomunit{[$---$]}}%
\nomenclature[T]{$\tilde{T}$}{Temperatura \nomunit{[$�C$]}}%
\nomenclature[T]{$\theta$}{Temperatura absoluta adimensionalizada con la del punto triple \nomunit{[$---$]}}%
\nomenclature[S]{$\infty$}{Variable en el infinito sin perturbar}%
\nomenclature[S]{$e$}{Borde de la capa l�mite (\textit{edge})}% 
\nomenclature[S]{$d$}{Gota (\textit{droplet})}%
\nomenclature[S]{$c$}{Caracter�stica}%
\nomenclature[S]{$w$}{Agua (\textit{water})}%
\nomenclature[S]{$f$}{Pel�cula de agua (\textit{water film})}%

\nomenclature[D]{$\bm U$}{Vector de variables dependientes de un sistema de ecuaciones diferenciales}%\\
\nomenclature[D]{$\bm A$}{Matriz del sistema de una ecuaci�n diferencial}%
\nomenclature[D]{$\bm F(\bm U)$}{Vector de flujos de una ecuaci�n diferencial}%
\nomenclature[D]{$\bm R$}{Vector de t�rminos fuente de una ecuaci�n diferencial}%
\nomenclature[D]{$\bm U$}{Vector de variables dependientes de una ecuaci�n diferencial}%
\nomenclature[D]{$\bm I$}{Matriz identidad}%
\nomenclature[D]{$\bm K^{(i)}$}{Autovector \textit{i}-�simo de la matriz del sistema}%
\nomenclature[D]{$\bm \lambda_i$}{Autovalor \textit{i}-�simo de la matriz del sistema}%
\nomenclature[D]{$\mathcal{L}(\bm U)$}{Operador diferencial espacial de una ecuaci�n diferencial en derivadas parciales}%
\nomenclature[D]{$\phi$}{Superficie caracter�stica de una ecuaci�n diferencial en derivadas parciales}%
\nomenclature[D]{$\Sigma$}{Superficie geom�trica}%
\nomenclature[D]{$\Omega$}{Volumen geom�trico}%
\nomenclature[D]{$\partial \Omega$}{Frontera del volumen $\Omega$}%
\nomenclature[D]{$\bm K^{(i)}$}{Autovector \textit{i}-�simo de la matriz del sistema}%
\nomenclature[D]{$d\gamma, \: d\sigma, \: d\omega$}{Diferenciales de arco, superficie y volumen}%
\nomenclature[D]{$\bm W$}{Vector de variables dependientes can�nicas de un sistema de ecuaciones diferenciales}%
\nomenclature[D]{$\bm \Lambda$}{Matriz de autovalores de un sistema de ecuaciones diferenciales}%
\nomenclature[D]{$s$}{Par�metro de longitud de arco}%
\nomenclature[D]{$\bm n$}{Vector normal exterior a una superficie}%
\nomenclature[D]{$\mathbb{I}$}{Unidad imaginaria}%

\nomenclature[G]{$\sigma$}{N�mero CFL}%
\nomenclature[G]{$\Omega_i$}{Volumen de control \textit{i}-�simo}%
\nomenclature[G]{$u_i^n$}{Soluci�n computada de un esquema num�rico}%
\nomenclature[G]{$\bar{u}_i^n$}{Soluci�n exacta de un esquema num�rico}%
\nomenclature[G]{$\tilde{u}_i^n$}{Soluci�n exacta de un modelo matem�tico}%
\nomenclature[G]{$\bm U_{i+1/2}$}{Soluci�n computada en el extremo superior del volumen finito \textit{i}-�simo}%
\nomenclature[G]{$x_{i + 1/2}$}{Extremo superior del volumen finito \textit{i}-�simo}%
\nomenclature[G]{$\bm F_{i + 1/2}$}{Flujos num�ricos en el extremo superior del volumen finito \textit{i}-�simo}%
\nomenclature[G]{$\Delta x, \: \Delta t$}{Pasos espacial y temporal}%
\nomenclature[G]{$S_{max}^n$}{Velocidad m�xima de propagaci�n de informaci�n en un problema de evoluci�n discretizado}
\nomenclature[G]{$\epsilon_T$}{Error de truncamiento de un esquema num�rico}%
\nomenclature[G]{$N(\bullet)$}{Esquema num�rico}%
\nomenclature[G]{$\tilde{\bm A}$}{Matriz promediada del sistema}%
\nomenclature[G]{$\tilde{\lambda}_i$}{Autovalor \textit{i}-�simo de la matriz promediada del sistema}%
\nomenclature[G]{$\tilde{K}^{(i)}$}{Autovector \textit{i}-�simo de la matriz promediada del sistema}%
\nomenclature[G]{$\bm Q$}{Vector de par�metros del m�todo de Roe}%
\nomenclature[G]{$I_{i}$}{Centroide del volumen finito \textit{i}-�simo}%
\nomenclature[G]{$\lambda_j$}{Longitud de onda del arm�nico \textit{j}-�simo}%
\nomenclature[G]{$k_j$}{N�mero de onda del arm�nico \textit{j}-�simo}%
\nomenclature[G]{$\phi_j$}{Fase del arm�nico \textit{j}-�simo}%
\nomenclature[G]{$V_j^n$}{Amplitud del arm�nico \textit{j}-�simo en el instante \textit{n}-�simo}%
\nomenclature[G]{$G_j$}{Factor de amplificaci�n o ganancia del arm�nico \textit{j}-�simo }%
\nomenclature[G]{$\omega $}{Relaci�n de dispersi�n num�rica}%
\nomenclature[G]{$\tilde{\omega} $}{Relaci�n de dispersi�n}%
\nomenclature[G]{$S$}{Velocidad de propagaci�n de una discontinuidad}%

\nomenclature[N]{$Re$}{N�mero de Reynolds \nomunit{$Re = \dfrac{\rho_c U_c L_c}{\mu_c}$}}%
\nomenclature[N]{$Pr$}{N�mero de Prandtl \nomunit{$Pr = \dfrac{\nu}{\alpha}$}}%
\nomenclature[N]{$Nu$}{N�mero de Nusselt \nomunit{$Nu = \dfrac{h_c L_c}{k}$}}%
\nomenclature[N]{$Fr$}{N�mero de Froude \nomunit{$Fr = \dfrac{U_c}{\sqrt{g_0 L_c}}$}}%










\appendix

\chapter{Absolute and relative orbital element sets.}
%
\label{chap: App_OEs}
%
\section{Introduction.}
%
\indent The description of a spacecraft's state is done via a \textbf{state vector}. While it can include several variables with other purposes (\eg filtering), its only information throughout this thesis is the position and velocity. There are two main ways to describe them:\\
%
\begin{itemize}
\item[A.] Through cartesian coordinates
\item[B.] Through orbital elements
\end{itemize}
%
\indent While the first option yields a very explicit and graphic-ready description, the second one usually has two advantages over it. Firstly, orbital elements are generally more intuitive about both the orbit and the position on it. Secondly, as orbital elements are generally slow-varying, they allow for a bigger integration timestep without losing accuracy. This is quite clear when studying keplerian motion, as most of the elements remain constant. Variational formulation and Hamilton-Jacobi theory (with the notion of changing variables as the full solution of a problem) relate to this fact. \\
%
\indent Throughout this thesis, several sets of orbital elements have been used. The goal of this appendix is to clarify on the definition and differences between them. Absolute orbital elements (OEs) will be described first, followed by relative OEs (ROEs).
%
\section{Absolute sets.}
%
	\subsection{Keplerian orbital elements (KOE).}
	%
	\indent The Keplerian set of OEs (KOE) is one of the most widely used and classic options. While the last element may change from author to author, an usual definition is the following:
	%
	\begin{equation}
	\left\{ 
	\begin{array}{lll}
	a & \equiv & \text{Semimajor axis}\\
	e & \equiv & \text{Eccentricity}\\
	i & \equiv & \text{Inclination}\\
	\Omega \text{or} RAAN & \equiv & \text{Right ascension of the ascending node}\\
	\omega & \equiv & \text{Argument of perigee}\\
	M & \equiv & \text{Mean anomaly}\\
	\end{array}
	\right\}
	\end{equation}
	%
	\subsection{Quasi-nonsingular orbital elements (QNSOE).}
	%
	%
	\subsection{Equinoctial orbital elements (EOE).}
	%
	%
	\subsection{Delaunay orbital elements (DOE).}
	%
	%
%
\section{Relative sets.}
%
% General workflow and explicit definition
%
	\subsection{Eccentricity/inclination vectors relative orbital elements (EIROE).}
	%
	%
	\subsection{Peters-Noomen C set of relative orbital elements (CROE).}
	%
	%
	\subsection{General workflow for arbitrary ROEs.}
	%
	%
%
%\chapter{Soluci�n anal�tica de las ecuaciones de Euler.}
%
\label{chap: App_euler}
%
\section{Introducci�n.}
%
\indent Las ecuaciones de Euler describen el movimiento de los denominados fluidos ideales, es decir, suponiendo negligibles la viscosidad y la conducci�n de calor del fluido. Ello estar� justificado siempre y cuando se cumplan una serie de condiciones respecto a los n�meros de Reynolds y de Prandtl, y suponen una simplificaci�n radical respecto a las ecuaciones de Navier-Stokes, al desaparecer las derivadas de orden superior y con ello, el car�cter parab�lico o irreversible de las mismas. Dicha simplificaci�n, pese a ser notoria a nivel matem�tico, sigue permitiendo obtener unos resultados (bajo ciertas premisas) bastante cercanos a la realidad.\\
%
\indent Sin embargo, la soluci�n anal�tica para estas ecuaciones se reduce a unos pocos casos, como puede ser el flujo unidimensional o en secciones lentamente variables. En el presente informe, se tratar� de resolver las ecuaciones de Euler reducidas a una dimensi�n (eje x), teniendo en cuenta por tanto solamente derivadas en dicha dimensi�n, pero resolviendo para todas las direcciones del espacio. En particular, se resolver� el problema de Riemann, caracterizado por la estructura de la condici�n inicial, compuesta por dos estados constantes separados por una discontinuidad. La soluci�n anal�tica de dicho problema es conocida, y se utilizar� para validar soluciones obtenidas por v�as distintas. \\
%
\indent El objetivo principal de este informe ser� validar el denominado m�todo de Roe, un m�todo aproximado para resolver problemas de Riemann de forma num�rica. Se establecer�n las bases y los precedentes de dicho m�todo para finalmente comparar las soluciones obtenidas con la soluci�n anal�tica, a fin de validar la implementaci�n del m�todo en el c�digo desarrollado y poder aplicarlo con confianza en futuros problemas.
\section{Ecuaciones de Euler y problema de Riemann.}
%
En general, un problema de Cauchy para un sistema de leyes de conservaci�n hiperb�lico en derivadas parciales  toma la siguiente forma:
%
\begin{equation} 
 [P]\left\{\begin{array}{l}
PDEs:{\bm U_t} + \bm F(\bm U)_x=\bm 0\\[1em]
ICs: {\bm U}(x,0) = \bm U^{0}(x)\\
BCs: \bm U(0, t) = \bm U_l (t), \quad \bm U(L, t) = \bm U_r(t) \\
\mathbb{D}: (x, t) \in [0, L] \times [0, \infty[
\end{array} \right.
\label{eqn: RP_general}
\end{equation}
%
\indent La ecuaci�n en s� se denomina ley de conservaci�n del sistema, tomando el significado f�sico que se ver� m�s adelante. En el caso de un problema de Riemann, la condici�n inicial pasa a ser una combinaci�n de dos estados constantes. Adem�s, suponiendo un dominio espacial infinito, las condiciones de contorno `` desaparecen", de modo que:
%
\begin{equation} 
 [P]\left\{\begin{array}{l}
PDEs:{\bm U_t} + \bm F(\bm U)_x=\bm 0\\[1em]
ICs: {\bm U}(x,0) = \left\{\begin{array}{l}
\bm U_L \quad si \quad x<0\\
\bm U_R \quad si \quad x>0
\end{array}\right.
 
\end{array} \right.
\end{equation}
%
\indent El objetivo es resolver la anterior ecuaci�n restringi�ndola a un cierto dominio abarcable (por ejemplo, desde -1 a 1) y para una ley de conservaci�n particular: las ecuaciones de Euler. Ellas quedan caracterizadas definiendo el vector de variables conservadas \bm U y la funci�n de flujo \bm F(\bm U). Es interesante recordar que estas ecuaciones est�n formuladas en variables conservadas(\bm U), que no deben ser confundidas con las variables primitivas o termodin�micas (\bm W). Esto cobrar� importancia m�s adelante, ya que los problemas num�ricos se resuelven en variables conservadas, mientras que la soluci�n anal�tica de esta ecuaci�n se conoce en variables primitivas. En este caso:\\
%
\[
{\bm U} = \left[\begin{array}{c}
\rho \\
\rho u\\
\rho v\\
\rho w\\
E
\end{array}\right] \qquad {\bm F} = \left[\begin{array}{c}
\rho u \\
\rho u^2+p\\
\rho uv\\
\rho uw\\
u(E+p)
\end{array}\right] \qquad
{\bm W} = \left[\begin{array}{c}
\rho \\
u\\
v\\
w\\
p
\end{array}\right]
\]
%
\indent A la hora de relacionar las variables conservadas y primitivas, son de aplicaci�n una serie de definiciones termodin�micas acompa�adas de las ecuaciones de estado (EOS). En este caso, suponiendo gases ideales y calor�ficamente perfectos, las relaciones son las siguientes:\\
%
\[E = \rho e + \dfrac{1}{2} \rho  \bm V  ^{2}, \quad  \bm V  ^{2} = u^2 + v^2 + w^2, \quad e =\dfrac{p}{(\gamma - 1) \rho}, \quad a^{2} = \gamma \dfrac{p}{\rho}, \quad H = \dfrac{E + p}{\rho}\]
\section{Soluci�n anal�tica.}\label{section: solucion analitica}
%
\subsection{Estructura de la soluci�n.}
\indent La soluci�n anal�tica de este problema es ampliamente conocida. Se trata de una generalizaci�n del denominado ``tubo de choque" (\textit{shock tube}), que modeliza el comportamiento de dos gases est�ticos separados por una membrana. La estructura de la soluci�n es un sistema de ondas, formado por tres discontinuidades. La soluci�n m�s general del problema se compone de una combinaci�n de dos ondas (ondas de choque, rarefacciones o ambas), separadas en todo caso por una discontinuidad de contacto. En la figura \ref{fig: Estructura_analitica} se observa dicha estructura, que divide el dominio (x, t) en, a priori, cuatro subdominios:
%
\begin{itemize}
\item La regi�n izquierda, donde la soluci�n coincide con la soluci�n inicial en $x<0$ ($\bm W_L$),
\item La regi�n derecha, donde la soluci�n coincide con la soluci�n inicial en $x>0$ ($\bm W_R$),
\item La regi�n estrella (\textit{star region}), encapsulada entre ambas ondas, y subdividida en:
\begin{itemize}
\item[\labelitemii] Regi�n estrella izquierda, comprendida entre la onda izquierda y la discontinuidad de contacto,
\item[\labelitemii] Regi�n estrella derecha, comprendida entre la discontinuidad de contacto y la onda derecha. 
\end{itemize}
\end{itemize}
%
\begin{figure}[ht]
\centering\includegraphics[width = 0.5\linewidth]{Appendix/Appendix_02/Estructura_analitica}
\caption{Estructura gen�rica de la soluci�n.}
\label{fig: Estructura_analitica}
\end{figure}
%
\indent Las posibles variantes de la soluci�n en funci�n de las ondas presentadas en cada lado se muestran a continuaci�n (\ref{fig: Estructura_multiples}), donde las l�neas arqueadas representan una rarefacci�n, la l�nea gruesa una onda de choque, y la discontinua una discontinuidad de contacto. Conviene remarcar que, en caso de implementar ciertos esquemas num�ricos, es necesario un an�lisis adicional, gener�ndose otras cuatro estructuras en el caso de una rarefacci�n s�nica. Ello tendr� implicaciones discutidas m�s adelante.
%
\begin{figure}[ht]
\centering\includegraphics[width = 0.7\linewidth]{Appendix/Appendix_02/Estructura_multiple}
\caption{Estructuras posibles de la soluci�n a efectos de la soluci�n anal�tica.}
\label{fig: Estructura_multiples}
\end{figure}
%
\subsection{Soluci�n en la regi�n estrella.}
%
\indent Un an�lisis basado en la estructura de las ecuaciones revela que tanto la presi�n como la velocidad en la regi�n estrella ($p_{\star}, u_{\star}$) entre ambas ondas son constantes, mientras que la densidad s� var�a a ambos lados de la discontinuidad de contacto ($\rho_{\star L}, \rho_{\star R}$), de modo que se buscan cuatro magnitudes f�sicas. El proceso comienza por obtener la presi�n y la velocidad, tras lo cual y en funci�n de la estructura de la soluci�n, se calculan las densidades.
%
\subsubsection{Obtenci�n de la presi�n y velocidad.}
%
\indent Tras analizar las evoluciones de las magnitudes fluidas en ambos tipos de discontinuidades, se concluye que la presi�n en la regi�n estrella satisface la siguiente ecuaci�n:
%
\begin{equation}
f(p, \bm W_L, \bm W_R) \equiv f_{L}(p, \bm W_L) + f_{R}(p, \bm W_R) + \Delta u = 0, \quad\Delta u = u_R - u_L 
\label{eq: Pressure}
\end{equation}
%
\indent Donde las funciones $f_L$ y $f_R$ responden a las siguientes expresiones:
%
\begin{equation}
f_{k}(p, \bm W_{k}) \left\{ \begin{array}{cccc}
(p - p_k) \left[ \dfrac{A_k}{p + B_k}\right] ^{1/2}  & $si$ & \enskip p > p_k & $(onda de choque)$\\ [1.5em]
\dfrac{2a_k}{\gamma - 1} \left[ \left(\dfrac{p}{p_k}\right)^{\dfrac{\gamma - 1}{2\gamma}} - 1\right] & $si$ &\enskip p \leq p_k & $(rarefacci�n$)
\end{array}\right.
\end{equation}
%
\indent Donde el sub�ndice k hace las veces de L y R. Los coeficientes $A_k$ y $B_k$ se calculan como:
\[A_{k} = \dfrac{2}{(\gamma + 1) \rho_{k}}, \quad B_{k} = \dfrac{(\gamma - 1)}{(\gamma + 1)} p_{k} \]
%
\indent Para una deducci�n detallada de ambas funciones as� como de los coeficientes, se recomienda acudir a \cite{toro}, apartado 4.2. La presi�n, por tanto, se calcula resolviendo de forma num�rica la ecuaci�n \ref{eq: Pressure}, en particular con el m�todo de Newton-Raphson, previo an�lisis de la monoton�a y concavidad de la misma. Conocida la presi�n, la velocidad del fluido se calcula como:
%
\begin{equation}
u_{\star} = \dfrac{1}{2}\left(u_L + u_R\right) + \dfrac{1}{2} \left[f_R(p_{\star}) - f_L(p_{\star}\right)]
\end{equation}
%
\subsubsection{Obtenci�n de las densidades.}
%
\indent Las densidades en ambas regiones se calculan de distinta manera dependiendo del fen�meno que suceda en la discontinuidad adyacente. De todos modos, las expresiones s�lo dependen del tipo de discontinuidad y del estado adyacente, de modo que se pueden plantear las expresiones generales (k = R, L):
%
\paragraph{\indent Onda de choque:} Aplicando la relaci�n de Rankine-Hugoniot, se obtiene:
%
\begin{equation}
\rho_{\star k} = \rho_{k} \left[ \dfrac{\left(\dfrac{\gamma - 1}{\gamma + 1}\right) + \left(\dfrac{p_{\star}}{p_L}\right)}{\left(\dfrac{\gamma - 1}{\gamma + 1}\right) \left(\dfrac{p_{\star}}{p_L}\right)+1}\right]
\end{equation}
%
\paragraph{\indent Rarefacci�n:} Dado el car�cter isentr�pico de la evoluci�n:
%
\begin{equation}
\rho_{\star k} = \rho_{k} \left(\dfrac{p_{\star}}{p_{k}}\right)^{\dfrac{1}{\gamma}}
\end{equation}
%
\subsubsection{Obtenci�n de las caracter�sticas y soluci�n dentro de una rarefacci�n.}
\indent Las �ltimas magnitudes a obtener para poder muestrear la soluci�n completa del problema son las caracter�sticas de cada discontinuidad, que responden a las siguientes expresiones:
%
\begin{figure}
\centering\includegraphics[width = 0.8\linewidth]{Appendix/Appendix_02/Caracteristicas}
\caption{Esquema de las caracter�sticas de cada una de las posibles discontinuidades.}
\label{fig: Caracteristicas}
\end{figure}
%
\paragraph{\indent Onda de choque izquierda. \medskip \\} 
%
\indent Se corresponde con la figura \ref{fig: Caracteristicas} (a). Queda completamente caracterizada con el c�lculo de la velocidad de la onda, resultando ser: 
%
\begin{equation}
S_L = u_L - a_L \left[\dfrac{\gamma + 1}{2\gamma} \dfrac{p_{\star}}{p_L} + \dfrac{\gamma - 1}{2\gamma} \right]^{\dfrac{1}{2}}
\end{equation}
%
\paragraph{\indent Rarefacci�n izquierda. \medskip \\} 
%
\indent Representado en la figura \ref{fig: Caracteristicas} (b), queda definida con el c�lculo de las velocidades de las caracter�sticas de cabeza (\textit{head}) y de cola (\textit{tail}):
%
\begin{equation}
S_{HL} = u_L - a_L, \quad S_{TL} = u_{\star} - a_{\star L}, \quad a_{\star L} = a_L \left(\dfrac{p_{\star}}{p_L}\right)^{\dfrac{\gamma - 1}{2\gamma}}
\end{equation}
%
\indent En este caso, la soluci�n entre la cabeza y la cola de la rarefacci�n difiere de ambas regiones lindantes. La soluci�n en un punto $(x,t)$ responde a:
%
\begin{equation}
\bm W_{Lfan} = \left\{\begin{array}{cll}
\rho & = & \rho_L \left[\dfrac{2}{\gamma + 1}+ \dfrac{\gamma-1}{(\gamma + 1)a_L} \left(u_L - \dfrac{x}{t}\right)\right]^{\dfrac{2}{\gamma - 1}}\\[1.5em]
u & = & \dfrac{2}{\gamma + 1}\left[ a_L + \dfrac{\gamma - 1}{2}u_L + \dfrac{x}{t}\right]\\[1.5em]
p & = & p_L \left[\dfrac{2}{\gamma + 1}+ \dfrac{\gamma-1}{(\gamma + 1)a_L} \left(u_L - \dfrac{x}{t}\right)\right]^{\dfrac{2\gamma}{\gamma - 1}}
\end{array}\right.
\end{equation}
%
\paragraph{\indent Onda de choque derecha \medskip \\}
%
\begin{equation}
S_R = u_R + a_R \left[\dfrac{\gamma + 1}{2\gamma} \dfrac{p_{\star}}{p_R} + \dfrac{\gamma - 1}{2\gamma} \right]^{\dfrac{1}{2}}
\end{equation}
\paragraph{\indent Rarefacci�n derecha \medskip \\}
\indent Se calculan la cabeza y la cola de la rarefacci�n (ver figura \ref{fig: Caracteristicas} (d)): 
%
\begin{equation}
S_{HL} = u_L - a_L, \quad S_{TL} = u_{\star} - a_{\star L}, \quad a_{\star L} = a_L \left(\dfrac{p_{\star}}{p_L}\right)^{\dfrac{\gamma - 1}{2\gamma}}
\end{equation}
%
\indent An�logamente al caso izquierdo:
%
\begin{equation}
\bm W_{Rfan} = \left\{\begin{array}{cll}
\bigbreak
\rho & = & \rho_R \left[\dfrac{2}{\gamma + 1} - \dfrac{\gamma-1}{(\gamma + 1)a_R} \left(u_R - \dfrac{x}{t}\right)\right]^{\dfrac{2}{\gamma - 1}}\\[1.5em]
u & = & \dfrac{2}{\gamma + 1}\left[ - a_R + \dfrac{\gamma - 1}{2}u_R + \dfrac{x}{t}\right]\\[1.5em]
p & = & p_R \left[\dfrac{2}{\gamma + 1}- \dfrac{\gamma-1}{(\gamma + 1)a_R} \left(u_R - \dfrac{x}{t}\right)\right]^{\dfrac{2\gamma}{\gamma - 1}}
\end{array}\right.
\end{equation}
%
	\subsection{Soluci�n completa. Muestreo de la soluci�n.}
	%
\indent Sup�nganse conocidas las variables termodin�micas en la totalidad del espacio de muestreo, es decir, densidades, velocidades y presiones en cada una de las regiones descritas anteriormente; as� como las curvas caracter�sticas del problema. El objetivo de este apartado es, dado un punto determinado por (x,t), calcular la soluci�n en �l, es decir, el vector de variables primitivas $\bm W = (\rho, u, p)^{T}$. En este momento es importante remarcar que la soluci�n es de tipo auto-semejante, lo que quiere decir que el sistema carece de una longitud caracter�stica o escala de tiempo, de modo que la soluci�n es semejante consigo misma se represente en la escala que se represente. Ello permite muestrear la soluci�n en un punto caracteriz�ndolo �nicamente con el valor $S  = x/t$, com�nmente denominada velocidad del punto.\\
%
\indent Se ir� localizando el punto con relaci�n a las referencias del espacio soluci�n de manera sucesiva, tomando diversas referencias.
%
		\subsubsection{Parte izquierda de la discontinuidad de contacto: $x/t < u_{\star}$}
		%
		\indent Tal y como se muestra en la figura \ref{fig: Caracteristicas} (a) y (b), existen dos posibles soluciones en funci�n del tipo de discontinuidad:
		%
		\paragraph{\indent A: Onda de choque izquierda. \medskip \\} 
		%
		\indent La condici�n es $p_{\star} > p_L$. Es necesario determinar si est� a la derecha o a la izquierda de la discontinuidad:
		%
		\begin{equation}
		\bm W(x,t) \left\{\begin{array}{lcc} 
		\bm W_{\star L}^{OCH} & $si$ &\quad S_L \leq \dfrac{x}{t} \leq u_{\star}\\
		\bm W_L & $si$ &\dfrac{x}{t}\leq S_L
		\end{array}\right.
		\end{equation}
		%
		\paragraph{\indent B: Rarefacci�n izquierda. \medskip \\} 
		%
		\indent La condici�n es $p_{\star} \leq p_L$. En este caso, como se observa en \ref{fig: Caracteristicas}(b), surgen 3 regiones:
		%
		\begin{equation}
		\bm W(x,t) \left\{\begin{array}{lcc} 
		\bm W_{L} & $si$ &\quad \dfrac{x}{t} \leq S_{HL}\\
		\bm W_{Lfan} & $si$ &\quad S_{HL} \leq \dfrac{x}{t}\leq S_{TL}\\
		\bm W_{L}^{RF} & $si$ & \quad S_{TL} \leq \dfrac{x}{t}\leq u_{\star}
		\end{array}\right.
		\end{equation}
		%
		\subsubsection{Parte derecha de la discontinuidad de contacto: $x/t > u_{\star}$}
		%
		\indent An�logamente al apartado anterior:
		%
		\paragraph{\indent A: Onda de choque derecha. \medskip \\} La condici�n es $p_{star} > p_R$. Es necesario determinar si est� a la derecha o a la izquierda de la discontinuidad:
		\begin{equation}
		\bm W(x,t) \left\{\begin{array}{lcc} 
		\bm W_{\star R}^{OCH} & $si$ &\quad u_{\star} \leq \dfrac{x}{t} \leq S_R\\
		\bm W_R & $si$ &\dfrac{x}{t}\geq S_R
		\end{array}\right.
		\end{equation}
		\paragraph{\indent B: Rarefacci�n derecha. \medskip \\} La condici�n es $p_{star} \leq p_R$:
		\begin{equation}
		\bm W(x,t) \left\{\begin{array}{ccc} 
		\bm W_{\star R}^{RF} & $si$ &\quad u_{\star} \leq \dfrac{x}{t} \leq S_{TR}\\
		\bm W_{Rfan} & $si$ &\quad S_{TR} \leq \dfrac{x}{t}\leq S_{HR}\\
		\bm W_{\star L}^{RF} & $si$ & \quad \dfrac{x}{t}\geq S_{HR}
		\end{array}\right.
		\end{equation}
		%
		\indent A modo de resumen visual y pseudo-c�digo, se adjuntan sendos diagramas de flujo para ambos lados de la discontinuidad de contacto en la figura \ref{fig: FLOW_CHARTS}.
		%
		\begin{figure}[ht]
		\centering
		\medskip
		\begin{subfigure}[t]{.7\linewidth}
		\centering\includegraphics[width=\linewidth]{Appendix/Appendix_02/Flow_chart_left}
		\caption{Caso $S<u_{\star}$ (izquierda).}
		\label{fig: Flow_chart_left}
		\end{subfigure}
		\begin{subfigure}[t]{.7\linewidth}
		\centering\includegraphics[width=\linewidth]{Appendix/Appendix_02/Flow_chart_Right}
		\caption{Caso $S>u_{\star}$ (derecha).}
		\label{fig Flow_chart_right}
		\end{subfigure}
		\caption{Diagramas de flujo para el muestreo de la soluci�n.}
		\label{fig: FLOW_CHARTS}
		\end{figure}
%
\clearpage
\subsection{Resultados obtenidos.}
%
\indent Para probar el correcto funcionamiento del m�todo de Roe, se utilizar�n una serie de tests extra�dos de Toro \cite{toro}, siendo en particular:
%
\begin{table}[ht]
\begin{center}
\begin{tabular}{|c|c|c|c|c|c|c|}
\hline Test & \(\rho_{\mathrm{L}}\) & \(u_{\mathrm{L}}\) & \(p_{\mathrm{L}}\) & \(\rho_{\mathrm{R}}\) & \(u_{\mathrm{R}}\) & \(p_{\mathrm{R}}\) \\
\hline 1 & 1.0 & -0.75 & 1.0 & 0.125 & 0.0 & 0.1 \\
\hline 2 & 1.0 & -2.0 & 0.4 & 1.0 & 2.0 & 0.4 \\
\hline 3 & 1.0 & 0.0 & 1000.0 & 1.0 & 0.0 & 0.01 \\
\hline 4 & 5.99924 & 19.5975 & 460.894 & 5.99242 & -6.19633 & 46.0950 \\
\hline 5 & 1.0 & -19.5975 & 1000.0 & 1.0 & -19.5975 & 0.01 \\
\hline
\end{tabular}
\caption{Datos para los ensayos.}
\label{table: Tests_EULER}
\end{center}
\end{table}\\
%
\indent Es interesante remarcar que la elecci�n del tiempo final en cada caso se realiza de acuerdo a los �rdenes de magnitud de las variables fluidas. Dado que en los m�todos num�ricos el paso temporal es inversamente proporcional al valor m�ximo de las variables fluidas, con condiciones iniciales de valores superiores al orden unidad (v�anse tests 3, 4 o 5), el n�mero de pasos temporales crece de forma exorbitada, proporcionalmente a su orden de magnitud (millares). Es por eso por lo que se opta por tiempos finales cortos en dichos casos.
%
\subsubsection{Test 1.}
%
\indent Com�nmente denominado test de \textit{Sod}, supone una prueba bastante permisiva para los m�todos num�ricos, consistiendo en una rarefacci�n izquierda acompa�ada de una onda de choque. Existen dos variantes: El test de Sod original, el cual supone velocidad nula en ambos lados ($u_{L} = 0$) y el modificado, que es el mostrado en la tabla \ref{table: Tests_EULER} como (1).\\
\begin{figure}[ht]
\centering\includegraphics[width = 0.8\linewidth]{Appendix/Appendix_02/Subplot_1_SOD}
\caption{Soluci�n anal�tica del test de Sod (sin modificar), con $t_f = 0.25$.}
\label{fig: SOD_analytical}
\end{figure}
\begin{figure}[ht]
\centering\includegraphics[width = 0.8\linewidth]{Appendix/Appendix_02/Subplot_1_SOD_mod}
\caption{Soluci�n anal�tica del test de Sod modificado, con $t_f = 0.25$.}
\label{fig: SODmod_analytical}
\end{figure}
%
\indent En la figura \ref{fig: SOD_analytical} se puede observar en todas las variables un comportamiento coherente, donde se manifiesta la rarefacci�n a la izquierda mediante un salto progresivo en las variables fluidas, as� como un salto brusco en la derecha, correspondi�ndose con una onda de choque. Por otra parte, en la figura \ref{fig: SODmod_analytical}, los resultados son pr�cticamente iguales, salvo que la velocidad es mayor en la parte izquierda (lo cual es l�gico a la vista de las condiciones iniciales). 
%
\clearpage
\subsubsection{Test 2.}
%
\indent Se trata del llamado test 123, exigente dadas las bajas densidades y presiones presentes, existiendo dos rarefacciones realmente fuertes:
\begin{figure}[ht]
\centering\includegraphics[width = 0.8\linewidth]{Appendix/Appendix_02/Subplot_test2}
\caption{Soluci�n anal�tica del test 2, con $t_f = 0.25$.}
\label{fig: Test_2_analytical}
\end{figure}\\
%
\indent Los resultados (\ref{fig: Test_2_analytical}) muestran de forma clara dicho comportamiento, con evoluciones progresivas en las variables fluidas, como corresponde a rarefacciones. Podr�a interpretarse como una expansi�n de una zona central de bajas presiones hacia ambos costados, lo cual es corroborado por la asimetr�a en la velocidad del fluido.
%

\subsubsection{Test 3.}
\indent Se trata de un test bastante severo con m�todos poco adecuados. Tal y como se ve, se compone de una onda de choque derecha muy fuerte (M = 198) y una rarefacci�n izquierda. La soluci�n es en cierto modo similar al test de Sod, pero con gradientes en las variables mucho m�s notorios, abocando a discontinuidades mucho m�s salvajes.
\clearpage
\begin{figure}[ht]
\centering\includegraphics[width = 0.75\linewidth]{Appendix/Appendix_02/Subplot_test3_analytical}
\caption{Soluci�n anal�tica del test 3, con $t_f = 0.012$.}
\label{fig: Test_3_analytical}
\end{figure}
%
\subsubsection{Test 4.}
%
\indent Se observa la existencia de dos ondas de choque, correspondi�ndose con sendos saltos bruscos en las variables, que adem�s se desplazan en todo caso hacia la derecha del campo fluido (siendo la velocidad consistentemente positiva).\\
\begin{figure}[ht]
\centering\includegraphics[width = 0.75\linewidth]{Appendix/Appendix_02/Subplot_test4_analytical}
\caption{Soluci�n anal�tica del test 4, con $t_f = 0.05$.}
\label{fig: Test_4_analytical}
\end{figure}
%
\clearpage
\subsubsection{Test 5.}
%
\indent En la figura \ref{fig: Test_5_analytical} se aprecian una rarefacci�n en la izquierda del campo fluido, acompa�ada de una onda de choque centrada en el origen. Se trata de un test exigente una vez m�s dados los gradientes existentes en el problema.\\
\begin{figure}[ht]
\centering\includegraphics[width = 0.8\linewidth]{Appendix/Appendix_02/Subplot_test5_analytical}
\caption{Soluci�n anal�tica del test 5, con $t_f = 0.012$.}
\label{fig: Test_5_analytical}
\end{figure}

%\chapter{Desarrollo de software.}
%
\section{Introducci�n.}
\indent Una parte vital a la hora de realizar an�lisis y simulaciones num�ricas, sean del tipo que sean, es la elecci�n del lenguaje de c�digo para su implementaci�n. El abanico es ampl�simo, pero dada su robustez y el conocimiento previo, se decide utilizar FORTRAN 95 para la totalidad de los problemas simulados en el trabajo. Adicionalmente, por su gran adaptabilidad y variedad de opciones con respecto a las representaciones gr�ficas, se ha trabajado con Matlab, que si bien cuenta con innumerables aplicaciones y herramientas, solamente se ha utilizado a estos efectos.\\
%
\indent Por tanto, para la resoluci�n de cualquiera de los problemas desarrollados a lo largo del proyecto, la simulaci�n ha sido realizada con el lenguaje FORTRAN, de la mano con el editor de c�digo Visual Studio. Mediante un conjunto de subrutinas, los datos num�ricos obtenidos se exportan a un fichero \textit{.m}, de tipo Matlab. En dicho fichero, adem�s de los propios datos num�ricos, se incluyen todas las �rdenes necesarias para la representaci�n gr�fica de dichos datos (ejes, t�tulos, leyendas, formato...). Una alternativa, tambi�n abordada, es la utilizaci�n de ficheros de tipo \textit{tecplot}, aunque no se desarroll� en profundidad en vista de la bondad del programa Matlab.\\
%
\indent A continuaci�n se va a dar una visi�n general sobre el c�digo desarrollado asociado al m�dulo termodin�mico de la formaci�n de hielo, a fin de evaluar su funcionamiento b�sico.
%
\section{Programa $\text{prICE\_K}$.}
%
\indent El programa $\text{prICE\_K}$ constituye el c�digo del m�dulo termodin�mico de la simulaci�n de la formaci�n de hielo. Su objetivo es, por tanto, dados unos datos del campo fluido y otros del impacto de gotas, as� como unos ciertos par�metros, la obtenci�n de la masa de hielo formada y su distribuci�n sobre la geometr�a estudiada (bidimensional en todo caso). A continuaci�n se adjuntan diversos diagramas de flujo que describen su funcionamiento b�sico.
%
	\subsection{Diagrama de flujo general.}
	%
	\indent En la figura \ref{fig: DFLUJO_A} se observa un diagrama de flujo que representa el funcionamiento del programa completo, desde la carga del \textit{setup} hasta la representaci�n de los resultados, distingui�ndose las fase de pre-proceso, computaci�n y post-proceso.
	%
	\begin{figure}[ht]
	\centering\includegraphics[width = \linewidth]{Appendix/Appendix_03/DFLUJO_A}
	\caption{Diagrama de flujo del funcionamiento global del programa.}
	\label{fig: DFLUJO_A}
	\end{figure}
	%
	\subsection{Diagrama de flujo de la actualizaci�n de la soluci�n.}
	%
	\indent Dada la complejidad de la obtenci�n de la soluci�n en este problema, se incluye en la figura \ref{fig: DFLUJO_B} un diagrama de flujo de detalle sobre el proceso de actualizaci�n de la soluci�n:
	%
	\begin{figure}[ht]
	\centering\includegraphics[width =0.9 \linewidth]{Appendix/Appendix_03/DFLUJO_B}
	\caption{Diagrama de flujo del proceso de actualizaci�n de la soluci�n.}
	\label{fig: DFLUJO_B}
	\end{figure}
	%
	\subsection{Descripci�n de los m�dulos del programa.}
	%
	\indent Por el mayor orden que confiere al c�digo en su conjunto, se ha optado por una programaci�n de car�cter modular. En particular, los m�dulos de que est� compuesto el programa y sus funciones b�sicas son:
	%
	\begin{itemize}
	\item[I.] $\textbf{prICE\_K}$: Se trata del programa principal (\textit{main}), y sus funciones son, a grandes rasgos, orquestar el funcionamiento global del programa. Corre el \textit{setup}, inicializa las variables del mallado y de la soluci�n, asimila los datos de entrada del GOTA y del TAU, y llama al \textit{solver}, y por �ltimo, exporta los datos a un fichero Matlab mediante la llamada a una cierta subrutina.
	%
	\item[II.] $\textbf{Solver\_K}$: Es el m�dulo que, dado un mallado, par�metros de la simulaci�n, y unos datos de captaci�n y del campo fluido, obtiene la soluci�n final del problema. Incluye una serie de subrutinas, como la principal $\textbf{prICE\_Solver}$, que avanza la simulaci�n en el tiempo, el subprograma \textbf{Sampling}, que para cada punto, realiza el proceso de estimaci�n de la soluci�n y su validaci�n, la subrutina $\textbf{System\_Solve}$, que calcula la soluci�n del sistema de EDPs en funci�n de la regi�n, y por �ltimo la subrutina \textbf{Compatibility}, que eval�a el cumplimiento de las condiciones termodin�micas de compatibilidad.
	%
	\item[III.] \textbf{Models}: Incluye toda la modelizaci�n termodin�mica del problema: modelos de los t�rminos fuente, de presi�n de vapor, del coeficiente de pel�cula...
	%
	\item[IV.] $\textbf{Graphics\_utilities}$: Contiene las subrutinas cuya funci�n es exportar un archivo de Matlab listo para ejecutar, que represente gr�ficamente la soluci�n del problema.
	%
	\item[V.] \textbf{Setup}: Su funci�n es correr el archivo $\textit{Settings \_IA.ini}$, cargando todos los par�metros de la simulaci�n as� como ciertas constantes termodin�micas.
	%
	\item[VI.] \textbf{Constants}: Incluye constantes de diversas naturalezas, a nivel de programaci�n b�sica (n�mero $\pi$, longitud de \textit{strings}...)
	%
	\end{itemize}
%\bibliographystyle{ThesisStyle}
%\bibliographystyle{siam} % We choose the "plain" reference style

%\bibliography{Bibliography/refs}
\printbibheading
\printbibliography
%\printbibliography[type=book,heading=subbibliography,title={Book Sources}]
%\printbibliography[type=paper,heading=papers,title={Paper Sources}]
%\printbibliography[type=paper,heading=papers,title={Paper Sources}]
\end{document}
