\chapter{Introduction.}
%
\indent In recent years, artificial neural networks have become ubiquitous across industries, fundamentally reshaping operations and decision-making processes. Their integration into various sectors has heralded a new era of innovation and efficiency. From optimizing flight trajectories\cite{xu2023machine} to enhancing predictive maintenance\cite{shukla2020opportunities,adhikari2018machine,korvesis2017machine}, artificial neural networks have emerged as indispensable tools, enabling organizations to unlock insights and drive transformative outcomes. As the aerospace sector embraces this technological revolution, the adoption of neural networks underscores a strategic imperative to harness the power of machine learning in pursuit of greater precision, reliability, and safety. Particularly in the aerospace industry, the integration of ML models presents both tremendous opportunities and formidable challenges (vid.). As the demand for advanced predictive analytics and automation escalates, so too does the necessity for rigorous statistical validation methodologies. The burgeoning reliance on ML algorithms for critical decision-making processes necessitates a paradigm shift towards comprehensive validation pipelines. Amidst this transformation, air-safety authorities have intensified their demands for stringent validation and verification processes\cite{force2021concepts,roadmap2021easa} to ensure the safety and reliability of ML-driven systems deployed within aerospace environments. Yet, industry leaders have only recently begun to confront the complexities of certifying ML models\cite{henderson2022toward,durand2023formal,dmitriev2021toward,el2022certification,paul2023assurance}, prompting the initiation of discussions around the development of guidelines and a roadmap for design assurance, especially concerning network-related technologies. This pressing need underscores the imperative for collaborative efforts within the industry to establish robust validation frameworks that not only meet regulatory standards but also address the evolving challenges posed by ML integration.\\
\begin{figure}[!htb]
	\centering
	\includegraphics[width=0.8\textwidth]{Figures/ml_applications.png}
	\caption{Research fields in which Machine Learning represents an outbreak in technological development. \cite{le2023improving}}
\end{figure}

\indent In the forthcoming chapters, this article delineates a comprehensive validation pipeline tailored to address the challenges of certifying neural network models, with a special focus in the aerospace industry. Chapter 1 serves as the foundational framework, focusing on the class of problems amenable to the proposed validation methodology. Principally, the validation pipeline targets regression problems, acknowledging their prevalence in industrial settings employing artificial neural networks (ANNs). While the proposed pipeline aspires to versatility, accommodating a broad spectrum of problems, this work primarily centers on regression problems encountered with ANNs in industry.\\
%
\indent Within \autoref{chap:1}, a formal exposition of these regression problems is provided, elucidating key concepts essential to subsequent chapters. Notably, the discussion encompasses vital notions within the context of surrogate models, including overfitting, applicability regions, and data preprocessing. These concepts serve as recurring themes throughout the ensuing chapters, underpinning the methodology and rationale behind the proposed validation pipeline.\\
%
\indent Moreover, to exemplify each facet of the validation pipeline, rigorous testing is conducted employing an industrial surrogate model. This surrogate model encapsulates real-world complexities inherent in a significant problem domain: structural calculation and failure mode prediction in aeronautical structures. The utilization of this surrogate model facilitates a practical illustration of the validation pipeline's efficacy, demonstrating its applicability to high-stakes aerospace applications and showcasing its potential to enhance safety and efficiency within the industry.\\
%
\indent Transitioning to \autoref{chap:2}, a concise introduction to the industrial problem domain and the surrogate model utilized for validation purposes is provided. This chapter serves to contextualize the subsequent discussions on validation methodologies within the specific industrial context, laying the groundwork for a detailed exploration of the validation pipeline's application.\\
%
\indent Chapter \ref{chap:3} constitutes the core of the article, delving into the validation pipeline in depth. Employing a box diagram to encapsulate its operational framework, this chapter offers a comprehensive overview of the validation process. Each component of the pipeline is scrutinized in detail, with dedicated sections elucidating their roles and contributions to the overarching validation methodology. By systematically dissecting the validation pipeline, this chapter elucidates its intricacies, offering insights into its design rationale and practical implementation within aerospace contexts.\\
%
\indent Finally, the concluding chapter provides a synthesis of the findings, offering insights into the efficacy of the proposed validation pipeline and outlining avenues for future research. Additionally, an ''outlook research'' section propounds potential directions for advancing validation methodologies in response to evolving industry demands and technological advancements. Through these concluding reflections, this article underscores the significance of collaborative efforts in establishing robust validation frameworks and driving innovation within the aerospace sector.
