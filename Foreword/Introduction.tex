\chapter{Introduction.}
%
\indent In recent years, artificial neural networks have become ubiquitous across industries, fundamentally reshaping operations and decision-making processes. Their integration into various sectors has heralded a new era of innovation and efficiency. From optimizing flight trajectories\cite{xu2023machine} to enhancing predictive maintenance\cite{shukla2020opportunities,adhikari2018machine,korvesis2017machine}, artificial neural networks have emerged as indispensable tools, enabling organizations to unlock insights and drive transformative outcomes. As the aerospace sector embraces this technological revolution, the adoption of neural networks underscores a strategic imperative to harness the power of machine learning in pursuit of greater precision, reliability, and safety. Particularly in the aerospace industry, the integration of ML models presents both tremendous opportunities and formidable challenges (vid.). As the demand for advanced predictive analytics and automation escalates, so too does the necessity for rigorous statistical validation methodologies. The burgeoning reliance on ML algorithms for critical decision-making processes necessitates a paradigm shift towards comprehensive validation pipelines. Amidst this transformation, air-safety authorities have intensified their demands for stringent validation and verification processes\cite{force2021concepts,roadmap2021easa} to ensure the safety and reliability of ML-driven systems deployed within aerospace environments. Yet, industry leaders have only recently begun to confront the complexities of certifying ML models\cite{henderson2022toward,durand2023formal,dmitriev2021toward,el2022certification,paul2023assurance}, prompting the initiation of discussions around the development of guidelines and a roadmap for design assurance, especially concerning network-related technologies. This pressing need underscores the imperative for collaborative efforts within the industry to establish robust validation frameworks that not only meet regulatory standards but also address the evolving challenges posed by ML integration.\\
\begin{figure}[!htb]
	\centering
	\includegraphics[width=0.8\textwidth]{Figures/ml_applications.png}
	\caption{Research fields in which Machine Learning represents an outbreak in technological development. \cite{le2023improving}}
\end{figure}
