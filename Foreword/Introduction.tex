\chapter{Introduction.}
%
\indent In recent years, artificial neural networks (or ANNs)\cite{Marsland2015Machine} have become ubiquitous across industries, fundamentally reshaping operations and decision-making processes. Their integration into various sectors has heralded a new era of innovation and efficiency. From optimizing flight trajectories\cite{xu2023machine} to enhancing predictive maintenance\cite{shukla2020opportunities,adhikari2018machine,korvesis2017machine}, artificial neural networks have emerged as indispensable tools, enabling organizations to unlock insights and drive transformative outcomes. As the aerospace sector embraces this technological revolution, the adoption of neural networks underscores a strategic imperative to harness the power of machine learning in pursuit of greater precision, reliability, and safety. Particularly in the aerospace industry, the integration of ML models presents both tremendous opportunities and formidable challenges. As the demand for advanced predictive analytics and automation escalates, so too does the necessity for rigorous statistical validation methodologies. The burgeoning reliance on ML algorithms for critical decision-making processes necessitates a paradigm shift towards comprehensive validation pipelines. Amidst this transformation, air-safety authorities have intensified their demands for stringent validation and verification processes\cite{force2021concepts,roadmap2021easa} to ensure the safety and reliability of ML-driven systems deployed within aerospace environments. Yet, industry leaders have only recently begun to confront the complexities of certifying ML models\cite{henderson2022toward,durand2023formal,dmitriev2021toward,el2022certification,paul2023assurance}, prompting the initiation of discussions around the development of guidelines and a roadmap for design assurance, especially concerning network-related technologies. This pressing need underscores the imperative for collaborative efforts within the industry to establish robust validation frameworks that not only meet regulatory standards but also address the evolving challenges posed by ML integration.\\
%
\clearpage
%
\indent The aerospace industry stands at a critical juncture as it navigates the integration of machine learning (ML) models into its operational framework. While ML offers unprecedented opportunities for enhancing efficiency and performance, its adoption raises significant concerns regarding safety and reliability, particularly in an industry where even the slightest error can have catastrophic consequences. In response to these concerns, air-safety authorities have intensified their demands for stringent validation and verification processes to ensure that ML algorithms meet the rigorous standards necessary for certification. However, the complexity of ML systems poses unique challenges for traditional certification approaches, prompting industry leaders to confront a pressing need for comprehensive guidelines and a roadmap. Until recently, the discourse around design assurance for network-related technologies in the aerospace sector has been limited. Recognizing the urgency of the situation, industry stakeholders have begun to collaborate in earnest to develop frameworks that address the intricacies of certifying ML-driven systems. This collective effort aims not only to comply with regulatory mandates but also to instill confidence among stakeholders by establishing robust validation protocols that account for the dynamic nature of ML algorithms and their interactions within aerospace environments. As such, the aerospace industry finds itself at the forefront of shaping the future of certification and safety standards for ML applications, with the imperative to strike a delicate balance between innovation and risk mitigation.\\
%
\indent Spacecraft relative motion is an essential part of distributed space systems analysis. It focuses on the dynamics of two or more vehicles, whose relative position (and generally attitude) should be controlled or at least monitored. Rendez-vous operations, formation flying and satellite constellations are three examples of relative motion, all of them being present in contemporary and past missions. Its relevance cannot be underestimated, and its complexity leads to the currently non-existent quorum on its modelling. However, it is quite widely accepted that linear models using State Transition Matrices (\ie STMs onwards) provides a comforting balance between accuracy and computational cost. One could think that, given the huge computational resources that most people have access to, computational efficiency should not even be a concern. Be that as it may, satellites on-board computers have limited resources, both computational and energy-related. Any reduction in any activity is greatly welcomed, and that naturally includes the GNC computation. That is the reason why this thesis focuses on the review, analysis and implementation of STM models of relative motion.\\
%
\indent A bottom-up approach to this topic is herewith followed, starting from the simplest relative motion model (HCW), building up complexity with elliptic models (YA) and finally reaching state-of-the-art perturbed models. With respect to perturbations, non-spherical gravity arises as the main concern, due to its prevailing effect over the other sources - third bodies and drag among others - for the Earth-centered orbits (excluding very low LEO orbits or very high orbits, like GEO).\\
%
\indent \textbf{Chapter \ref{chap:Chap_1}} provides a brief description of the roots of this thesis. Relative motion and its main branches are firstly described, to then continue with STMs and its applications.\\
%
\indent \textbf{Chapter \ref{chap:Chap_2}} starts with the development of the equations of relative motion, eventually reaching Hill equations, which model the linearised equations for circular unperturbed reference orbits. Several approaches to its solution are detailed, performing then a small trade-off analysis between them. Finally, orbit safety is presented, to be then applied to the circular, unperturbed case. \\
%
\indent \textbf{Chapter \ref{chap:Chap_3}} looks for extending the analysis to elliptic (still unperturbed) reference orbits. The state-of-the-art Yamanaka-Ankersen model is developed and tested against a High-Fidelity propagator. To end this chapter, the safe orbit theory is also extended to this elliptic case, through the usage of an analogue set of relative orbital elements and a slightly different relative reference frame. \\
%
\indent \textbf{Chapter \ref{chap:Chap_4}} tackles the analysis of propagation theory. A brief analysis of averaging methods is discussed, including Von Zeipel's and Lie series' approach. Afterwards, the spotlight is turned on the non-spherical gravity field modelling. The classical Brouwer's theory is first briefed, to then move on to probably the main topic of the thesis: Kaula's theory, which tries to develop a general approach to the spherical harmonic description of the gravity field.\\
%
\indent After this theoretical analysis, \textbf{chapter \ref{chap:Chap_5}} moves back to the relative motion field, considering perturbations in this case. An implementation of Brouwer's theory is drafted through Gim-Alfriend STM, after which a survey of the state of the art of perturbed relative motion is presented. To conclude, perturbed orbit safety is simply sketched out.\\
%
\indent The pure content of the thesis is then followed by a set of appendices to provide some background in certain relevant topics, if that is required or desired by the reader. \\
%
\indent \textbf{Appendix \ref{app:App_A}} provides a comprehensive review and description of the main orbital element sets. \\
%
\indent In a somewhat parallel fashion, \textbf{appendix \ref{app:App_B}} focuses on the reference systems used or mentioned throughout the thesis, as well as the transformations between them and from/to orbital elements.\\
%
\indent The variational formulation of the spacecraft equations of motion is introduced in \textbf{appendix \ref{app:App_C}}, providing context for some subjects of perturbation theory. \\
%
\indent \textbf{Appendix \ref{app:App_D}} is devoted to the software approach and structure of the developed content during the thesis.\\
%
\indent Finally, \textbf{appendix \ref{app:App_E}} contains several relevant topics that are not big enough to be included in an individual appendix, and do not fully fit into the existing ones.\\
%
\indent The goal of this thesis is then to provide an educated outlook on the linearised formulation of spacecraft relative motion, building up complexity as it unfolds, eventually reaching state-of-the-art models which are actually used in operational missions. Due to the structure of this field, this finds a niche in the manipulation and understanding of perturbations; as a result of which a considerable part of the thesis is devoted to it. The validation and trade-off evaluation of said models is naturally an essential part of this work as well. However, this thesis is not a result-based work: it is rather a knowledge, analysis-based project, which should be kept in mind along its reading.
%