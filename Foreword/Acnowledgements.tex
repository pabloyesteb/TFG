\chapter{Agradecimientos.}
%
\minitoc
%
\indent Personalmente, este trabajo lo he vivido como una de esas monta�as, que te impone al principio, que se escalan paso a paso sin mirar d�nde est� la cima, y que cuando est�s cerca, y parece que ya ha terminado, aparece la �ltima subida. Es un desaf�o que disfrutas simplemente dejando la mente en blanco y confiando paso a paso en que llegar�s, y que, en el momento de llegar arriba, su inmensidad te hace dudar de que lo hayas conseguido, y te hace apreciar ese momento en el que decidiste comenzarla, con m�s ganas y fe que conocimiento de lo que te esperaba. Y el orgullo que sientes en ese momento s�lo es comparable al agradecimiento que sientes por haber compartido el camino y haber sido guiado por ciertas personas.\\
%
\indent En primer lugar, quiero dar las gracias a Mariola por el infinito apoyo, consejos, motivaci�n y pasi�n que ha mostrado por m�, no s�lo durante este TFG, que es lo m�s apreciable, sino tambi�n desde las clases de Estad�stica y de Ampliaci�n de Matem�ticas. Me decid� por el trabajo gracias a ella, y no me he arrepentido en ning�n momento de la decisi�n. Ha sembrado en m� la semilla de la investigaci�n, y espero poder devolv�rselo de alguna manera alg�n d�a. Quiero agradecer tambi�n al Departamento de Matem�ticas (Mancebo, Ignacio Delgado) de la ETSIAE por los excelentes docentes que me han educado desde el primer curso al �ltimo, demostrando un gran apoyo y un gran inter�s por el alumnado. No me puedo olvidar de todos los profesores que me han educado, todos ellos de la escuela p�blica, desde los tres a�os hasta el bachillerato. Cada vez que reflexiono sobre ello, siento un profundo orgullo del camino por el que he llegado hasta aqu�. \\
%
\indent A nivel personal, quiero agradecer infinitamente el apoyo de mis padres, �ngeles y Gonzalo, desde peque�o, cuando quer�a ser dise�ador de coches, hasta que entr� en la escuela y, como si nada, acab� la carrera. Es obvio que sin la educaci�n y los valores que me han ense�ado, ni una miga de todo lo que conseguido habr�a sido posible. Por supuesto, quiero tambi�n agradecer al resto de mi familia, en particular a mi hermano, por todo el apoyo y los consejos, especialmente en mi vida universitaria. Tambi�n a mi abuela, aunque piense que ahora soy piloto. A Paula, por traerme esa pizca de calma, alegr�a y felicidad que hace que todo funcione a las mil maravillas. No me olvido de mis amigos, los de Bembibre y los de Madrid, que de m�s de uno y de dos aprietos me han sacado a rastras, y sin los cuales, a saber donde estar�a. \\
%
\begin{flushright}
Rodrigo Fern�ndez\\
10 de julio de 2020
\end{flushright}
