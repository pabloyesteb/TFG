\documentclass[a4paper,11pt,twoside,makeidx, svgnames]{ThesisStyle}
%% === === === === === === INITIAL PACKAGE LOAD  === === === === === ===
\usepackage[english]{babel}   
\usepackage[latin1]{inputenc}
\usepackage{graphicx}
\usepackage{epic}
\usepackage{enumitem}
\usepackage{xcolor}
\usepackage{curves}
\usepackage{multicol}
\usepackage{longtable}
\usepackage{amstext}
\usepackage{amsfonts}
\usepackage{amsbsy}
\usepackage{amsthm}
	\newtheorem{theorem}{Theorem}[section]

\usepackage{amsmath}
\usepackage{amssymb}
\usepackage{latexsym}  %\mathbb{R}
\usepackage[ruled,linesnumbered]{algorithm2e}
\usepackage{wrapfig}
\usepackage{tikz}
\usetikzlibrary{positioning, arrows.meta}
\usepackage{scalerel}
\usepackage{stackengine,wasysym}
\usepackage[overload]{empheq}
\usepackage{pgfplotstable,booktabs,siunitx,array}
\newcommand\reallywidetilde[1]{\ThisStyle{%
  \setbox0=\hbox{$\SavedStyle#1$}%
  \stackengine{-.1\LMpt}{$\SavedStyle#1$}{%
    \stretchto{\scaleto{\SavedStyle\mkern.2mu\AC}{.5150\wd0}}{.6\ht0}%
  }{O}{c}{F}{T}{S}%
}}
\usepackage{eurosym}
\usepackage{setspace}
\usepackage{makeidx}
\usepackage{textpos}
\usepackage{xfp}
\usepackage{pgfplotstable,booktabs,siunitx,array}
\usepackage{import}
\usepackage{rotating}    % Sideways of figures & tables
\usepackage{multirow}
\usepackage{mathtools}
\usepackage{ctable}
\usepackage{extarrows}
\usepackage{lscape}
\usepackage{pdflscape}
\usepackage{rotating}
\usepackage{cases}
\usepackage{tabularx}
\usepackage{soul}
\usepackage{tcolorbox} 
%% === === === === === === LISTINGS  === === === === === ===
\usepackage{listings}
\definecolor{codegreen}{rgb}{0,0.6,0}
\definecolor{codegray}{rgb}{0.5,0.5,0.5}
\definecolor{codepurple}{rgb}{0.58,0,0.82}
\definecolor{backcolour}{rgb}{0.95,0.95,0.92}
\lstdefinestyle{mystyle}{
    backgroundcolor=\color{backcolour},   
    commentstyle=\color{codegreen},
    keywordstyle=\color{magenta},
    numberstyle=\tiny\color{codegray},
    stringstyle=\color{codepurple},
    basicstyle=\ttfamily\footnotesize,
    breakatwhitespace=false,         
    breaklines=true,                 
    captionpos=b,                    
    keepspaces=true,                 
    numbers=left,                    
    numbersep=5pt,                  
    showspaces=false,                
    showstringspaces=false,
    showtabs=false,                  
    tabsize=2
}
\lstdefinestyle{pseudocode}{
	backgroundcolor=\color{gray!5},
	basicstyle=\ttfamily,
	frame=single,
	breaklines=true,
	numbers=left,
%	framexleftmargin=2em, % Ajusta el margen izquierdo
	xleftmargin=2em,
	escapeinside={(*@}{@*)}, % Delimitador para el modo matem�tico
}
\definecolor{mygreen}{RGB}{28,172,0} % color values Red, Green, Blue
\definecolor{mylilas}{RGB}{170,55,241}
\usepackage[numbered,framed]{matlab-prettifier}
\let\ph\mlplaceholder % shorter macro
\lstMakeShortInline"
\usepackage{bigfoot}
\usepackage{verbatim}
\lstset{
  style              = Matlab-editor,
  basicstyle         = \ttfamily,
  escapechar         = ",
  mlshowsectionrules = true,
}
\newcommand{\code}[1]{\texttt{#1}}
%% === === === === === === BIBLATEX  === === === === === ===
\usepackage[backend=biber,
        style=numeric-comp, hyperref, giveninits=true, backref = true, sorting=none]{biblatex}
\usepackage[hyperindex=true]{hyperref}
%\usepackage[pagebackref,hyperindex=true]{hyperref}
\defbibheading{subsubbibintoc}[\refname]{%
  		\subsection*{#1}%
      	\phantomsection
      	\addcontentsline{toc}{subsection}{#1}}
    		\defbibheading{subsubbibliography}[\refname]{%
      	\subsection*{#1}}
      
\defbibfilter{papers}{
  	type=article or
	type=inproceedings or
  	type=report
    }
%% === === === === === === FLOAT BARRIER  === === === === === ===
\usepackage[section]{placeins}
    
\usepackage{titlesec}
    
\titlespacing\section{0pt}{12pt plus 4pt minus 2pt}{0pt plus 2pt minus 2pt}
\titlespacing\subsection{0pt}{12pt plus 4pt minus 2pt}{0pt plus 2pt minus 2pt}
\titlespacing\subsubsection{0pt}{12pt plus 4pt minus 2pt}{0pt plus 2pt minus 2pt}

%% === === === === === === TIKZ SETUP  === === === === === ===
\definecolor{GMVred}{rgb}{0.8745, 0, 0.1412}
\definecolor{GMVgarnet}{rgb}{0.5254, 0, 0.0823}
\usepackage{tikz}
\usepackage{adjustbox}
%\usetikzlibrary{shapes.misc}
\usetikzlibrary{graphs}
\usetikzlibrary{arrows}
\usetikzlibrary{decorations.pathmorphing}
\usetikzlibrary{calc}
%\usetikzlibrary{shapes.arrows}
\usetikzlibrary{positioning,fit,backgrounds}
\usetikzlibrary{shapes}
%\usetikzlibrary{snakes}
% PATHS
\tikzset{hv path/.style={to path={-| (\tikztotarget)}}}
\tikzset{vh path/.style={to path={ |- (\tikztotarget)}}}
\tikzset{vhv path/.style={to path={ |- ++(0ex, -5ex) -| (\tikztotarget)}}}
\tikzset{hvh path/.style={to path={-|  ++(5ex, 0ex) |- (\tikztotarget)}}}
% NODES
\tikzset{nonterminal/.style = {
rectangle, minimum size=10mm, very thick, draw=black!50, top color=white,
bottom color=white, font=\itshape, text height=1.5ex,text depth=.25ex}}
%___________________________________________________________________________________________________________________
\tikzset{nonterminal_red/.style = {
rectangle, minimum size=10mm, very thick, draw=GMVred, top color=GMVred!20 ,
bottom color=GMVred!20, text height=1.5ex,text depth=.25ex}}
%___________________________________________________________________________________________________________________
\tikzset{bluerect/.style = {
rectangle, minimum size=10mm, very thick, draw=CadetBlue, top color=CadetBlue!20 ,
bottom color=CadetBlue!20, font=\itshape, text height=1.5ex,text depth=.25ex}}
%___________________________________________________________________________________________________________________
\tikzset{terminal/.style={
rounded rectangle, minimum size=10mm, very thick, draw=black!50, top color=white,
bottom color=black!20, font=\ttfamily, text height=1.5ex, text depth=.25ex}}
%___________________________________________________________________________________________________________________
\tikzset{terminal_gray/.style={
rounded rectangle, minimum size=10mm, very thick, draw=black!50, top color=black!10,
bottom color=black!10, font=\ttfamily, text height=1.5ex, text depth=.25ex}}
%___________________________________________________________________________________________________________________
\tikzset{terminal_blue/.style={
rounded rectangle, minimum size=10mm, very thick, draw=CadetBlue!50, top color=white,
bottom color=CadetBlue!50, font=\ttfamily, text height=1.5ex, text depth=.25ex}}
%___________________________________________________________________________________________________________________
\tikzset{terminal_white/.style={
rounded rectangle, minimum size=10mm, very thick, draw=black, top color=white,
bottom color=white, font=\ttfamily, text height=1.5ex, text depth=.25ex}}
%___________________________________________________________________________________________________________________
\tikzset{terminal_red/.style={
rounded rectangle, minimum size=10mm, very thick, draw=GMVred!50, top color=GMVred!5,
bottom color=GMVred!20, font=\ttfamily, text height=1.5ex, text depth=.25ex}}
%___________________________________________________________________________________________________________________
\tikzset{terminal_garnet/.style={
rounded rectangle, minimum size=10mm, very thick, draw=GMVgarnet!50, top color=GMVgarnet!5,
bottom color=GMVgarnet!20, font=\ttfamily, text height=1.5ex, text depth=.25ex}}
%___________________________________________________________________________________________________________________
\tikzset{empty/.style={
rounded rectangle, minimum size=10mm, very thick, draw=white, top color=white,
bottom color=white, font=\ttfamily, text height=1.5ex, text depth=.25ex}}
%___________________________________________________________________________________________________________________
\tikzset{sum_red/.style={
circle, thick, draw=GMVred, fill=GMVred!20, minimum size=8mm, font=\ttfamily, text height=1.5ex, text depth=.25ex}}
\tikzset{sum/.style={
circle, thick, draw=Gray, fill=Gray!10, minimum size=8mm, font=\ttfamily, text height=1.5ex, text depth=.25ex}}
%___________________________________________________________________________________________________________________
\tikzstyle{decision} = [diamond, draw, fill=white,
    text width=4.5em, text centered, node distance=3cm, inner sep=0pt]
%___________________________________________________________________________________________________________________
\tikzstyle{block} = [rectangle, draw, fill=gray!20, text width=7.5em, rounded corners, minimum height=4.1em]
%___________________________________________________________________________________________________________________
\tikzstyle{blocks} = [rectangle, fill=white!20, text width=9em, rounded corners, minimum height=4em]
%___________________________________________________________________________________________________________________
\tikzstyle{line} = [draw, -latex']
%___________________________________________________________________________________________________________________
\tikzstyle{cloud} = [ellipse,fill=white!20, node distance=2cm,
    minimum height=2em]
%___________________________________________________________________________________________________________________    
\newcommand{\lbsum}[1]{
	\node(#1)[sum]{};
 	\draw [draw=Gray]	(#1.north east) -- (#1.south west)
    		(#1.north west) -- (#1.south east);
	\draw [draw=Gray]	(#1.north east) -- (#1.south west)
			(#1.north west) -- (#1.south east);
 
	\node[left=-1pt] at (#1.center){\textcolor{Gray}{\scriptsize $\bm +$}};
	\node[below]	 at (#1.center){\textcolor{Gray}{\scriptsize $\bm + $}};
	}
\newcommand{\pptlsum}[1]{
	\node(#1)[sum]{};
 	\draw 	[draw=Gray](#1.north east) -- (#1.south west)
    		(#1.north west) -- (#1.south east);
	\draw 	[draw=Gray](#1.north east) -- (#1.south west)
			(#1.north west) -- (#1.south east);
 
	\node[above] at (#1.center){\textcolor{Gray}{\scriptsize $\bm + $}};
	\node[left=-1pt] at (#1.center){\textcolor{Gray}{\scriptsize \textbf{+}}};}
	%
\newcommand{\mptbsum}[1]{
	\node(#1)[sum]{};
 	\draw 	[draw=Gray](#1.north east) -- (#1.south west)
    		(#1.north west) -- (#1.south east);
	\draw 	[draw=Gray](#1.north east) -- (#1.south west)
			(#1.north west) -- (#1.south east);
 
	\node[above] at (#1.center){\textcolor{Gray}{\scriptsize $\bm - $}};
	\node[below] at (#1.center){\textcolor{Gray}{\scriptsize $\bm + $}};}
	%
	\newcommand{\pmtbsum}[1]{
	\node(#1)[sum]{};
 	\draw 	[draw=Gray](#1.north east) -- (#1.south west)
    		(#1.north west) -- (#1.south east);
	\draw 	[draw=Gray](#1.north east) -- (#1.south west)
			(#1.north west) -- (#1.south east);
 
	\node[above] at (#1.center){\textcolor{Gray}{\scriptsize $\bm + $}};
	\node[below] at (#1.center){\textcolor{Gray}{\scriptsize $\bm -$}};}
\newcommand{\ppptlbsum}[1]{
	\node(#1)[sum]{};
 	\draw 	[draw=Gray](#1.north east) -- (#1.south west)
    		(#1.north west) -- (#1.south east);
	\draw 	[draw=Gray](#1.north east) -- (#1.south west)
			(#1.north west) -- (#1.south east);
 
	\node[above] at (#1.center){\textcolor{Gray}{\scriptsize $\bm + $}};
	\node[left=-1pt] at (#1.center){\textcolor{Gray}{\scriptsize \textbf{+}}};
	\node[below] at (#1.center){\textcolor{Gray}{\scriptsize $\bm + $}};}
\newcommand{\pmlbsum}[1]{
	\node(#1)[sum]{};
 	\draw 	[draw=Gray](#1.north east) -- (#1.south west)
    		(#1.north west) -- (#1.south east);
	\draw 	[draw=Gray](#1.north east) -- (#1.south west)
			(#1.north west) -- (#1.south east);
 
	\node[left=-1pt] at (#1.center){\textcolor{Gray}{\scriptsize $\bm +$}};
	\node[below] at (#1.center){\textcolor{Gray}{\scriptsize $\bm - $}};}
%\tikzset{point/.style={circle,inner sep=0pt,minimum size=2pt,fill=red}}
\tikzset{point/.style={circle,inner sep=0pt,minimum size=0pt, fill=black!50}}
\tikzset{skip loop/.style={to path={-- ++(0,#1) -| (\tikztotarget)}}}


\newcommand{\snake}{decorate,
decoration={snake,amplitude=.8mm,segment length=2mm,post length=1mm}}

%% === === === === === === NOMENCLATURE SETUP  === === === === === ===
\usepackage[intoc, english]{nomencl}
\newcommand{\nomunit}[1]{%
\renewcommand{\nomentryend}{\hspace*{\fill}#1}}
\makenomenclature
\usepackage{ifthen}
\usepackage{etoolbox}
%\renewcommand{\nomgroup}[1]{%
%\item[\bfseries
%    \ifthenelse{\equal{#1}{A}}{Acronyms}{%
%    \ifthenelse{\equal{#1}{P}}{Perturbations}{%
%    \ifthenelse{\equal{#1}{S}}{Subindices}{%
%    \ifthenelse{\equal{#1}{O}}{Orbital elements}{%
%    \ifthenelse{\equal{#1}{U}}{Useful expressions}{%
%    \ifthenelse{\equal{#1}{G}}{General}{%
%    \ifthenelse{\equal{#1}{M}}{Mathematics}{}}}}}}%
%    ]}
\renewcommand\nomgroup[1]{%
\item[\bfseries
    \ifstrequal{#1}{A}{Acronyms}{%
    \ifstrequal{#1}{P}{Perturbations}{%
    \ifstrequal{#1}{S}{Subindices}{%
    \ifstrequal{#1}{O}{Orbital elements}{%
    \ifstrequal{#1}{U}{Useful expressions}{%
    \ifstrequal{#1}{G}{General}{%
    \ifstrequal{#1}{M}{Mathematics}{}}}}}}}%
    ]}
\renewcommand{\nompreamble}{\scriptsize \begin{multicols}{2}}
\renewcommand{\nompostamble}{\end{multicols}}    
    
\setcounter{secnumdepth}{3}

%% === === === === === === SUBCAPTION  === === === === === ===
\usepackage{subcaption}
\captionsetup[subfigure]{subrefformat=simple,labelformat=simple}
\renewcommand\thesubfigure{(\alph{subfigure})}
    
\captionsetup[table]{name=Table}
    
% float para las figuras
\def\floatpagefraction{.8}

\DeclareGraphicsExtensions{.pdf,.png,.jpg} %solo para PDFLaTeX

%\setlength{\unitlength}{0.18mm}

\usepackage[super]{nth}
\newcolumntype{a}{>{\columncolor{Gray!30}}c}
%% === === === === === === NEW MATH COMMANDS  === === === === === ===
\newcommand{\abs}[1]{\mbox{$\left\lvert #1 \right\rvert$}}
\newcommand{\sign}{\text{sign}}
\newcommand{\degree}{\mbox{$^{\circ}$}}
\newcommand{\eye}{\mathbb{I}}
\newcommand{\atantwo}{\text{atan2}}
\newcommand{\norm}[1]{\mbox{$\left\lVert #1 \right\rVert$}}
%\newcommand{\const}{\mbox{$\text{const.}$}}
\newcommand{\const}{\text{const.}}
%\newcommand{\wrt}{w.r.t.}
\newcommand{\wrt}{\vert \vert}
\newcommand{\bm}[1]{\mbox{\boldmath{${#1}$}}}
%\newcommand{\dfrac}{\displaystyle\frac}
\newcommand{\dsqrt}{\displaystyle\sqrt}
\newcommand{\dint}{\displaystyle\int}
\newcommand{\dsum}{\displaystyle\sum}
\newcommand{\tsum}{\textstyle\sum}
\newcommand{\dprod}{\displaystyle\prod}
\newcommand{\dlim}{\displaystyle\lim}
\newcommand{\eg}{{\it e.g.}\ }
\newcommand{\ie}{{\it i.e.}\ }
\newcommand{\ea}{\mbox{\it et al.}\ }
\renewcommand{\labelitemi}{$\bullet$}
\renewcommand{\labelitemii}{--}
\renewcommand{\labelitemiii}{$\spadesuit$}
\newcommand{\GMVred}[1]{\textcolor{GMVred}{#1}}
\newcommand{\longeq}{\scalebox{3}[1]{=}}
\newcommand{\longapprox}{\scalebox{2}[1]{$\approx$}}
\newcommand{\myul}[2][black]{\setulcolor{#1}\ul{#2}\setulcolor{black}}
    
\newcommand{\be}{\begin{equation}}
% $$ 
\newcommand{\ee}{\end{equation}}

\newcommand{\caja}[2]{
\begin{center}
\begin{tabular}{|p{#1}|}
\hline
\vspace{#2}\\
\hline
\end{tabular}
\end{center}
}

%% === === === === === === COLOR DEFINITION  === === === === === ===
\definecolor{Blue}          {cmyk}{1,1,0,0}            % palabras clave
\definecolor{BrickRed}      {cmyk}{0,1.0,0.91,0.60}    % texto resaltado
\definecolor{PineGreen}     {cmyk}{0.92,0,0.59,0.25}   % comentarios programa
\definecolor{CornflowerBlue}{cmyk}{0.65,0.13,0,0}
\definecolor{SkyBlue}       {cmyk}{0.62,0,0.12,0}
\definecolor{Melon}         {cmyk}{0,0.46,0.50,0}
\definecolor{Apricot}       {cmyk}{0,0.32,0.52,0}
\definecolor{GreenYellow}   {cmyk}{0.15,0,0.69,0}

\definecolor{CadetBlue}     {cmyk}{0.62,0.57,0.23,0}
\definecolor{CornflowerBlue}{cmyk}{0.65,0.13,0,0}
\definecolor{MidnightBlue}  {cmyk}{0.98,0.13,0,0.43}
\definecolor{NavyBlue}      {cmyk}{0.94,0.54,0,0}
\definecolor{RoyalBlue}     {cmyk}{1,0.50,0,0}

\definecolor{verde}{cmyk}{0.40,0.0,0.32,0.70}
\definecolor{azul}{cmyk}{1.0,0.78,0.0,0.18}
\definecolor{linkcol}{rgb}{0,0,0.4}
\definecolor{citecol}{rgb}{0.5,0,0}
\definecolor{Skyblue}{cmyk}{0.4,0.1,0.0,0.0}
\newcommand{\BRnem}[1]{\normalsize\color{BrickRed}\em #1}
\newcommand{\BRlbf}[1]{\Large\color{BrickRed}\rmfamily\bfseries #1}
\newtheorem{prop}{Property}
\newtheorem{definition}{Definition}
\newenvironment{Figure}
  {\par\medskip\noindent\minipage{\linewidth}}
  {\endminipage\par\medskip}
  
\newcommand\ccg[1]{\cellcolor{gray!20}{#1}} % for cells in second column 
                                                   % and gray colored rows  
\newcommand\ccgr[1]{\cellcolor{gray!30!red!30}{#1}} % for cells in second column 
                                                   % and gray colored rows
\newcommand\ccr[1]{\cellcolor{red!15}{#1}}         % for cells in  second column 
                                                   % white colored rows
%% === === === === === COLORBOX  === === === === ===     
\newtcolorbox{GMVbox}{colback=red!5!white,colframe=red!75!black}              
%------------------------------------------------------------------------------
\def\floatpagefraction{.8}
\DeclareGraphicsExtensions{.pdf,.png,.jpg} % solo para PDFLaTeX
\def\baselinestretch{1.5}
%------------------------------------------------------------------------------

%% === === === === === === A4 SIZE  === === === === === ===

\def\psnormal{\textwidth=16cm\textheight=23cm
          \oddsidemargin=0cm\evensidemargin=0.2cm
          \parindent=1cm\topmargin=-1cm}
\psnormal

%% === === === === === === FANCY PAGE === === === === === ===

\usepackage{fancyhdr}                    % Fancy Header and Footer

\pagestyle{fancy}                       % Sets fancy header and footer
\fancyfoot{}                            % Delete current footer settings

\fancyhead[LE,RO]{\bfseries\thepage}    % Page number (boldface) in left on even
% pages and right on odd pages
\fancyhead[RE]{\bfseries\nouppercase{\leftmark}}      % Chapter in the right on even pages
\fancyhead[LO]{\bfseries\nouppercase{\rightmark}}     % Section in the left on odd pages

\let\headruleORIG\headrule
\renewcommand{\headrule}{\color{black} \headruleORIG}
\renewcommand{\headrulewidth}{1.0pt}
\usepackage{colortbl}
\arrayrulecolor{black}

\fancypagestyle{plain}{
  \fancyhead{}
  \fancyfoot{}
  \renewcommand{\headrulewidth}{0pt}
}
\newenvironment{changemargin}[2]{%
\begin{list}{}{%
\setlength{\topsep}{0pt}%
\setlength{\leftmargin}{#1}%
\setlength{\rightmargin}{#2}%
\setlength{\listparindent}{\parindent}%
\setlength{\itemindent}{\parindent}%
\setlength{\parsep}{\parskip}%
}%
\item[]}{\end{list}}
\parindent30pt
%% === === === === === === MINITOC  === === === === === ===
\usepackage{minitoc}
%\usepackage[tight,spanish]{minitoc}
%\usepackage[tight]{minitoc}

\setcounter{minitocdepth}{2}         	%default
\setlength{\mtcindent}{24pt}        	%default
\renewcommand{\mtcfont}{\small\rm}   	%default
\renewcommand{\mtcSfont}{\small\bf}  	%default
\mtcsetoffset{minitoc}{-2.0em}			% indent del minitoc
\usepackage{listings}

%--------------------------------------------------------------------------

\lstdefinestyle{interfaceNF}
                {basicstyle=\ttfamily\small,
                identifierstyle=\ttfamily,
                emphstyle=\ttfamily
                }

\lstdefinestyle{ejemplos}
                {basicstyle=\ttfamily,
                keywordstyle=\em,
                commentstyle=\ttfamily,
                morecomment=[l][\em]{Dec},
                morecomment=[l][\em]{sent},
                keywords={numero,estructura,componente}
                }

%% === === === === === === SCALING SETUP   === === === === === ===
% Measure the width of a reference image
\newlength{\imagewidth}
\settowidth{\imagewidth}{\includegraphics{/home/pablo/Documentos/TFGP/Thesis/Figures/tt-split/geo.png}}

% The former image is considered to be in the document adjusted to the textwidth.
\edef\tabscale{\fpeval{\textwidth/\imagewidth}}

\newcommand{\mytable}[2]{%
	\newlength{\orgWidth}%
	\settowidth{\orgWidth}{\includegraphics{#1}}%
	\edef\scaledWidth{\fpeval{\orgWidth*\tabscale}}%
	%
	\begin{minipage}{\dimexpr\scaledWidth pt\relax}%
		\centering%
		\caption{#2}%
		\includegraphics[scale=\tabscale]{#1}%
	\end{minipage}%
}
%% === === === === === === HYPERREF SETUP  === === === === === ===

\usepackage{hyperref}
\hypersetup
{
bookmarksopen=true,
pdftitle=Validation methods for industrial ML models,
pdfauthor="P. Yeste Blesa",
pdfsubject="ML validation", 	% subject of the document
%pdftoolbar=false, 				% toolbar hidden
pdfmenubar=true, 				% menubar shown
pdfhighlight=/O,           		% effect of clicking on a link
colorlinks=true, 				% couleurs sur les liens hypertextes
pdfpagemode=None, 				% aucun mode de page
pdfpagelayout=SinglePage, 		% ouverture en simple page
pdffitwindow=true,              % pages ouvertes entierement dans toute la fenetre
linkcolor=linkcol, 				% couleur des liens hypertextes internes
citecolor=citecol, 				% couleur des liens pour les citations
urlcolor=linkcol 				% couleur des liens pour les url
}
\usepackage{cleveref}