\chapter{Absolute and relative orbital element sets.}
%
\label{chap: App_OEs}
%
\section{Introduction.}
%
\indent The description of a spacecraft's state is done via a \textbf{state vector}. While it can include several variables with other purposes (\eg filtering), its only information throughout this thesis is the position and velocity. There are two main ways to describe them:\\
%
\begin{itemize}
\item[A.] Through cartesian coordinates
\item[B.] Through orbital elements
\end{itemize}
%
\indent While the first option yields a very explicit and graphic-ready description, the second one usually has two advantages over it. Firstly, orbital elements are generally more intuitive about both the orbit and the position on it. Secondly, as orbital elements are generally slow-varying, they allow for a bigger integration timestep without losing accuracy. This is quite clear when studying keplerian motion, as most of the elements remain constant. Variational formulation and Hamilton-Jacobi theory (with the notion of changing variables as the full solution of a problem) relate to this fact. \\
%
\indent Throughout this thesis, several sets of orbital elements have been used. The goal of this appendix is to clarify on the definition and differences between them. Absolute orbital elements (OEs) will be described first, followed by relative OEs (ROEs).
%
\section{Absolute sets.}
%
	\subsection{Keplerian orbital elements (KOE).}
	%
	\indent The Keplerian set of OEs (KOE) is one of the most widely used and classic options. While the last element may change from author to author, an usual definition is the following:
	%
	\begin{equation}
	\left\{ 
	\begin{array}{lll}
	a & \equiv & \text{Semimajor axis}\\
	e & \equiv & \text{Eccentricity}\\
	i & \equiv & \text{Inclination}\\
	\Omega \text{or} RAAN & \equiv & \text{Right ascension of the ascending node}\\
	\omega & \equiv & \text{Argument of perigee}\\
	M & \equiv & \text{Mean anomaly}\\
	\end{array}
	\right\}
	\end{equation}
	%
	\subsection{Quasi-nonsingular orbital elements (QNSOE).}
	%
	%
	\subsection{Equinoctial orbital elements (EOE).}
	%
	%
	\subsection{Delaunay orbital elements (DOE).}
	%
	%
%
\section{Relative sets.}
%
% General workflow and explicit definition
%
	\subsection{Eccentricity/inclination vectors relative orbital elements (EIROE).}
	%
	%
	\subsection{Peters-Noomen C set of relative orbital elements (CROE).}
	%
	%
	\subsection{General workflow for arbitrary ROEs.}
	%
	%
%