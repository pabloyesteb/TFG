\chapter{Cartesian reference systems.}
%
\label{chap: App_cartesian}
%
\section{Introduction.}
%
\indent Cartesian states are, as mentioned in appendix \ref{chap: App_OEs}, one of the two main alternatives to describe the state of a certain spacecraft (or celestial body). Though orbital elements (OEs) are generally more intuitive and meaningful, these states are quite critical for the description of both absolute and relative motion. Ultimately, and specially considering the latter, we wish to know the relative orientation and linear distance between the involved bodies. During this appendix, a set of absolute and relative reference frames will be described and related via transformations, which have been used time and again along this thesis.
%
	\subsection{Inertial and rotating reference frames.}
	%
	\indent Technically, an inertial reference frame is one where Newton's law holds. Effectively, it is a frame which is not object of any acceleration whatsoever. It is then, when interpreted to the letter, an idealization, as there will always be any perturbation which disavows this assumption. Nonetheless, it is usual to neglect said perturbations up to a certain point, thus considering pseudo-inertial reference frames. From now on then, when inertial reference frames are mentioned they will be considered so, even though they are actually not. Along this thesis, both inertial and rotating frames will be considered, each bearing its different advantages and disadvantages.
	%
	\subsection{Absolute and relative frames.}
	%
	\indent Another distinction that will be made is between absolute and relative frames. In this thesis, absolute frames are those who are centered in the Earth's center of mass, while relative frames are defined with respect to a reference orbit (the chief's generally). Again, they have different scopes, though relations between them need to be developed.
	%
	\subsection{Time measurement.}
	%
	\indent Later it will be described how Earth's rotational state influences the dynamics of the spacecrafts, due to its non-homogeneous mass distribution. That leads to the need of precisely computing it, which in turn requires the time elapsed since a given epoch. This section intends to briefly describe the most usual conventions for time definition, without diving in technical considerations. For further description, see \cite{Time_handbook}. These conventions are:%
	%	
	\begin{itemize}
	% 
	\item[I.] International Atomic Time (TAI): Physical timescale which is calculated through the measurement of cesium radiation. Lacks intuitive meaning, but acts as a ultra-high precision time system and reference for other timescales.
	%
	\item[II.] Universal Time (UT1/UT2): Civil time system, which is defined by the right ascension of the mean Sun. It is not a continuous time system, varying as time passes. There are some smoothed versions of UT1, which filters some seasonal variations, thus accounting only for long-term changes. 
	%
	\item[III.] Coordinated Universal Time (UTC): Civil time system, which is measured with TAI and synchronized with UT1 via leap seconds. This means that the time rate is the one from the atomic counterpart, whereas the `reference' is set to follow UT1 within 0.9 seconds. If the difference between UTC and UT1 exceeds this value, 1 second is added/substracted to get within range. It is then a non-continuous time system.
	%
	\item[IV.] Terrestrial Time (TT): Civil time system, which is measured with TAI but has a delay with respect to it (32.184 seconds).
	\end{itemize}
	%
	\indent Figure \textbf{PUT FIGURE} shows more clearly the differences between them.
% 
\section{Absolute reference systems.}
%
	\subsection{Earth-Centered-Inertial reference system (ECI).}
	%
	\indent As previously stated, any Earth-centered reference system will in turn be non-inertial. That leads to the need of defining a common baseline, \ie an epoch at which the reference system is known. The chosen epoch is denoted as J2000.0\footnote{J2000 denotes a reference frame, being analog to ECI. J2000.0 refers to the mentioned epoch.}, which translates to January \nth{1}, at 12:00:00.000 (midday) in Julian years \cite[see][, glossary]{Time_handbook}. Effectively, the ECI reference system, is geometrically defined as follows \cite{Tapley}:
	%
	\[
	\arraycolsep=10pt
	\def\arraystretch{1.3}
	ECI \equiv\left\{
	\begin{array}{lll}
	\text{Origin} 	& \equiv 	& \text{Earth's COM} \\[0.5 em]
	\text{X-axis} 	& \equiv 	& \text{Earth's COM} \longrightarrow \text{Mean vernal equinox at epoch J2000.0} \\
	\text{Z-axis} 	& \equiv 	& \text{Normal to the mean equatorial plane at epoch J2000.0, } \\
	  				& 			& \text{pointing towards the Northern Hemisphere} \\
	\text{Y-axis} 	& \equiv 	& \text{Perpendicular to the X and Z axes forming a right-handed system} \\
	\end{array}
	\right.
	\]
	%
	\indent The main reason behind using this system is that it considerably simplifies the dynamics equation of any spacecraft. It is then the most adequate frame on which dynamics can be solved. Furthermore, when considering relative motion, the reference axis are not a critical axis, as we are rather focused on the motion between spacecrafts. On the other hand, this frame is not able to describe the position relative to Earth's surface, thus being useless in communications or visibility analysis.
	%
	\subsection{Earth-Centered, Earth-Fixed reference system (ECEF).}
	%
	\indent Due to the formerly mentioned concerns, another Earth-centered reference frame must be defined. In this case, that will be ECEF. Geometrically, it is defined as \cite{Tapley}
	%
	\[
	\arraycolsep=10pt
	\def\arraystretch{1.3}
	ECEF \equiv\left\{
	\begin{array}{lll}
	\text{Origin} 	& \equiv 	& \text{Earth's COM} \\[0.5 em]
	\text{X-axis} 	& \equiv 	& \text{Earth's COM} \longrightarrow \text{Intersection of prime meridian and true equatorial plane} \\
	\text{Z-axis} 	& \equiv 	& \text{Earth's true angular velocity vector (rotation axis)} \\
	\text{Y-axis} 	& \equiv 	& \text{Perpendicular to the X and Z axes forming a right-handed system} \\
	\end{array}
	\right.
	\]
	%
	\indent Once defined, it is turn to evaluate how ECI and ECEF frames differ.
	%
		\subsubsection{Conversion from ECI to ECEF.}
		%
		%
		\paragraph{Decomposition of the conversion. \\}
		%
		\indent There are four essential differences between ECI and ECEF frame, due to four motions that ECEF include due to it being fixed to Earth:
		%
		\begin{itemize}
		\item[1.] Precession of the equinoxes.
		%
		\item[2.] Nutations (small oscillations) of the equinoxes.
		%
		\item[3.] Earth's rotation around its axis.
		%
		\item[4.] Spin axis motion.
		\end{itemize}
		%
		\indent Each of this motions can be characterized by a rotation to an associated frame. That is, we can decompose the conversion between ECI and ECEF in four rotations, which will now be analyzed.
		%
		\paragraph{Involved intermediate frames \& rotations. \\}
		%
		\subparagraph{ECI(J200) to Mean of Date. \\}
		%
		\indent The equinoxes rotate at a slow, but relevant rate. That means that the vernal equinox today differs considerably from the one at J2000.0. The Mean of Date (MOD) frame arises from this notion, being defined as \cite{Tapley}: \\
		%
		\[
		\arraycolsep=10pt
		\def\arraystretch{1.3}
		MOD\equiv\left\{
		\begin{array}{lll}
		\text{X-axis} 	& \equiv 	& \text{Earth's COM} \longrightarrow \text{Mean vernal equinox at current epoch} \\
		\text{Z-axis} 	& \equiv 	& \text{Perpendicular to the mean equatorial plane at current epoch} \\
		\text{Y-axis} 	& \equiv 	& \text{Perpendicular to the X and Z axes forming a right-handed system} \\
		\end{array}
		\right.
		\]
		%
		\indent The rotation matrix from J200 to MOD results:
		%
		\begin{equation}
		R_{ECI\rightarrow MOD} = 
		\left[ 
		\begin{array}{lll}
		C \zeta_A C \theta_A C z_A - S \zeta_A S z_A 	& - S \zeta_A C \theta_A C z_A - C \zeta_A S z_A 	& -S \theta_A C z_A \\
		C \zeta_A C \theta_A S z_A + S \zeta_A C z_A 	& - S \zeta_A C \theta_A S z_A + C \zeta_A C z_A 	& -S \theta_A S z_A \\
 		C \zeta_A S \theta_A 							& - S \zeta_A S \theta_A 							& C \theta_A
		\end{array}
		\right]
		\label{eq: R_ECI_MOD}
		\end{equation}
		%
		\noindent where the precession angles $\zeta_A$, $\theta_A$ and $z_A$ are evaluated by:
		%
		\[
		\left\{ 
		\begin{array}{lll}
		\zeta_A 	& = & 2306.2181'' t + 0.30188'' t^2 + 0.017998'' t^3 \\
		\theta_A 	& =	& 2004.3109'' t - 0.42665'' t^2 - 0.041833'' t^3 \\
		z_A 		& =	& 2306.2181'' t + 1.09468'' t^2 + 0.018203'' t^3 \\ 
		\end{array}
		\right.
		\]
		\noindent and the time $t$ is defined as:
		%
		\[
		t 	= \dfrac{\left(TT - J2000.0\right)}{36525}		
		\]
		%
		\noindent where TT is the Terrestrial Time (dd/mm/yyyy, hh:mm). The transformation matrix is then perfectly defined for a certain epoch
		%
		\subparagraph{Mean of Date to True of Date \\}
		%
		\indent Besides the ``long-term'' precession motion, equinoxes suffer also short-period, small oscillations, which are denoted as nutations. For more clarity, see figure \textbf{PUT FIGURE NUTATION PRECESSION}. The True of Date (TOD) frame is thus defined as: \\
		%
		\[
		\arraycolsep=10pt
		\def\arraystretch{1.3}
		TOD\equiv\left\{
		\begin{array}{lll}
		\text{X-axis} 	& \equiv 	& \text{Earth's COM} \longrightarrow \text{True vernal equinox at current epoch} \\
		\text{Z-axis} 	& \equiv 	& \text{Perpendicular to the true equatorial plane at current epoch} \\
		\text{Y-axis} 	& \equiv 	& \text{Perpendicular to the X and Z axes forming a right-handed system} \\
		\end{array}
		\right.
		\]
		%
		\indent The rotation matrix now has the following shape:
		%
		\begin{equation}
		R_{MOD\rightarrow TOD} = 
		\left[ 
		\begin{array}{lll}
		C \Delta \psi 				& - C \epsilon_m S \Delta \psi 	& - S \epsilon_m S \Delta \psi \\
		C \epsilon_t S \Delta \psi 	& C \epsilon_m C \epsilon_t C \Delta \psi + S \epsilon_m S \epsilon_t & S \epsilon_m C \epsilon_t C \Delta \psi - C \epsilon_m S \epsilon_t \\
		S \epsilon_t S \Delta \psi 	& C \epsilon_m S \epsilon_t C \Delta \psi - S \epsilon_m C \epsilon_t & S \epsilon_m S \epsilon_t C \Delta \psi + C \epsilon_m C \epsilon_t \\
		\end{array}
		\right]
		\label{eq: R_MOD_TOD}
		\end{equation}
		%
		\noindent where four angles appear. Firstly, the mean obliquity, which is the angle between the mean ecliptic and the mean equatorial plane ($\approx 23.5^{\circ}$), whose value is computed by:
		%
		\[
		\epsilon_m = 84381.448'' - 46.8150''t - 0.00059'' t^2 + 0.001813'' t^3		
		\]
		%
		\indent Moreover, two nutations appear: the one in longitude and the one in obliquity ($\Delta \psi$ and $\Delta \epsilon$, respectively). This angles are computed by a summation of a large number of sinusoidal functions, whose construction and coefficients are shown in \cite{IERS_conventions}. Lastly, the true obliquity is simply the addition of its mean counterpart and the nutation ($\epsilon_t = \epsilon_m + \Delta \epsilon$).
		%
		\subparagraph{True of Date to Pseudo-Body-Fixed\\}
		%
		\indent Perhaps the biggest and most intuitive difference between ECI and ECEF is Earth's rotation around its axis. The pseudo-body-fixed is simply a clockwise rotation (seen from north pole towards the Earth's COM) around said axis from the True of Date frame:  \\
		%
		\[
		\arraycolsep=10pt
		\def\arraystretch{1.3}
		PBF\equiv\left\{
		\begin{array}{lll}
		\text{X-axis} 	& \equiv 	& \text{Earth's COM} \longrightarrow \text{Intersection between prime meridian and true equatorial plane.} \\
		 				&  	 		& \text{(without accounting for the axis' displacement).} \\
		\text{Z-axis} 	& \equiv 	& \text{Perpendicular to the true equatorial plane at current epoch} \\
		\text{Y-axis} 	& \equiv 	& \text{Perpendicular to the X and Z axes forming a right-handed system} \\
		\end{array}
		\right.
		\]
		%
		\indent The rotation matrix is now as simple as:
		%
		\begin{equation}
		R_{TOD\rightarrow PBF} = 
		\left[ 
		\begin{array}{lll}
		C \alpha_G 		& S\alpha_G 	& 0 \\
		-S\alpha_G 		& C\alpha_G 	& 0 \\
		0 				& 0 			& 1 \\
		\end{array}
		\right]
		\label{eq: R_TOD_PBF}
		\end{equation}
		%
		\noindent where $\alpha_G$ is referred to as the Greenwich Mean Sidereal Time (GMST). $\dot{alpha_G}$ is then the rotation rate of the Earth. An analytical expression for this angle is provided in \cite{Tapley}, which is:
		\[
		\begin{aligned}
		\mathrm{GMST}(\mathrm{UT} 1)=& 4.894961212823058751375704430 \\
				+& \Delta T\{6.300388098984893552276513720\\
				+& \Delta T\left(5.075209994113591478053805523 \times 10^{-15}\right.\\
				&\left.\left.-9.253097568194335640067190688 \times 10^{-24} \Delta T\right)\right\}
				\end{aligned}
		\]
		%
		\noindent where $\Delta T = UT1 - J2000.0$ is expressed in days (including the fractional part of a day).
		%
		\subparagraph{Pseudo-Body-Fixed to Body-Fixed (ECEF)\\}
		%
		\indent The last, and surely most subtle transformation, is the one that accounts for the displacement in Earth's axis of rotation. This displacement is parametrized with the polar angles $x_p$ and $y_p$, which can again be found at \cite{IERS_conventions}. As these angles are sufficiently small, the rotation matrix from PBF to ECEF is:\\
		%
		%
		\begin{equation}
		R_{PBF\rightarrow BF} = 
		\left[ 
		\begin{array}{lll}
		1		& 0 	& x_p \\
		0 		& 1 	& -y_p \\
		-x_p 	& y_p 	& 1 \\
		\end{array}
		\right]
		\label{eq: R_PBF_ECEF}
		\end{equation}
		\paragraph{Full rotation matrix $R_{ECI\rightarrow ECEF}$. \\}
		%
		\indent By simply successively composing rotations, the full rotation matrix from ECI to ECEF is computed as:
		%
		\[
		R_{ECI\rightarrow ECEF} = R_{PBF \rightarrow ECEF} R_{TOD \rightarrow PBF} R_{MOD\rightarrow TOD} R_{ECI\rightarrow MOD}		
		\]
		%
	%
	\subsection{Perifocal (PQW) reference frame.}
	%
		\subsubsection{Definition.}
		%
		\indent The perifocal reference frame is defined as:
		%
		\[
		\arraycolsep=10pt
		\def\arraystretch{1.3}
		PQW\equiv\left\{
		\begin{array}{lll}
		\text{Origin} 	& \equiv 	& \text{Central body's COM} \\
		\text{X-axis} 	& \equiv 	& \text{Origin} \longrightarrow \text{Periapsis.} \\
		\text{Z-axis} 	& \equiv 	& \text{Perpendicular to the osculating orbital plane (out-of-plane)} \\
		\text{Y-axis} 	& \equiv 	& \text{Perpendicular to the X and Z axes forming a right-handed system} \\
		\end{array}
		\right.
		\]
		%	
		\indent This frame takes advantage of the motion being contained in the orbital plane (when using osculating elements, see \textbf{CITE MEAN2OSC}). That means that usually, the problem reduces to evaluating two components of the position and velocity. It also allows for a quite straightforward description of the motion in terms of the Keplerian OEs, assuming elliptical motion. In this case, and using $\underline{q}$ and $\underline{\dot{q}}$ to denote perifocal position and velocity, the perifocal state vector is expressed as:
		%
		\[\begin{array}{cc}
		\underline{q} = \left\{ 
		\begin{array}{c}
		r \cos \theta \\
		r \sin \theta \\
		0
		\end{array}				
		\right\} = 
		\left\{ 
		\begin{array}{c}
		a \left( \cos E - e \right) \\
		a \sin E\\
		0
		\end{array}				
		\right\} 		&
		\underline{\dot{q}} = \dfrac{na}{\sqrt{1-e^2}}
		\left\{ 
		\begin{array}{c}
		-\sin\theta\\
		e + \cos \theta\\
		0
		\end{array}				
		\right\}		
		\end{array}
		\]
		%
		\noindent where $E$ is the eccentric anomaly, $r$ is the orbital radius, $n$ is the mean orbital rate and the remaining parameters are the regular Keplerian OEs. $r$ and $n$ can be expressed as a function of them as:
		%
		\[
		r = \dfrac{a (1 - e^2)}{1 + e \cos\theta} = a (1 - e\cos E) \; ; \qquad n = \sqrt{\dfrac{\mu }{a^3}}		
		\]
		%
		\noindent with $\mu = GM$ being the gravitational parameter of the central body.
		%
		\subsubsection{State vector transformation.}
		%
			\paragraph{Rotation matrix from \& to ECI. \\}
			%
			\indent As done with the ECI to ECEF transformation, this one can also be decomposed in three rotations, each associated with one Keplerian angle.\\
			%
			\indent The first rotation is associated to $\Omega$, being done around the Z ECI axis. The resulting frame will be named I1 (intermediate 1), and the rotation matrix from ECI is:
			%
			\[
			R_{ECI\rightarrow I1}(\Omega) = \left[
			\begin{array}{lll}
			\cos \Omega 	& \sin \Omega 	& 0 \\
			- \sin \Omega 	& \cos \Omega 	& 0 \\
			0 				& 0 			& 1 
			\end{array}						
			\right]	
			\]
			%
			\indent Afterwards, a rotation around X axis of I1 (nodal line) of value $i$ is performed, leading to frame I2. The rotation matrix is simply:
			%
			\[
			R_{I1\rightarrow I2}(\Omega) = \left[
			\begin{array}{lll}
			1 	& 0			 	& 0 \\
			0 	& \cos i 		& \sin i \\
			0 	& -\sin i		& \cos i 
			\end{array}						
			\right]	
			\]
			%
			\indent Finally, a rotation around Z axis of I2 (out-of-plane direction) of value $\omega$ is done, yielding the desired perifocal frame (which we will denote as PQW):
			%
			\[
			R_{I2\rightarrow PQW}(\Omega) = \left[
			\begin{array}{lll}
			\cos \omega 	& \sin \omega 	& 0 \\
			- \sin \omega 	& \cos \omega 	& 0 \\
			0 				& 0 			& 1 
			\end{array}						
			\right]	
			\]
			%
			\indent By composition, the matrix $R_{ECI\rightarrow PQW}$ can easily be calculated:
			%
			\[
			R_{ECI \rightarrow PQW} = R_{I2 \rightarrow PQW}R_{I1 \rightarrow I2} R_{ECI \rightarrow I1} = 
			\left[\begin{array}{lll}
			C \Omega C \omega - S \Omega Ci S\omega 	& S\Omega C\omega + C\Omega Ci S\omega 	& Si S\omega \\
			-C \Omega S \omega - S \Omega Ci c\omega 	& C\Omega Ci C\omega + S\Omega S\omega 	& Si C\omega \\
			S \Omega Si									& - C\Omega Si  						& Ci \\
			\end{array}\right]
			\]
			%
			\paragraph{Angular velocity. \\}
			%
			\indent In order to fully transform the system, it is necessary to calculate the relative angular velocity between both frames. Again, that can be done by composing the angular movements:
			%
			\[
			\underline{\omega}_{PQW \wrt ECI} = \dot{\Omega} \underline{\hat{k}_{I1}} + \dot{i} \underline{\hat{i}_{I2}} + \dot{\omega} \underline{\hat{k}_{PQW}}
			\]
			%			
			\indent Expressing everything in PQW frame:
			%
			\[
			\underline{\omega}_{PQW \wrt ECI}\rvert_{PQW} = \left\{\begin{array}{c} \omega_x \\ omega_y \\ \omega_z \end{array}\right\}\dot{\Omega} R_{ECI\rightarrow PQW} \left\{\begin{array}{c} 0 \\ 0 \\ 1 \end{array}\right\} + \dot{i} R_{I2 \rightarrow PQW}R_{I1 \rightarrow I2} \left\{\begin{array}{c} 1 \\ 0 \\ 0 \end{array}\right\} + \dot{\omega} R_{I2 \rightarrow PQW} \left\{\begin{array}{c} 0 \\ 0 \\ 1 \end{array}\right\}
			\]
			% 
			\indent In virtue of the axial dual form principle, there exists one matrix that, when applied to a certain vector, yields the same result as doing the cross product between $\underline{\omega}$ and that vector. This matrix has the following shape:
			%
			\[
			\Omega_{PQW \wrt ECI}\rvert_{PQW} = \left[ \begin{array}{lll}
			0 			& -\omega_z & \omega_y \\
			\omega_z 	& 0 		& -\omega_x \\
			-\omega_y 	& \omega_x 	& 0
			\end{array}\right]
			\]
			%
			\indent As the relative rotation of the opposite motion is just the opposite of the original, and applying a rotation:
			%
			\[
			\Omega_{ECI \wrt PQW}\rvert_{ECI} = - \Omega_{PQW \wrt ECI}\rvert_{ECI} = - R_{PQW\rightarrow ECI} \Omega_{PQW \wrt ECI}\rvert_{PQW}
			\]
			%
			\indent This will useful later on. 
			\paragraph{Transformation matrices $T_{PQW\rightarrow ECI}$, $T_{ECI\rightarrow PQW}$. \\}
			%
			\indent Let us consider the following coordinate transformation:
			%
			\[
			\underline{x}_{PQW} = R_{ECI\rightarrow PQW} \underline{x}_{ECI}
			\]
			%
			\indent If we want then to also transform its time derivative, one option is to directly differentiate by parts with respect to time:
			%
			\[
			\underline{\dot{x}}_{PQW} = \dot{R}_{ECI\rightarrow PQW} \underline{x}_{ECI} + R_{ECI\rightarrow PQW} \underline{\dot{x}}_{ECI}
			\]
			%
			\noindent where the matrix $\dot{R}_{ECI\rightarrow PQW}$ can be either directly calculated term by term, or otherwise computed as:
			%
			\[
			\dot{R}_{ECI\rightarrow PQW} = \dot{R}_{ECI\rightarrow PQW} \Omega_{ECI \wrt PQW}\rvert_{ECI}
			\]
		
\section{Relative reference systems.}
%
	
	\subsection{LVLH reference frame.}
	%
		\subsubsection{Definition.}
		%
		%
		\subsubsection{State vector transformation.}
		%
		\paragraph{A) Using reference orbit's Keplerian OEs. \\}
		%
			\subparagraph{Rotation matrix from \& to ECI. \\}
			%
			%
			\subparagraph{Angular velocity. \\}
			%
			%			
			\subparagraph{Transformation matrices $T_{LVLH\rightarrow ECI}$, $T_{ECI\rightarrow LVLH}$. \\}
			%
			%
		\paragraph{B) Using reference ECI state vector. \\}
		%
			\subparagraph{Rotation matrix from \& to ECI. \\}
			%
			%
			\subparagraph{Angular velocity. \\}
			%
			%			
			\subparagraph{Transformation matrices $T_{PQW\rightarrow ECI}$, $T_{ECI\rightarrow PQW}$. \\}
			%
			%
		
	\subsection{RTN reference frame.}
	%
		\subsubsection{Definition.}
		%
		%
		\subsubsection{State vector transformation.}
		%
			\paragraph{Rotation matrix from \& to ECI. \\}
			%
			%
			\paragraph{Angular velocity. \\}
			%
			%			
			\paragraph{Transformation matrices $T_{RTN\rightarrow ECI}$, $T_{ECI\rightarrow RTN}$. \\}
			%
			%
		
		
	\subsection{TAN reference frame.}
	%
		\subsubsection{Definition.}
		%
		%
		\subsubsection{State vector transformation.}
		%
			\paragraph{Rotation matrix from \& to LVLH. \\}
			%
			%
			\paragraph{Angular velocity. \\}
			%
			%			
			\paragraph{Transformation matrices $T_{TAN\rightarrow LVLH}$, $T_{LVLH\rightarrow TAN}$. \\}
			%
			%
		
\section{Conversions from OEs to cartesian coordinates.}
%
%
	\subsection{ECI to Keplerian OEs and vice versa.}
	%
	%
	\subsection{Relative Keplerian OEs to LVLH.}
	%
	%
