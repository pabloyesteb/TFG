\chapter{Basics of analytical mechanics.}
%
\label{app: 	Analytical_mechanics}
%
\section{Introduction.}
%
\indent Lagrangian and Hamiltonian mechanics are the two most widely used approaches to analytical mechanics, which constitute a generalization of the classical Newtonian mechanics. These formalisms are closely related to each other, and they usually lead to the same solution. Both approaches will be here succinctly describe, as they embody significant advantages for the treatment of some problems of relative motion. For more in-depth content on this topic, please see \cite{Wiesel, SCFormationFlying}.
%
\section{Lagrangian formulation.}
%
%
	\subsection{Generalized coordinates \& energy of a dynamical system.}
	%
	\indent Let us consider any given dynamical system. Lagrangian mechanics deal with a configuration space $\mathrm{Q}$, which is parametrized by a set of generalized coordinates $q_i\in \mathrm{Q}$ and generalized velocities $\dot{q}_i$. It is through $(q_i, \dot{q}_i)$ how we will describe said dynamical system.
	%
	\subsection{Lagrangian function.}
	%
	%
	\subsection{Lagrange equations.}
	%
	%
\section{Hamiltonian formulation.}
%
%
	\subsection{Hamiltonian function.}
	%
	%
	\subsection{Hamilton equations.}
	%
	%
	\subsection{Canonical transformations.}
	%
	%
	\subsection{Hamilton-Jacobi theory.}
	%
	%
	\subsection{Application to the two-body problem.}
	%
	%