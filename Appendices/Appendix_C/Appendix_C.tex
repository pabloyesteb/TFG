\chapter{Analytical mechanics applied to the two-body problem.}
%
\label{app:App_C}
%
\section{Introduction.}
%
\indent Lagrangian and Hamiltonian mechanics are the two most widely used approaches to analytical mechanics, which constitute a generalization of the classical Newtonian mechanics. These formalisms are closely related to each other, and they usually lead to the same solution. Both approaches will be here succinctly describe, as they embody significant advantages for the treatment of some problems of relative motion. For more in-depth content on this topic, please see \cite{Wiesel, SCFormationFlying}.
%
\section{Lagrangian formulation.}
%
%
	\subsection{Generalized coordinates \& energy of a dynamical system.}
	%
	\indent Let us consider any given dynamical system. Lagrangian mechanics deal with a configuration space $\mathrm{Q}$, which is parametrized by a set of generalized coordinates $q_i\in \mathrm{Q}$ and generalized velocities $\dot{q}_i$. It is through $(q_i, \dot{q}_i)$ how we will describe said dynamical system. \\
	%
	\indent The equations that describe the evolution of the system can be reached through two main approaches. The first one is the Lagrangian formulation, which is described next.
	%
	\subsection{Lagrangian function.}
	%
	\indent For a conservative system, the Lagrangian function $\mathcal{L}$ is a function of the time and the generalized coordinates and velocity, and it is defined as:
	%
	\[
	\mathcal{L} (q_i, \dot{q}_i, t) = T - U	
	\]
	%
	\noindent where $T$ and $U$ are respectively the kinetic and potential energy of the system. 
	%
	\subsection{Lagrange equations.}
	%
	\indent Hamilton's variational principle states that the variation of the integral of the Lagrangian must be zero, that is: 
	%
	\begin{equation}
	\delta \dint_a^b \mathcal{L}(q_i, \dot{q}_i, t) dt= 0
	\label{eqAppC:Hamilton_variational}
	\end{equation}
	%
	\indent The differential counterpart of \eqref{eqAppC:Hamilton_variational} can be reached by using calculus of variations, leading to:
	%
	\begin{equation}
	\label{eqAppC:Euler_lagrange}
	\dfrac{d}{dt} \dfrac{\partial \mathcal{L}}{\partial q_i} -\dfrac{\partial \mathcal{L}}{\partial \dot{q}_i} = 0
	\end{equation}
	%
\section{Hamiltonian formulation.}
%
%
	\subsection{Hamiltonian function.}
	%
	\indent Hamiltonian formulation is a very widely used alternative to the Lagrangian one. The Hamiltonian function $H$ is defined as:
	%
	\begin{equation}
	\label{eqAppC:Hamiltonian} H(q_i, \dot{q}_i,t) = \dsum_i p_i \dot{q}_i  - \mathcal{L}(q_i, \dot{q}_i, t)
	\end{equation}
	%
	\noindent where $p_i$ are the so-called conjugate momenta, who are defined as:
	%
	\begin{equation}
	\label{eqAppC:conj_mom}
	p_i = \dfrac{\partial \mathcal{L}}{\partial \dot{q}_i}
	\end{equation}
	%
	\subsection{Hamilton equations.}
	%
	\indent Considering the Hamiltonian definition \eqref{eqAppC:Hamiltonian}, we can substitute into \eqref{eqAppC:Euler_lagrange}. Identifying coefficients leads to Hamilton's equations:
	%
	\begin{equation}
	\left\{ \begin{array}{ccc}
	\dot{q}_i = \dfrac{\partial H }{\partial p_i} \\
	\dot{p}_i = -\dfrac{\partial H}{\partial q_i}
	\end{array}\right.
	\label{eqAppC:Ham_eqs}
	\end{equation}
	%
	\subsection{Canonical transformations.}
	%
	\indent As with many mathematical problems, a change of variable might simplify the solution process of Hamilton equations. Nonetheless, this must be done carefully, as the variables must remain canonical. Canonical variables are those for which Hamilton's equations \eqref{eqAppC:Ham_eqs} are satisfied. Assuming the original set $(q_i, p_i)$ is canonical, a different canonical set $(Q_i, P_i)$ is obtained through a so-called canonical transformation. \\
	%
	\indent A new Hamiltonian function appears by merely substituting the variable change $(Q_i, P_i) = f(q_i, p_i, t)$ into \eqref{eqAppC:Hamiltonian}, which is denoted by $K(Q_i, P_i, t)$. Hamilton's equations will by definition be obeyed:\\
	%
	\begin{equation}
	\left\{ \begin{array}{ccc}
	\dot{Q}_i = \dfrac{\partial K }{\partial P_i} \\
	\dot{P}_i = -\dfrac{\partial K}{\partial Q_i}
	\end{array}\right.
	\label{eqAppC:Ham_eqs_tx}
	\end{equation}
	%
	\indent Applying Hamilton's principle \eqref{eqAppC:Hamilton_variational} for both sets and substituting \eqref{eqAppC:Hamiltonian}, leads to:
	%
	\begin{subequations}
	\begin{alignat}{4}[left = \empheqlbrace]
	\delta \dint_a^b \left( \dsum_{i = 1}^N p_i \dot{q}_i - H\right) dt = 0\\
	\delta \dint_a^b \left( \dsum_{i = 1}^N P_i \dot{Q}_i - K\right) dt = 0
	\end{alignat}
	\end{subequations}
	%
	\indent The variation of both integrals is identically zero, and although the integrands describe the same dynamical system, they may not be equal. Anyway, it is derived that they can only differ by the total derivative of an arbitrary function $F$, so that:
	%
	\begin{equation}
	\delta\dint_a^b \left(\dsum_{i=1}^N p_i  \dot{q}_i - H(p_i, q_i, t) - \dsum_{i=1}^N P_i \dot{Q}_i + K(Q_i, P_i, t) - \dfrac{dF}{dt}\right)dt = 0
	\end{equation}	
	%
	\indent The function $F$ is the so-called generating function of the transformation. It provides a time-dependent relation between the old $(q_i, p_i) $ and the new $(Q_i, P_i)$ phase spaces. There are four possibilities, depending on which generalized coordinates and/or conjugate momenta are selected as independent variables. The actual relations for the canonical transformation for each of the four cases is provided in table \ref{tabAppC:Can_table}.
	
	%
	\begin{table}[!htb]
	\begin{center}
	\begin{tabular}{|c|c|c|c|}
	\hline 
	\rowcolor{Gray!10}
	$F_1(q_i, Q_i, t)$ & $F_2(q_i, P_i, t)$ & $F_3(p_i, Q_i, t)$ & $F_4(p_i, P_i, t)$ \\ 
	\hline
	\hline 
	$p_i = \dfrac{\partial F_1}{\partial q_i}$ & $p_i = \dfrac{\partial F_2}{\partial q_i}$ & $q_i= -\dfrac{\partial F_3}{\partial p_i}$ & $q_i = -\dfrac{\partial F_4}{\partial p_i}$ 	\\ 
	\hline 
	$P_i = -\dfrac{\partial F_1}{\partial Q_i}$ & $Q_i = \dfrac{\partial F_2}{\partial P_i}$ & $P_i = - \dfrac{\partial F_3}{\partial Q_i}$ & $Q_i = \dfrac{\partial F_4}{\partial P_i}$ \\ 
	\hline 
	\end{tabular} 
	\caption{Canonical transformation relations \cite{Wiesel}.}
	\label{tabAppC:Can_table}
	\end{center}
	\end{table}	
	%
	\indent A further relationship, which is common for all the types of generating functions $F$, allows us to get the new hamiltonian $K$ in terms of the old one $H$ and the generating function itself:
	%
	\begin{equation}
	\label{eqAppC:Ham_tx}K(Q_i, P_i, t) = H(q_i, p_i, t) + \dfrac{\partial F}{ \partial t}
	\end{equation}
	%
	\subsection{Hamilton-Jacobi theory.}
	%
	\indent It is clear that expressing a problem in a certain choice of coordinates can dramatically help with its solution. A possible approach would be to find the transformation which leads to the most simple form of the problem. In this case, the more coordinates or momenta that are missing in the transformed hamiltonian \eqref{eqAppC:Ham_tx}, the more momenta or coordinates are constant. The most advantageous situation then would be that the hamiltonian itself is null, which translates into a constant set of generalized coordinates and conjugate momenta. This is expressed mathematically as:
	%
	\begin{equation}
	H(q_i, p_i, t) + \dfrac{\partial F}{\partial t} = 0
	\label{eqAppC:HJ_eq}
	\end{equation}
	%
	\noindent which is called the Hamilton-Jacobi equation. It is valid for all kinds of transformating functions, although generally, it is applied for $F_2(q_i, P_i$, meaning:
	%
	\begin{equation}
	H\left(q_i, \dfrac{\partial F_2}{\partial q_i}, t\right) + \dfrac{\partial F_2}{\partial t} = 0
	\end{equation}
	%
	\indent This last equation is a partial differential equation for $F_2$, which is also known as Hamilton's principal function, denoted by $S$. Now, the approach to the problem has changed. The direct variant seeks to achieve the transformed Hamiltonian once the generating function has been built. Conversely, the Hamilton-Jacobi approach goes the opposite way: knowing what it is to be achieved (a null hamiltonian), the generating function is to be calculated. This is a staggering fact: deeply, it means that the solution to the dynamical system in the original phase space $(q_i, p_i)$ involves a new phase space $(Q_i, P_i)$ in which all of the coordinates and conjugate momenta are constant. \\
	%
	\indent Additionally, in the common case that the hamiltonian is not an explicit function of time:
	%
	\begin{equation}
	H\left(q_i, \dfrac{\partial S}{\partial q_i}\right) + \dfrac{\partial S}{\partial t} = 0
	\label{eqAppC:HJ_2}
	\end{equation}
	%
	\indent Separating the variables of Hamilton's principal function:
	%
	\begin{equation}
	S(q_i, P_i, t ) = W(q_i, P_i) + S_t(t)
	\label{eqAppC:HJ_3}	
	\end{equation}
	%
	\noindent where $W$ is called Hamilton's characteristic function. Substituting \eqref{eqAppC:HJ_3} into \eqref{eqAppC:HJ_2}:
	%
	\begin{equation}
	H\left(q_i, \dfrac{\partial W}{\partial q_i}\right) + \dfrac{d S}{dt} = 0
	\label{eqAppC:HJ_4}
	\end{equation}%
	%
	\indent This way, we have two different terms that are functions of different variables. It is clear that if two functions of different variables are equal, both must be constant, so we can write:
	%
	\begin{subequations}
	\begin{alignat}{4}[left = \empheqlbrace]
	\dfrac{d S_t}{dt} = -P_1  \Rightarrow S_t(t) = - P_1 t \\[1em]
	H \left( q_i, \dfrac{\partial W}{\partial q_i}\right) = P_1
	\end{alignat}
	\end{subequations}	
	%
	\indent Hamilton-Jacobi theory is object of a very wide range of application. Nonetheless, the next section focuses on how it transforms and simplifies the two-body problem.\\
	%
%
\section{Application to the two-body problem.}
%
\indent \cite[Wiesel][]{Wiesel} provides a succinct variational approach for the two body problem. In particular, the Hamiltonian function of a certain body rotating within the gravity a central body can be expressed as:
%
\begin{equation}
H = \dfrac{1}{2} \left( p_r^2 + \dfrac{1}{r^2} p_{\theta}^2 + \dfrac{1}{r^2 \sin^2 \theta} p_{\phi}^2 \right) - \dfrac{\mu}{r}
\label{eqAppC:Ham_2BP}
\end{equation}
%
\noindent where $r$ is the radial distance, $\theta$ is the latitude and $\phi$ is the longitude. As $H$ is not an explicit function of the time, the Hamilton-Jacobi equation follows directly to the next expression:
%
\begin{equation}
\dfrac{1}{2} \left( \left(\dfrac{\partial W}{\partial r}\right)^2 + \dfrac{1}{r^2} \left(\dfrac{\partial W}{\partial \theta}\right)^2 + \dfrac{1}{r^2 \sin^2 \theta} \left(\dfrac{\partial W}{\partial \phi}\right)^2 \right) - \dfrac{\mu}{r} \equiv P_E
\label{eqAppC:HJ_2BP_1}
\end{equation}
%
\noindent where $P_E$ is the first conjugante momentum of the new canonical variables, and corresponds with the total energy of the orbit. As it can be seen, equation \eqref{eqAppC:HJ_2BP_1} is separable, that is, each derivative can be isolated and assumed to be constant (alone or together with a coefficient). \\
%
\indent Firstly, it is assumed that the generating function $W$ is of the following form:
%
\[
W = W_r(r) + W_{\theta} (\theta) + W_{\phi} (\phi)
\]
%
\indent Substituting this into \eqref{eqAppC:HJ_2BP_1}, it is possible to separate to one side of the equation all dependence on $\phi$. This leads, after some manipulation, to:
%
\begin{equation}
\dfrac{dW_{\phi}}{d\phi} \equiv P_{\phi} = p_{\phi}
\end{equation}
%
\noindent which leads to:
%
\begin{equation}
\dfrac{1}{2} \left( \left(\dfrac{\partial W}{\partial r}\right)^2 + \dfrac{1}{r^2} \left\{\left(\dfrac{\partial W}{\partial \theta}\right)^2 + \dfrac{1}{r^2 \sin^2 \theta} \left(P_{\phi} \right)^2 \right\} \right)- \dfrac{\mu}{r} = P_E
\end{equation}
%
\indent It can be easily identified that the only $\theta$-dependent part of the equation is enclosed by the braces, which leads to:
%
\begin{equation}
\left(\dfrac{\partial W}{\partial \theta}\right)^2 + \dfrac{1}{r^2 \sin^2 \theta} \left(P_{\phi} \right)^2 \equiv P_3^2
\end{equation}
%
\indent As of now, three new constant conjugate momenta have been reached. Half of the problem is now solved, as we still have to find the actual form of the generating function $W$, which allows the calculation of the generalized coordinates $Q_i$. This involves a much more arduous manipulation which will not be herewith developed: just quoted. \\
%
\indent The constant coordinate $Q_E$ conjugate to the total energy $P_E$ needs to be a constant time, as its time derivative has to be non-dimensional. This time turns out to be the perigee passage time $T_0$, or most commonly, its negative $- T_0$. $P_3$ represents the total angular momentum, whose conjugate coordinate is the argument of perigee $\omega$. Finally, the conjugate coordinate to the z-angular momentum $P_{\phi}$ is the RAAN $\Omega$.\\
%
\indent As imposed, the new Hamiltonian function $K(Q_i, P_i)$ is null, which then again means that both the coordinates and the conjugate momenta are conserved along the motion.