\chapter{Relative dynamics around a near-circular reference orbit.}
%
\label{chap: HCW}
%
\section{Introduction.}
%
%
\section{Motion model: Hill equations.}
%
\indent
%
	\subsection{Differential equations of proximity relative motion.} \label{sec: Diff_eqs_prox}
	%
	\indent As the proximity assumption ($\norm{\underline{r}} <<\norm{\underline{R}}$) is quite widely common and valid for a fair range of operations, it is interesting to describe them here briefly, following \cite{Yamanaka_ankersen}. Let us then consider the motion of two spacecrafts, namely, chief and deputy. The general equations of motion for each of them can be written as:
	%
	\begin{alignat}{3}
	&\textbf{Chief} 		&& \Rightarrow   \underline{\ddot{R}}   && = - \mu \dfrac{\underline{R}}{\norm{\underline{R}}} + \underline{a}_{C, d} \label{eqCh2:	chief_eq} \\
	&\textbf{Deputy} 		&& \Rightarrow   \underline{\ddot{R}} + \underline{\ddot{r}}  && =  - \mu \dfrac{\underline{R} + \underline{r}}{\norm{\underline{R} + \underline{r}}} + \underline{a}_{D, d} + \underline{a_f}  \label{eqCh2:deputy_eq}
	\end{alignat}
	%
	\noindent where $\underline{R}$ and $\underline{r}$ are the chief's absolute position vector and the deputy's relative position vector, respectively. $\underline{a}_{\bullet, d}$ is the disturbing acceleration on each spacecraft, while $\underline{a}_f$ denotes the thrust vector of the deputy. In order to facilitate the linearization, let us rewrite the orbital radius of the deputy as:
	%
	\[
	\norm{\underline{R} + \underline{r}} = \left[ (\underline{R} + \underline{r})^T (\underline{R} + \underline{r}) \right]^{1/2} = \norm{\underline{R}} \left( 1 + 2 \dfrac{\underline{R}^T \underline{r}}{\norm{\underline{R}}^2} + \dfrac{\norm{\underline{r}}^2}{\norm{\underline{R}}^2} \right)^{1/2}
	\]
	%
	\noindent then, the effect of the gravity field on the deputy can be expressed as:
	%
	\[
	\dfrac{\underline{R} + \underline{r}}{\norm{\underline{R} + \underline{r}}^{3}} = \dfrac{\underline{R} + \underline{r}}{\norm{\underline{R}}^{3}} = \left( 1 + 2 \dfrac{\underline{R}^T \underline{r}}{\norm{\underline{R}}^2} + \dfrac{\norm{\underline{r}}^2}{\norm{\underline{R}}^2} \right)^{-\frac{3}{2}}
	\]
	\indent Assuming that the relative distance is much smaller than the chief's orbital radius:
	%
	\begin{equation}
	\dfrac{\underline{R} + \underline{r}}{\norm{\underline{R} + \underline{r}}^{3}} \approx \dfrac{\underline{R} + \underline{r}}{\norm{\underline{R}}^{3}} \left[1 - \frac{3}{2} \left( 2 \dfrac{\underline{R}^T \underline{r}}{\norm{\underline{R}}^2} + \dfrac{\norm{\underline{r}}^2}{\norm{\underline{R}}^2} \right) \right] \approx  \dfrac{1}{\norm{\underline{R}}^{3}} \left(\underline{R} + \underline{r} - 3 \dfrac{\underline{R}^T \underline{r}}{\norm{\underline{R}}^2} \underline{R} \right)
	\label{eqCh2: 	Rplusr}
	\end{equation}
	%
	\indent If we now substitute \eqref{eqCh2: 	Rplusr} in the difference \eqref{eqCh2:	deputy_eq} minus \eqref{eqCh2:	chief_eq}, we arrive to:
	%
	\begin{equation}
	\underline{\ddot{r}} = - \dfrac{\mu}{\norm{\underline{R}}^3} \left(\underline{r} - 3 \dfrac{\underline{R}^T \underline{r}}{\norm{\underline{R}}^2} \underline{R} \right) + \underline{a}_f + \underline{a}_{D, d} - \underline{a}_{C, d}
	\label{eqCh2: 	Rel_motion_1}
	\end{equation}
	%
	\indent Experience shows it is convenient to express equation \eqref{eqCh2: 	Rel_motion_1} in a chief-centered frame, for example, the LVLH frame (see section \ref{secAppB:LVLH}). This leads to the need of applying Coriolis' Theorem twice, so as to get the non-inertial effects derived from describing the motion in a rotating frame. The equations of motion take now the following form:
	%
	\begin{equation}
	\underline{\ddot{r}} = - \dfrac{\mu}{\norm{\underline{R}}^3} \left(\underline{r} - 3 \dfrac{\underline{R}^T \underline{r}}{\norm{\underline{R}}^2} \underline{R} \right) - 2 \underline{\omega}\times  \underline{\dot{r}} - \underline{\dot{\omega}}\times \underline{r} - \underline{\omega}\times \left( \underline{\omega} \times \underline{r}\right) + \underline{a}_f + \underline{a}_{D, d} - \underline{a}_{C, d}
	\label{eqCh2: 	Rel_motion_2}
	\end{equation}
	%
	\noindent where $\underline{\omega}$ is the target orbital rate. Let us now express each vector in the RTN frame:
	%
	\begin{align}
	\underline{\omega} = 
	\left\{ \begin{array}{c}
	0 \\
	-\omega \\
	0
	\end{array}\right\} 
	& \qquad &
	\underline{R} = 
	\left\{ \begin{array}{c}
	0 \\
	0 \\
	-R
	\end{array}\right\} 
	& \qquad &
	\underline{r} = 
	\left\{ \begin{array}{c}
	x \\
	y \\
	z
	\end{array}\right\} 
	\end{align}
	\noindent leading to the next expressions for the terms in equation \eqref{eqCh2: 	Rel_motion_2}:
	%
	\begin{align*}
	\underline{\omega}\times \underline{\dot{r}} = 
	\left\{ \begin{array}{c}
	\omega \dot{z} \\
	0 \\
	\omega \dot{x}
	\end{array}\right\} 
	& \qquad &
	\underline{\dot{\omega}} \times \underline{r} = 
	\left\{ \begin{array}{c}
	-\dot{\omega} z\\
	0 \\
	\dot{\omega} x
	\end{array}\right\} 
	\\
	\underline{\omega}\times\left(\underline{\omega} \times\underline{r}\right) = 
	\left\{ \begin{array}{c}
	- \omega^2 x \\
	0 \\
	- \omega^2 z 
	\end{array}\right\} 
	& \qquad &
	\underline{r} - 3 \dfrac{\underline{R}^T \underline{r}}{\norm{\underline{R}}^2} = 
	\left\{ \begin{array}{c}
	x \\
	y \\
	- 2 z 
	\end{array}\right\} 
	\end{align*}
	%
	\noindent and introducing these results into \eqref{eqCh2: 	Rel_motion_2}, we arrive to:
	%
	\begin{equation}
	\left\{
	\begin{array}{c}
	\ddot{x}\\
	\ddot{y}\\
	\ddot{z}
	\end{array}\right\}
	= 
	\left\{
	\begin{array}{c}
	- k \omega^{3/2} x + 2\omega \dot{z} + \dot{\omega} z + \omega^2 x	\\
	-k \omega^{3/2} y \\
	2k\omega^{3/2} z - 2 \omega \dot{x} - \dot{\omega} x + \omega^2 z
	\end{array}\right\}
	+ \underline{a}_f + \underline{a}_{D, d} - \underline{a}_{C, d}
	\label{eqCh2: Diff_eqs_prox}
	\end{equation}
	%
	\noindent where $k$ is a constant defined by 
	%
	\[
	\dfrac{\mu}{R^3} = \left(\dfrac{\mu }{h^{3/2}}\right) \equiv k \omega^{3/2} \Leftrightarrow k \equiv \dfrac{\mu} {h^{3/2}}
	\]
	%
	\indent and $h = \omega R^2$ is the chief's angular momentum. TO BE CONTINUED, MATRIX FORM OF HILL EQUATIONS. ALSO ADD VECTOR NOTATION OF EACH. RTN FRAME FOR HCW?

	\subsection{Hill equations.}
	%
	%
\section{Solutions of Hill equations.}
%
%
	\subsection{Approaches.}
	%
	%
		\subsection{Direct numerical integration.}
		%
		%
		\subsection{Clohessy-Wiltshire solution.}
		%
		%
		\subsection{State transition matrix propagation.}
		%
		%
		\subsubsection{Jordan canonical decomposition.}
		%
		%
	\subsection{Results: Comparison against High-Fidelity.}
	%
	%
		\subsubsection{Scenario 1: XXXXXX}
		%
		%
		\subsubsection{Scenario 2: XXXXXX}
		%
		%
		\subsubsection{Scenario 3: XXXXXX}
		%
		%

\section{Orbit safety in near-circular orbits.}
%
%
	\subsection{Orbit safety concept.}
	%
	\indent Orbit or trajectory safety can be understood as how protected from danger or risk our spacecraft set is. It is essentially a kinematic condition, being closely related to the notions of relative position and distance, and how these vary over time. Both branches of relative motion (\ie rendez-vous and formation flying) are subject to this concept. \\
	%
	\indent One of the biggest concerns about orbit safety is trajectory uncertainty. If we knew the exact position of each spacecraft, safety margins could be lowered down to almost zero, but unfortunately, that is not the case. For this reason, distance margins have to be established. As we have seen, there are three basic components in relative motion: radial, along-track and cross-track. The most susceptible one to estimation errors is the along-track component, due to the high influence of the semimajor axis on the angular rate, thus on the angular position \cite{Eckstein}. Hence, along-track uncertainty will always tend to be much higher than either radial or cross-track.\\
	%
	\indent With this fact in mind, it seems logical to try and separate the spacecrafts in radial or cross-track components, as motion can be more accurately predicted in those directions. It is here where the eccentricity/inclination vector separation concept raises, as an approach to describe the periodic relative motion that takes place in said components. In the following section, its grounds and applications will be discussed, both in general terms and applied to near-circular reference orbits.
	%
	\subsection{Eccentricity-inclination vector separation strategy.}
	%
	%
		\subsubsection{Eccentricity and inclination vectors.}
		%
		\indent The eccentricity and inclination vectors constitute an interesting way to parametrize relative motion. This description was firstly introduced by Eckstein \cite{Eckstein}, aimed at geostationary orbits. Later on, it was extended to proximity LEO operations \cite{Montenbruck_DAmico}, which is today's main scope of application. \\
		%
		\indent This approach to orbit safety will be treated recurrently along the thesis, being progressively extended as the orbits grow in complexity. In this section, the basics of this concept will be described, starting with the parametrization of proximity relative motion in terms of the eccentricity and inclination vectors (see appendix \ref{sec:EIROE}). After that, the two setups of $\delta \underline{e}$ and $\delta \underline{i}$ are discussed.
		%
%		\subsubsection{Linearized effect of $\delta e$ and $\delta i$ in relative position.}
		\subsubsection{Linearized equations of relative motion in terms of $\delta\underline{e}$ and $\delta\underline{i}$.}
		%
		\indent Our target is to obtain an expression which relates the eccentricity and inclination vector components to the radial, along-track and cross-track distances. This can be done by applying some transformations to the already available Keplerian OE set to RTN (see \ref{sec:RKOE2RTN}), or through a geometric analysis of the motion. As the first one is almost trivial (considered said section), let us proceed with the second one, explained in more detail in \cite{DAmico_Montenbruck}:\\
		%
			\paragraph{I. Effect of relative eccentricity vector $\delta \underline{e}$. \\}
			%
			\indent The relative eccentricity vector accounts for the variation in the eccentricity value and the argument of perigee. Assuming no other variation in any element, it can be projected on the chief's orbital plane, leading to:
			%
			\[
			\underline{e} = \left\{ 
			\begin{array}{c}
			 e_x \\[1.5em]
			 e_y
			\end{array}\right\} 
			 = e \left\{\begin{array}{c}
			 \cos\omega \\[1.5em]
			 \sin\omega 
			\end{array}\right\} 
			\Rightarrow
			\delta \underline{e} = 
			\left\{ 
			\begin{array}{c}
			\delta e_x \\[1.5em]
			\delta e_y
			\end{array}
			\right\} = \delta e
			\left\{ 
			\begin{array}{c}
			\cos\varphi \\[1.5em]
			\sin\varphi
			\end{array}
			\right\}
			\]
			%
			\indent The relative eccentricity has an effect on the in-plane motion. That is to say, on the radial and along-track components. In order to get these, we must first derive some expressions for the orbital radius $r$ and the difference $\theta - M$. \\
			%
			\indent The orbital radius for near-circular orbits ($\abs{e}<<1$) can be expressed as:
			%
			\begin{align}
			&\nonumber \dfrac{r}{a} = 1 -e\cos E \approx 1 - e\cos M = 1 - e\cos\left(\lambda - \omega\right) = 1 - e\cos\omega \cos \lambda+ e \sin \lambda \sin\omega \\
			& \dfrac{r}{a} \approx 1 - e_x \cos\lambda - e_y \sin \lambda
			\label{eqCh2:orbital_radius}
			\end{align}
			%
			\noindent where $\lambda$ is the mean argument of latitude, which embodies the time-varying element of the E/I element set. The second auxiliary expression $\theta - M$ can be obtained in many handbooks, by looking for the series expansion of the mean anomaly in terms of the true \cite{Battin}:
			%
			\begin{align}
			&\nonumber M =\theta+ 2\dsum_{n=1}^{\infty} (-1)^{n} \left( \dfrac{1}{n} + \sqrt{1 - e^2} \right) \beta^n \sin n\theta \underset{e<<1}{\approx} \theta - 2 e \sin\theta\\
			& \Rightarrow \theta - M = 2 e\sin M = 2 e \sin\left(\lambda - \omega\right) = 2e\left(\sin\lambda\cos\omega - \cos\lambda\sin\omega\right) = 2 e_x\sin\lambda - 2 e_y \cos\lambda
			\label{eqCh2:M_expansion}
			\end{align}
			%%
			\indent Assuming that both spacecrafts have the same mean argument of latitude and semimajor axis, the radial and along-track distance ($\delta r_R$, $\delta r_T$) between chief and deputy are due to (a) the difference in orbital radius $\delta r$ and (b) the difference in true argument of latitude $\delta u = \delta \theta + \delta \omega$. With this in mind, let us construct a graphical representation of the situation, shown in \ref{figCh2:ecc_effect}:
			%
			\begin{figure}[!htb]
			\centering\includegraphics[width = 0.4\linewidth]{Figures/Dummy_figure}
			\caption{Effect of $\delta r$ and $\delta u$ in radial and along-track distances.}
			\label{figCh2:ecc_effect}
			\end{figure}
			%
			\FloatBarrier
			%
			\indent The angle $\alpha$ can be developed in terms of known magnitudes, that is:
			%
			\begin{align}
			&\nonumber \alpha = u_2 - u_1 = \left(\lambda_2 - \lambda_1\right) + \left(\theta_2 - M_2\right) - \left(\theta_1 - M_1\right) = \\
			&\label{eqCh2:alpha}0 + 2 e_2 \sin M_2 - 2 e_1 \sin M_1 = 2 \delta e_x \sin\lambda - 2 \delta e_y \cos\lambda \mathcal{O}(\delta e) \sim 10^{-3} \ \
			\end{align}
			\subparagraph{\textcolor{GMVred}{I.A.} Effect of $\delta \underline{e}$ in $\delta r_R$. \\}
			%
			\indent Radial distance can be derived from figure \ref{figCh2:ecc_effect} as:
			%
			\begin{align}
			\nonumber \delta r_R = r_2 \cos\alpha - r_1 \approx r_2 - r_1 = \delta r
			\end{align}
			%
			\indent In virtue of equation \eqref{eqCh2:orbital_radius}:
			%
			\begin{align}
			&\nonumber \delta r_R = \delta_r = a \left(e_1 \cos M_1 - e_2 \cos M_2 
			&\nonumber \dfrac{\delta r_R}{a} = e_1 \left( \cos\lambda \cos\omega_1 - \sin\lambda\sin\omega_1 \right) - e_2 \left(\cos \lambda \cos\omega_2 - \sin\lambda\sin\omega_2\right) = \cos\lambda \left(e_{x1} - e_{x2}\right) - \sin\lambda \left(e_{y1} - e_{y2}\right) \\
			&\label{eqCh2:r_R_ecc} \Rightarrow \dfrac{\delta r_R}{a}\rvert_{\delta e} \approx - \delta e_y\sin\lambda - \delta e_x \cos\lambda
			\end{align}
			%
			\subparagraph{\textcolor{GMVred}{I.B.} Effect of $\delta \underline{e}$ in $\delta r_T$. \\}
			%
			\indent The along-track distance can be computed in a similar manner, neglecting terms of order $e\delta e$ and higher:
			%
			\begin{align}
			&\nonumber \delta r_T = r_2 \sin\alpha \underset{\abs{\alpha} << 1}{\approx} a \left(1 - e_2 \cos M_2\right) \alpha \underset{\abs{e\delta e} << \delta e}{\approx} a\alpha \\
			& \label{eqCh2:r_T_ecc}\Rightarrow \dfrac{\delta r_T}{a}\rvert_{\delta e} \approx \alpha =  2 \delta e_x \sin\lambda - 2 \delta e_y \cos\lambda
			\end{align}
			%
			\paragraph{\textcolor{GMVred}{II.} Effect of relative inclination vector $\delta \underline{i}$. \\}
			%
			\indent As presented in section \ref{sec:EIROE}, the relative inclination vector takes the following form:
			%
			\[
			\delta \underline{i} = \sin\delta i\left\{
			\begin{array}{c}
			\cos\psi\\
			\sin\psi
			\end{array}
			\right\}
			\approx \left\{
			\begin{array}{c}
			\delta i \\
			\sin i \delta \Omega
			\end{array}
			\right\}
			\]
			%
			\indent Its effect on the cross-track distance can be derived from the spherical triangle in figure \ref{figCh2:inc_rN}, applying the law of sines as:
			%
			\begin{align}
			&\label{eqCh2:r_N_1} \dfrac{\sin \left(u_2 - \psi\right)}{\sin\frac{\pi}{2}} = \dfrac{\sin\dfrac{\delta r_N}{a}}{\sin \delta i} \Rightarrow \sin \dfrac{\delta r_N}{a} \underset{\delta r_N << a}{\approx} \dfrac{\delta r_N}{a} = \sin \delta i \sin\left(u_2 - \psi\right)
			\end{align}
			%
			\indent However, we need an expression with the mean argument of latitude instead of the true one. As we look to retain only first-order terms:
			%
			\begin{align}
			&\nonumber \sin(u -  \psi) 	&&= \sin \left[ (\lambda - \psi) + (u - \lambda)\right] = \sin (\lambda - \psi) \cos (\u - \lambda) + \cos(\lambda-\psi)\sin (u -  \lambda) =\\
			&\label{eqCh2:Lambda_u} 							&&  \sin (\lambda - \psi) \cos(2e\sin\theta) + \cos(\lambda-\psi)\sin (2e\sin \theta) \underset{\abs{2e\sin\theta} <<1}{\approx}\sin (\lambda - \psi) \left(1 - \dfrac{(2e\sin\theta)^{2}}{2}\right) +  \cos(\lambda-\psi) 2e\sin\theta = \sin (\lambda - \psi) + \mathcal{O}(e)
			\end{align}
			%
			\indent Substituting \eqref{eqCh2:Lambda_u} into \eqref{eqCh2:r_N_1}:
			%
			\begin{align}
			&\nonumber \dfrac{\delta r_N}{a} \approx \sin\delta i \sin(\lambda - \psi) = \sin\delta i \cos\psi \sin \lambda - \sin\delta i \sin \theta \cos\lambda \\
			& \label{eqCh2:r_N_2} \Rightarrow \dfrac{\delta r_N}{a}\rvert_{\delta i} \approx \delta i_x \sin \lambda - \delta i_y \cos\lambda
			\end{align}
			%
			\begin{figure}[!htb]
			\centering\includegraphics[width = 0.4\linewidth]{Figures/Dummy_figure}
			\caption{Inclination vector and cross-track distance.}
			\label{figCh2:inc_rN}
			\end{figure}
			%
			\FloatBarrier
			%
			\paragraph{\textcolor{GMVred}{III.} Effect of relative semimajor axis and mean argument of latitude $\delta a$, $\delta \lambda$. \\}
			%
			\indent As previously stated, the relative semimajor axis has a crucial influence on the relative dynamics, as it varies the angular rate of the orbit, leading to a unbounded along-track drift. Consequently, let us estimate, to first order, the effect of a relative semimajor axis $\delta a$ in the angular rate $\dot{\lambda}$:
			%
			\begin{align}
			&\nonumber \dot{\lambda}(a) = \dot{\omega} + \dot{M}(a) = \dot{M}(a) = n = \dsqrt{\mu}{a^3} \\
			& \nonumber \Rightarrow \delta\dot{\lambda}_{\delta a} = \dot{\lambda}(a + \delta_a) - \dot{\lambda}(a) =\dsqrt{\mu}{(a + \delta a)^3} - \dsqrt{\mu}{a^3} = \dsqrt{\mu}{a^3}\left[ \dfrac{1}{\left(1 + \frac{\delta a}{a}\right)^{3/2}} - 1\right] \underset{\abs{\frac{\delta a}{a}<<1}}{\approx} - \dfrac{3}{2}\dfrac{\delta a}{a} n\\
			%
			& \label{eqCh2:dLambda} \delta \lambda\rvert_{\delta a} \dint_{t_0}^{t} \delta \dot{\lambda} dt = - \dfrac{3}{2}\dfrac{\delta a}{a} n (t - t_0) = - \dfrac{3}{2}\dfrac{\delta a}{a} (M - M_0) = - \dfrac{3}{2}\dfrac{\delta a}{a} (\lambda - \lambda_0)
			\end{align}
			%
			\subparagraph{\textcolor{GMVred}{III.A.} Effect of $\delta \underline{a}, \; \delta \lambda$ in $\delta r_R$. \\}
			%
			\indent The expression for the radial distance induced by $\delta a$ is, assuming a constant mean argument of latitude $\lambda$ and argument of perigee $\omega$:
			%
			\begin{align}
			&\nonumber \delta r_R \approx r_2 - r_1 \approx (a + \delta a) (1 - e_2 \cos M_2) - a (1 - e_1 \cos M_1) \approx \delta a \\
			& \label{eqCh2:r_R_a} \dfrac{\delta r_R}{a}\rvert_{\delta a} \approx \dfrac{\delta a}{a}
			\end{align}
			%
			\noindent where it is assumed that the relative mean argument of latitude has no effect on it.
			%
			\subparagraph{\textcolor{GMVred}{III.B.} Effect of $\delta \underline{a}, \; \delta \lambda$ in $\delta r_T$. \\}
			%
			\indent The effect in along-track distance is due to both $\delta a$ and $\delta \lambda$, as $\lambda$ is intrinsically affected by $\delta a$ (see eq. \eqref{eqCh2:dLambda}). Without further ado:
			%
			\begin{align}
			&\nonumber \delta r_T = r_2 \sin(u_2 - u_1) \underset{\eqref{eqCh2:dLambda}}{\approx} r_2 \sin(\lambda_2 - \lambda_1) \underset{\abs{\lambda_2 - \lambda_1} << 1}{\approx} (a + \delta a) \left(1 - e_2 \cos M_2\right) (\lambda_2 - \lambda_1) \underset{e<<1}{\approx} (a + \delta a) \left[\delta \lambda - \dfrac{3}{2}\dfrac{\delta a}{a} (\lambda - \lambda_0) \right] 
			& \label{eqCh2:r_T_a_lam} \dfrac{\delta r_T}{a}\rvert_{\delta a \delta\lambda} \approx \delta\lambda - \dfrac{3}{2}\dfrac{\delta a}{a} (\lambda - \lambda_0)
			\end{align}
			%
			\paragraph{\textcolor{GMVred}{IV.} Final set of linearized equations. \\}
			%
			\indent Now, considering all the previous effects, we can reach a compact set of equations \cite{Montenbruck_DAmico}:
			%
			\begin{align}
			\label{eqCh2:Lin_eqs_full} \left\{ \begin{array}{c}
			\delta r_R\\
			\delta r_T\\
			\delta r_N
			\end{array}\right\}
			 = 
			\left\{ \begin{array}{c}
			\delta a\\
			\delta \lambda - \frac{3}{2}\frac{\delta a}{a} (\lambda - \lambda_0) \\
			0
			\end{array}\right\}
			+
			a\delta e \left\{ \begin{array}{c}
			-\cos(\lambda - \varphi)\\
			2\sin (\lambda - \varphi) \\
			0
			\end{array}\right\}
			+
			a\delta i \left\{ \begin{array}{c}
			0 \\
			0 \\
			\sin (\lambda - \psi)
			\end{array}\right\}
			\\
			&\label{eqCh2:Lin_eqs_compact} \left\{ \begin{array}{c}
			\delta r_R\\
			\delta r_T\\
			\delta r_N
			\end{array}\right\} 
			= 
			\left[\begin{array}{cccc}
			\delta a / a 	& 0 						& -\delta e_x 	& -\delta e_y \\
			\delta \lambda 	& -\frac{3}{2} \delta a / a & -2\delta e_y 	& 2\delta e_x \\
			0				& 0 						& -\delta i_y 	& \delta i_x \\
			\end{array}\right]
			\left\{ \begin{array}{c}
			1\\
			\lambda - \lambda_0\\
			\cos\lambda \\
			\sin \lambda
			\end{array}\right\} 
			\end{align}
			%
			\indent A comparison of \ref{eqCh2:Lin_eqs_compact} and the analytical solution of the HCW equations show a complete correspondence of individual terms, hence proving the mathematical equivalence of both formulations. This equation can also be understood as a linearized approach to the Gauss Variational Equations, and finally, as an alternative to the mapping provided in \ref{sec:RKOE2RTN}.
			%
		\subsubsection{Collision avoidance for bounded trajectories: $\delta{\underline{e}} / \delta{\underline{i}}$ separation.}
		%
		\indent Let us now particularize the previous equations for the bounded trajectory case. This means that there is no mutual drift, hence relative semimajor axis drops to zero. The equations for the relative distances between both spacecrafts can be expressed as:
		%
		\begin{subequations}
		\label{eqCh2:EI_1}
		\begin{alignat}{4}[left = \empheqlbrace]
		& \dfrac{\delta r_R}{a} = -\delta e \cos(\lambda -  \varphi) \\
		& \dfrac{\delta r_T}{a} = 2\delta e \sin(\lambda -  \varphi) \\
		& \dfrac{\delta r_N}{a} =  \delta i \sin(\lambda -  \psi) \\
		\end{alignat}
		\end{subequations}
		%
		\indent This formulation eases facilitates the safety analysis, as we will later see. In order to avoid collision hazard, considering along-track position uncertainties, a proper separation in radial and cross-track components must be set up. As shown in  \cite{Terrasar}, two possible strategies to achieve this are (a) a parallel alignment of $\delta \underline{e}$ and $\delta \underline{i}$ and (b) an antiparallel (orthogonal) arrangement.\\
		%
		\indent Before discussing these alternatives, let us do a quick analysis of the relevant positions that may arise from \eqref{eqCh2:EI_1}. If $\lambda = \varphi$, tangential distance vanishes, that is, the deputy is right below the chief (at a certain cross-track distance). Conversely, if $\lambda = \varphi + \frac{pi}{2}$, radial distance vanishes, and the deputy comes just in front of the chief. Similar statements can be made with the out-of-plane motion. A graphical representation of the mentioned geometry is shown in figure \ref{figCh2:E_I_general}.
		%
		%
		\begin{figure}[!htb]
		\centering\includegraphics[width = 0.4\linewidth]{Figures/Dummy_figure}
		\caption{Relative motion in RTN frame for $\delta a = 0$ and a general $\delta \underline{e}$ - $\delta \underline{i}$ alignment.}
		\label{figCh2:E_I_general}
		\end{figure}
		%
		\FloatBarrier
		%
			\paragraph{\textcolor{GMVred}{A.} Parallel configuration. \\}
			%
			\indent If the relative eccentricity and inclination vectors are parallel, then we can write:
			%
			\begin{align}
			& \delta\underline{e} \rVert \delta \underline{i} \Rightarrow \delta \underline{e} \cdot \delta\underline{i} = \delta e \delta i \left\{ \begin{array}{cc}
			\cos\varphi & \sin\varphi
			\end{array}\right}
			\left\{ \begin{array}{c}
			\cos\psi\\
			\sin\psi
			\end{array}\right\} =  \delta e \delta i \cos (\varphi - \psi) \underset{\delta\underline{e} \rVert \delta \underline{i}} {\xlongequal}  1\\
			& \label{eqCh2:Parallel_ei}\Longrightarrow \varphi = \psi + 2\pi k\; , \qquad k \in \mathbb{Z}
			\end{align}
			%
			\indent For this configuration, the radial and cross-track distances never drop to zero simultaneously. In fact, if $\delta r_R = 0$, then $\delta r_N$ is maximum, and viceversa. Hence, minimum separation satisfies:
			%
			\[
			\mod{\delta \underline{r}} \geq \min\left(a \delta e, \; a\delta i\right)
			\]
			%
			\indent In conclusion, separation between the spacecrafts is ensured, even if tangential distance is null. A graphical representation of this configuration can be seen in 
			%

			\paragraph{\textcolor{GMVred}{B.}Antiparallel configuration. \\}
			%
			\indent The condition for antiparallel configuration can be expressed as:
			%
			\begin{align}
			& \delta\underline{e} \perp \delta \underline{i} \Rightarrow \delta \underline{e} \cdot \delta\underline{i} = \delta e \delta i \left\{ \begin{array}{cc}
			\cos\varphi & \sin\varphi
			\end{array}\right}
			\left\{ \begin{array}{c}
			\cos\psi\\
			\sin\psi
			\end{array}\right\} =  \delta e \delta i \cos (\varphi - \psi) \underset{\delta\underline{e} \perp \delta \underline{i}} {\xlongequal} 0\\
			& \label{eqCh2:Antiparallel_ei}\Longrightarrow \varphi = \psi + (2k+1)\frac{\pi}{2}\; , \qquad k \in \mathbb{Z}
			\end{align}
			%
			\begin{figure}[ht]
			\centering
			\medskip
			\begin{subfigure}[t]{.32\linewidth}
			\centering\includegraphics[width=\linewidth]{Figures/Dummy_figure}
			\caption{Parallel configuration.}
			\label{figCh2:Antiparallel}
			\end{subfigure}
			\begin{subfigure}[t]{.32\linewidth}
			\centering\includegraphics[width=\linewidth]{Figures/Dummy_figure}
			\caption{Antiparallel configuration.}
			\label{figCh2:Antiparallel}
			\end{subfigure}
			\caption{Relative motion for parallel and antiparallel $\delta \underline{e}/\delta\underline{i}$ vectors.}
			\label{fig:E_I_par_anti}
			\end{figure}
			%
			\FloatBarrier
			%
