\chapter{The industrial problem}\label{chap:2}
The whole validation pipeline proposed in the next section is illustrated with realistic data from a real industrial problem of particular interest to the aerospace sector: stress analysis and Reserve Factor prediction in fuselage structures. Results from statistical tests, graphs and tables are provided alongside the theoretical background at each step. This chapter is thus dedicated to introduce the industrial problem used for illustration of results.\\
%
\indent The regression problem at hand is an application of neural networks (NN) in aircraft stress engineering. For typical aircraft fuselage panel design, the dominant form of stiffened post buckling failure under shear loading is forced crippling\cite{bijlaard1955buckling}. This occurs when the shear buckles in the panel skin force the attached stiffener flanges to deform out-of-plane. There are other failure modes related to buckling, tension, and compression.\\
%
\indent The surrogate model presented here is the ''MS-S18'' model, whose aim is to predict Reserve Factors ($RF$) that quantify failure likelihood for regions of the aircraft subject to different loads happening in flight maneuvers (vid. \autoref{fig:diagrama_cajas}). There are six possible failure modes, whose names are ''Forced Crippling'', ''Column Buckling'', ''In Plane'', ''Net Tension'', ''Pure Compression'', and ''Shear Panel Failure''.\\
%
\indent Input data consists of 26 features (loads applied to a specific region of the aircraft and different maneuver specs). This variables are either numerical continuous variables (the magnitudes of the applied loads, for instnace) or categorical variables. This are variables which can only take a restricted set of values (\eg boolean variables, integer variables, etc.) There are three categorical variables alongside 23 numerical ones in MS-S18's dataset: ''dp'', ''Frame'', and ''Stringer''. They contain geometrical information about the region where the loads described by the numerical variables are being applied. For instance, ''Frame'' and ''Stringer'' identify structural elements of the fuselage (their values are string variables like ''Fr)\\
%
\indent Output data consists of 6 reserve factors that quantify the stress failure likelihood: $\mathbf{y}=(RF_1,RF_2,RF_3,RF_4,RF_5,RF_6)$, where $RF_i\in [0,5]$. In this (logarithmic) scale, $0$ means extreme risk and $5$ means risk extremely low.\\
%
\indent This study prioritizes assessing the efficacy of the developed validation tool, not optimizing the neural network model itself. Therefore, the specific architecture and additional features of the NN are intentionally treated as a black box. Our focus remains solely on its inputs --the 26 aforementioned features-- and its outputs --the 6 reserve factors representing 6 distinct failure modes--. This simplification allows us to isolate and evaluate the performance of the validation tool without introducing confounding variables related to the specific neural network design (\ie, without specifying the network's architecture and its parameters, $W$).\\
%
\begin{figure}[!htb]
	\centering
	\begin{tikzpicture}
		% Caja 1
		\node[draw, minimum width=2cm, minimum height=3cm, font=\scriptsize] (box1) at (0,0) {
			\begin{tabular}{c}
				\textbf{INPUT $X$}\\
				Loads + Geometry +\\
			\end{tabular}
		};
		\node[below=0.1cm of box1] {$\mathbf{X}=(x_1,x_2,...,x_n)$};
		
		% Caja 2
		\node[draw, minimum width=2cm, minimum height=3cm, right=0.5cm of box1, font=\scriptsize] (box2) {
			\begin{tabular}{c}
				\textbf{SURROGATE MODEL}\\
				Stress learning model\\
				trained by splitting cases into\\
				train/test set and minimize loss\\
				function (Deep Neural Network)
			\end{tabular}
		};
		\node[below=0.1cm of box2] {$\cal{F}:\mathbf{X} \rightarrow \mathbf{Y}$};
		
		% Caja 3
		\node[draw, minimum width=2cm, minimum height=3cm, right=0.5cm of box2, font=\scriptsize] (box3) {
			\begin{tabular}{c}
				\textbf{PREDICTION (OUTPUT) Y}\\
				A set of Reserve Factors ($RF$)\\
				which characterize the likelihood\\
				of different failure modes.
			\end{tabular}
		};
		\node[below=0.1cm of box3] {$\mathbf{Y}=(y_1,y_2,...,y_m)$};
		
		% Flechas
		\draw[->, >=Stealth] (box1.east) -- (box2.west);
		\draw[->, >=Stealth] (box2.east) -- (box3.west);
	\end{tikzpicture}
	\caption{Surrogate model pipeline}
	\label{fig:diagrama_cajas}
\end{figure}