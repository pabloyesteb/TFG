\chapter{Relative dynamics around a near-circular reference orbit.}
%
\label{chap: HCW}
%
\section{Introduction.}
%
%
\section{Equations of motion.}
%
%
	\subsection{Differential equations of proximity relative motion.} \label{sec: Diff_eqs_prox}
	%
	\indent Let us consider the motion of two spacecrafts, namely, chief and deputy. The general equations of motion for each of them can be written as:
	%
	\begin{equation}
	\begin{align}
	\left\{ \begin{array}{ccccc}
	\text{Chief} 		& \Rightarrow & \underline{\ddot{R}} & = & - \mu \dfrac{\underline{R}}{\norm{\underline{R}}} + \underline{a}_{C, d} \label{eqCh2:	chief_eq}\\[1.2em]
	\text{Deputy} 	& \Rightarrow & \underline{\ddot{R}} + \underline{\ddot{r}} & = & - \mu \dfrac{\underline{R} + \underline{r}}{\norm{\underline{R} + \underline{r}}} + \underline{a}_{D, d} + \underline{a_f} \label{eqCh2:	deputy_eq}
	\end{array}\right.
	\begin{align}
	\end{equation}
	%
	\noindent where $\underline{R}$ and $\underline{r}$ are the chief's absolute position vector and the deputy's relative position vector, respectively. $\underline{a}_{\bullet, d}$ is the disturbing acceleration on each spacecraft, while $\underline{a}_f$ denotes the thrust vector of the deputy. In order to facilitate the linearization, let us rewrite the orbital radius of the deputy as:
	%
	\[
	\norm{\underline{R} + \underline{r}} = \left[ (\underline{R} + \underline{r})^T (\underline{R} + \underline{r}) \right]^{1/2} = \norm{\underline{R}} \right( 1 + 2 \dfrac{\underline{R}^T \underline{r}}{\norm{\underline{R}}^2} + \dfrac{\norm{\underline{r}}^2}{\norm{\underline{R}}^2} \right)^{1/2}
	\]
	%
	\noindent then, the effect of the gravity field on the deputy can be expressed as:
	%
	\[
	\dfrac{\underline{R} + \underline{r}}{\norm{\underline{R} + \underline{r}}^{3}} = \dfrac{\underline{R} + \underline{r}}{\norm{\underline{R}}^{3}} = \left( 1 + 2 \dfrac{\underline{R}^T \underline{r}}{\norm{\underline{R}}^2} + \dfrac{\norm{\underline{r}}^2}{\norm{\underline{R}}^2} \right)^{-\frac{3}{2}}
	\]
	\indent Assuming that the relative distance is much smaller than the chief's orbital radius:
	%
	\begin{equation}
	\dfrac{\underline{R} + \underline{r}}{\norm{\underline{R} + \underline{r}}^{3}} \approx \dfrac{\underline{R} + \underline{r}}{\norm{\underline{R}}^{3}} \left[1 - \frac{3}{2} \left( 2 \dfrac{\underline{R}^T \underline{r}}{\norm{\underline{R}}^2} + \dfrac{\norm{\underline{r}}^2}{\norm{\underline{R}}^2} \right) \right] \approx  \dfrac{1}{\norm{\underline{R}}^{3}} \left(\underline{R} + \underline{r} - 3 \dfrac{\underline{R}^T \underline{r}}{\norm{\underline{R}}^2} \underline{R}
	\label{eqCh2: 	Rplusr}
	\end{equation}
	%
	\indent If we now substitute \eqref{eqCh2: 	Rplusr} in the difference \eqref{eqCh2:	deputy_eq} minus \eqref{eqCh2:	chief_eq}, we arrive to:
	%
	\begin{equation}
	\underline{\ddot{r}} = - \dfrac{\mu}{\norm{\underline{R}}^3} \left(\underline{r} - 3 \dfrac{\underline{R}^T \underline{r}}{\norm{\underline{R}}^2} \underline{R} \right) + \underline{a}_f + \underline{a}_{D, d} - \underline{a}_{C, d}
	\label{eqCh2: 	Rel_motion_1}
	\end{equation}
	%
	\indent Experience shows it is convenient to express equation \eqref{eqCh2: 	Rel_motion_1} in the RTN frame (see \ref{sec: RTN}). This leads to the need of applying Coriolis' Theorem twice, so as to get the non-inertial effects derived from describing the motion in a rotating frame. The equations of motion take now the following form:
	%
	\begin{equation}
	\underline{\ddot{r}} = - \dfrac{\mu}{\norm{\underline{R}}^3} \left(\underline{r} - 3 \dfrac{\underline{R}^T \underline{r}}{\norm{\underline{R}}^2} \underline{R} \right) - 2 \underline{\omega}\times  \underline{\dot{r}} - \underline{\dot{omega}}\times \underline{r} - \underline{\omega}\times \left( \underline{\omega} \times \underline{r}\right) + \underline{a}_f + \underline{a}_{D, d} - \underline{a}_{C, d}
	\label{eqCh2: 	Rel_motion_2}
	\end{equation}
	%
	\noindent where $\underline{\omega}$ is the target orbital rate. Let us now express each vector in the RTN frame:
	%
	\[\begin{array}{ccc}
	\underline{\omega} = \left\{ \begin{array}{c}
	
	\end{array}
	\end{array}
	
	\]
	
	%
	\noindent and introducing this result into \eqref{eqCh2: 	Rel_motion_2}, we arrive to \cite{Yamanaka_Ankersen}:
	%
	\begin{equation}
	\left\{
	\begin{array}{c}
	\ddot{x}\\
	\ddot{y}\\
	\ddot{z}
	\end{array}\right\}
	= 
	\left\{
	\begin{array}{c}
	- k \omega^{3/2} x + 2\omega \dot{z} + \dot{\omega} z + \omega^2 x	
	\end{array}\right\}
	+ \underline{a}_f + \underline{a}_{D, d} - \underline{a}_{C, d}
	\label{eqCh2: Diff_eqs_prox}
	\end{equation}
	%
	\noindent where $k$ is a constant defined by 
	%
	\[
	\dfrac{\mu}{R^3} = \left(\dfrac{\mu }{h^{3/2}} \equiv k \omega^{3/2} \Leftrightarrow k \equiv \dfrac{\mu} {h^{3/2}}
	\]
	%
	\indent and $h = \omega R^2$ is the chief's angular momentum. TO BE CONTINUED, MATRIX FORM OF HILL EQUATIONS. ALSO ADD VECTOR NOTATION OF EACH