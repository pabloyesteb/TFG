\chapter{The industrial problem}\label{chap:2}
The whole validation pipeline proposed in the next section is illustrated with realistic data from a real industrial problem of particular interest to the aerospace sector: stress analysis and failure prediction in aircraft fuselage structures. Results from statistical tests, graphs and tables are provided alongside the theoretical background at each step. This chapter is thus dedicated to introduce the industrial problem used for illustration of results.\\
%
\indent The industrial problem at hand is an application of ANNs in aircraft stress engineering. In aircraft engineering, the typical structural configuration of airplanes involves a semimonocoque design, characterized by a combination of internal framing and an external covering (known as ''skin''). This configuration has been used almost exclusively in the aviation industry since the II World War, owing to its ability to provide strength and durability while minimizing weight, which is crucial for aviation applications.\\
%
\indent The main elements of this structural configuration include (vid. \autoref{fig:semimono}):
\begin{itemize}
\item \textbf{Skin}: The outermost layer of the aircraft, forming the aerodynamic surface. It bears the aerodynamic loads and distributes them across the structure. The skin is typically made of lightweight yet robust materials such as aluminium or composite materials.
\item \textbf{Frames}: Internal structural components that run perpendicular to the longitudinal axis of the aircraft, providing shape and rigidity to the fuselage. Frames are spaced along the length of the fuselage and are connected to the skin, forming the structural framework.
\item \textbf{Stringers}: Longitudinal structural members that run parallel to the longitudinal axis of the aircraft. They are attached to the frames and provide additional reinforcement to the skin, helping to distribute loads and maintain structural integrity.
\end{itemize}

\begin{figure}
	\centering
	\includegraphics[width=0.8\textwidth]{Figures/semimonocoque.png}
	\caption{Typical fuselage section of an aeroplane with semi-monocoque configuration. The main structural elements are highlighted in the image. Image edited under Creative Commons License from the original \cite{Tosaka_Airframe_4_Types}.}
	\label{fig:semimono}
\end{figure}

Together, the skin, frames, and stringers form a robust yet lightweight structure that can withstand the stresses and loads experienced during flight. This configuration is essential for ensuring the safety and performance of aircraft.\\
\indent For typical aircraft fuselage panel design, the dominant form of stiffened post buckling failure under shear loading is forced crippling\cite{bijlaard1955buckling}. This occurs when the shear buckles in the panel skin force the attached stiffener (stringers) flanges to deform out-of-plane. There are other failure modes related to buckling, tension, and compression.\\
%
\indent The surrogate model presented here is the ''MS-S18'' model, whose aim is to predict Reserve Factors ($RF$) that quantify failure likelihood for regions of the aircraft subject to different loads happening in flight maneuvers (vid. \autoref{fig:diagrama_cajas}). There are six possible failure modes, whose names are ''Forced Crippling'', ''Column Buckling'', ''In Plane'', ''Net Tension'', ''Pure Compression'', and ''Shear Panel Failure''.\\
%
\indent Input data consists of $n=22$ features (loads applied to a specific region of the aircraft and different maneuver specs). This variables are either numerical continuous variables (the magnitudes of the applied loads) or categorical variables. This are variables which can only take a limited set of values (\eg boolean variables, integer variables, etc.) There are three categorical variables alongside 19 numerical ones in MS-S18's dataset. Their names are ''dp'', ''Frame'', and ''Stringer''\footnote{Variables' names are an arbitrary decision and completely irrelevant to this work. Nevertheless, they are provided here for illustration.}. They contain geometrical information about the region where the loads (whose magnitudes are described by the 19 numerical variables) are being applied. For instance, ''Frame'' and ''Stringer'' identify structural elements of the fuselage (their values are string variables like ''Fr64-Fr65'' and ''Str08'', referring to the interstitial region between fuselage frames no. 64 and no. 65 and the stringer no. 8, respectively).\\
%
\indent Output data consists of 6 reserve factors that quantify the stress failure likelihood:
$$
\mathbf{y}=(RF_1,RF_2,RF_3,RF_4,RF_5,RF_6)
$$
where $RF_i\in [0,5]$
In this (logarithmic) scale, $0$ means extreme risk and $5$ means risk extremely low.\\
%
\indent This study prioritizes assessing the efficacy of the developed validation tool, not optimizing the neural network model itself. Our focus remains solely on its inputs --the 22 aforementioned features-- and its outputs --the 6 reserve factors representing 6 distinct failure modes--. This simplification allows us to isolate and evaluate the performance of the validation tool without introducing confounding variables related to the specific neural network design (\ie, without explicitly specifying the network's architecture and its parameters, $W$).\\
%
\begin{figure}[!htb]
	\centering
	\begin{tikzpicture}
		% Caja 1
		\node[draw, minimum width=2cm, minimum height=3cm, font=\scriptsize] (box1) at (0,0) {
			\begin{tabular}{c}
				\textbf{INPUT $X$}\\
				Loads + Geometry +\\
			\end{tabular}
		};
		\node[below=0.1cm of box1] {$\mathbf{X}=(x_1,x_2,...,x_n)$};
		
		% Caja 2
		\node[draw, minimum width=2cm, minimum height=3cm, right=0.5cm of box1, font=\scriptsize] (box2) {
			\begin{tabular}{c}
				\textbf{SURROGATE MODEL}\\
				Stress learning model\\
				trained by splitting cases into\\
				train/test set and minimize loss\\
				function (Deep Neural Network)
			\end{tabular}
		};
		\node[below=0.1cm of box2] {$\cal{F}:\mathbf{X} \rightarrow \mathbf{Y}$};
		
		% Caja 3
		\node[draw, minimum width=2cm, minimum height=3cm, right=0.5cm of box2, font=\scriptsize] (box3) {
			\begin{tabular}{c}
				\textbf{PREDICTION (OUTPUT) Y}\\
				A set of Reserve Factors ($RF$)\\
				which characterize the likelihood\\
				of different failure modes.
			\end{tabular}
		};
		\node[below=0.1cm of box3] {$\mathbf{Y}=(y_1,y_2,...,y_m)$};
		
		% Flechas
		\draw[->, >=Stealth] (box1.east) -- (box2.west);
		\draw[->, >=Stealth] (box2.east) -- (box3.west);
	\end{tikzpicture}
	\caption{Surrogate model input-output data flux.}
	\label{fig:diagrama_cajas}
\end{figure}