\chapter{Relative dynamics around elliptic reference orbits.}
%
\label{chap: Eccentric}
%
\section{Introduction.}
%
\indent In the previous chapter, the Clohessy-Wiltshire set of equations of motion has been analyzed. Getting them from Newton's law required the fulfillment of two assumptions. Firstly, the distance between deputy and chief must be negligible compared to either spacecraft's orbital radius. This is usually the case when dealing both with formation flying and rendez-vous maneuvers. The second assumption is that both orbits are near-circular ($e<<1$), which is not as acceptable as the first one. This is specially relevant on formation flying, as the timescale is usually large enough to experience sensible deviations. \\
%
\indent Conceptually, there are some obvious differences between near-circular and eccentric orbits. First of all, orbital radius varies over time, which means that, at every point, the spacecraft is radially closing or moving away from the central body (\ie the Earth). But more importantly, the angular velocity is no longer constant, which means that the non-inertial effects when analyzing the relative motion are not constant. That will surely make it harder to get analytical expressions, though through some simplifications, it may be done.\\
%
\indent For this reason, several motion models for elliptic orbits have been developed \cite{Sullivan}. Both linear and nonlinear models are present in the literature, though the first ones are the most usually employed. Tschauner and Hempel \cite{Tschauner_Hempel} developed a linear, first-order model via the truncation of the Taylor series expansion of the differential gravity. The so-called Tschauner-Hempel equations were widely used at the time, as they are consistent with the Hill/Clohessy-Wiltshire (HCW) model. Nonetheless, they were subsequently improved, due to the existence of singularities in the in-plane motion. Carter \cite{Carter} provides a non-singular solution for this issue. \\
%
\indent Instead of these solutions, the motion model used for elliptic, unperturbed orbits in this thesis is one developed by Yamanaka and Ankersen \cite{Yamanaka_ankersen} (YA). It results in a fairly simpler STM, which is generally considered the state-of-the-art solution for linear propagation of the relative position and velocity in eccentric orbits. It will actually be used in the PROBA-3 mission, which flies in a highly elliptical orbit.\\
%
\indent During this chapter, we will firstly develop YA's approach for the motion model and their proposed solution for it. This enables in turn to develop the YA STM, which will be duly tested with their own scenarios. Lastly, orbit safety concerns will be approached, extending the prior knowledge from near-circular orbits to arbitrarily elliptical, as done by Peters and Noomen \cite{Peters_Noomen}.
%
\section{Motion model and STM.}
%
%
 	\subsection{Simplification of equations of motion: YA solution.}
	%
	\indent As developed in \ref{sec: Diff_eqs_prox}
	%
	\subsection{YA STM and integration constants.}
	%
	%
		\subsubsection{In-plane motion.}
		%
		%
		\subsubsection{Out-of-plane motion.}
		%
		%
		\subsubsection{Solution scheme.}
		%
		%
	\subsection{Results: Comparison with HCW and Hi-Fi propagation.}
	%
	%
\section{Orbit safety in eccentric, unperturbed orbits.}
%
%
	\subsection{Generalized eccentricity/inclination vectors separation.}
	%
	%
	\subsection{General trajectories and safe orbits.}
	%
	%
	
