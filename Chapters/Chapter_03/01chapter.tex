\chapter[Puedo poner otra cosa]{Example}
\label{chap:Example}

\minitoc




%----------------------------------------------------------------
%----------------------------------------------------------------
\section{Motivation}
\label{sec:101}

The increasing use of {\em Computational Fluid Dynamics} (CFD) in the aeronautical
industry allows to reduce the design cycle and thus the time to market, to optimize
and improve the quality of the product in terms of energy efficiency and contamination
reduction and, finally, to save a lot of money. The necessity to accurately predict
the global aerodynamic coefficients using CFD is a crucial question for the aeronautical
industry nowadays. Conventionally designed commercial airplanes are getting closer and
closer to the optimum design and the room for improvement is getting smaller and smaller.
Consequently, numerical optimization processes require an improvement of the accuracy of
the CFD tools used for aerodynamic prediction. For example, a drag coefficient prediction
within a ten drag counts band ($\pm 5\times 10^{-4}$) was defined as representative of the
desired numerical accuracy level by the EU aerospace industry within the EU project ADIGMA
\cite{Kroll_etal_2009} for several simple geometry test cases. Such drag accuracy may
translate to variations of $\pm 1.5\%$ in the total predicted drag for a typical long range
aircraft configuration. It has been shown \cite{Vassberg_etal_2003}, by using the simple
Breguet-range equation, that a $1\%$ under-prediction of the drag produces a $1\%$ shortfall
in range which may be recovered, assuming the same initial fuel weight, by an $8\%$ of payload
weight reduction. This translates, for such scenario, to a reduction of over 40 passengers
for a large transport aircraft. 

Sea $x \in \mathbb{R}$

\begin{figure}
     \begin{center}
           \includegraphics[width=0.485\textwidth]{Figures/chapter01/Lancelot.jpeg}
    \end{center}
\caption{Example of the base mesh.}
\label{fig:201}
\end{figure}


y otra





\begin{figure}
    \centering
      \begin{subfigure}{0.45\textwidth}
          \caption{}
        \includegraphics[width=\textwidth]{Figures/chapter01/Lancelot.jpeg}
          \label{fig:202a}
      \end{subfigure}
      \hfill
      \begin{subfigure}{0.45\textwidth}
          \caption{}
        \includegraphics[width=\textwidth]{Figures/chapter01/ReyArturo.jpeg}
          \label{fig:202b}
      \end{subfigure}
\caption{Two images}
\label{fig:202}
\end{figure}


El la figura \ref{fig:202} se les ve a todos y en la  \ref{fig:202b} se puede ver la mesa redonda al completo.

%Figure \ref{fig:NiceImage}
%Figure \ref{fig:NiceImage} \subref{fig:NiceImage1}
%\Cref{fig:NiceImage}
%\Cref{fig:NiceImage1}


adiosssss

\subsection{Prueba de minitoc}

Esto es un aprueba

\section{Ecuaciones numeradas }


Ecuaciones numeradas

\begin{equation}
x' = f(t,x)
\label{eq:01}
\end{equation}


Puedo hacer lo mismo, sin numerar

$$x' = f(t,x)$$


Puedo hacer lo mismo, $x' = f(t,x)$, pero dentro del texto


Varias  ecuaciones numeradas

\begin{equation}
\begin{array}{lcl}
x' = f(t,x) & \quad & \mbox{si } x \geq 0 \\ 
x'' = f(t,x) & \quad & \mbox{si } 0 < x \geq 10 \\ 
\label{eq:05}
\end{array}
\end{equation}

O tambén 

\begin{eqnarray}
x' = f(t,x)  \qquad \mbox{si } x \geq 0  \label{eq:10} \\ 
x' = f(t,x)  \qquad \mbox{si } x \geq 0  \nonumber \\ 
x'' = f(t,x) \qquad  \mbox{si } 0 < x \geq 10   \label{eq:11}  
\end{eqnarray}
Cambiar el tamaño en modo matemático



\vspace{1cm}  

\begin{table}[h] 
\Large
\begin{center} 
\begin{tabular}{l|c|r}
\hline
\hline
hola & $\frac{1}{2}$ & holaaaa \\
hola & $\frac{1}{2}$ & holaaaa \\
hola & $\frac{1}{2}$ & holaaaa \\[5mm]
hola & $\frac{1}{2}$ & holaaaa \\
\hline
hola & $\frac{1}{2}$ & holaaaa \\
\hline
\end{tabular}
\end{center}
\caption{Texto que quieras}
\label{tab:11}
\end{table}


ppppp

\Large
Hola

\normalsize
Hola