\chapter{Conclusions and research outlook}

This work presents a validation pipeline tailored to address the challenges inherent in certifying neural network models across various industrial domains. While the focus has been emphasized on its applications within the aerospace industry, it's crucial to underscore the pipeline's inherent versatility and adaptability to a wide range of industrial problems and solution methodologies.

\textbf{Versatility Across Industries}: Although the validation pipeline showcased in this work has been exemplified within the aerospace sector, its principles and methodologies are applicable to diverse industrial contexts. The structured approach to addressing regression problems with neural networks offers a universal framework that can be adapted to different domains.

\textbf{Practical Utility}: Through the practical application of the validation pipeline using an industrial surrogate model, this work demonstrates its efficacy in addressing real-world challenges. By illustrating its application to structural calculation and failure mode prediction in aeronautical structures, the pipeline's potential to enhance safety and efficiency becomes apparent, regardless of the specific industry context.

\textbf{Flexibility and Scalability}: Chapter \ref{chap:3} elucidates the comprehensive nature of the validation pipeline, emphasizing its flexibility and scalability to accommodate diverse problem domains and evolving technological landscapes. By providing a detailed examination of each component's role, the pipeline offers a customizable framework that can be tailored to meet the specific needs and challenges of different industries.

\textbf{Future Directions}: The potential applications of the validation pipeline presented here are vast and promising, extending beyond the confines of the aerospace industry. Future developments could explore leveraging the capabilities of physics-informed neural networks or PINNs\cite{raissi2019deep}. Furthermore, there is an opportunity to explore direct prediction of aerodynamic loads, showcasing the versatility of the pipeline in tackling complex multidimensional problems across various industrial sectors.

While this work primarily focuses on supervised learning paradigms, the scope can be expanded to encompass other learning methodologies. For instance, avenues for exploration include the application of reinforcement learning\cite{kaelbling1996reinforcement} techniques to refine and optimize decision-making processes within the validation pipeline. Additionally, the incorporation of collective learning approaches holds promise for leveraging distributed intelligence and collaborative problem-solving across diverse datasets and domains.

In essence, the future development and application of the validation pipeline are poised to unlock new frontiers in industrial innovation and efficiency. By embracing emerging technologies and methodologies, such as PINNs and reinforcement learning, and fostering collaboration across disciplines, researchers and industry practitioners can continue to push the boundaries of what is possible in the realm of neural network validation and application.


In conclusion, this  advocates for the widespread adoption of rigorous validation methodologies and the responsible utilization of neural networks across industrial domains. By embracing the principles of the proposed validation pipeline and fostering collaboration among industry stakeholders, organizations can ensure the reliability, safety, and transformative potential of neural network-driven systems in an ever-evolving industrial landscape.